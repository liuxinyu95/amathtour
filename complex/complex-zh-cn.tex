\ifx\wholebook\relax \else

\documentclass[b5paper]{ctexart}
\usepackage[nomarginpar
  %, margin=.5in
]{geometry}

\addtolength{\oddsidemargin}{-0.05in}
\addtolength{\evensidemargin}{-0.05in}
\addtolength{\textwidth}{0.1in}

\usepackage[cn]{../prelude}

\setcounter{page}{1}

\begin{document}

\title{复数}

\author{刘新宇
\thanks{{\bfseries 刘新宇} \newline
  Email: liuxinyu99@hotmail.com \newline}
  }

\maketitle
\fi

\markboth{复数}{数的旅程}

\ifx\wholebook\relax
\chapter{复数}
\fi

%% On earth there is nothing great but man; and in man there is nothing great but mind.
\epigraph{地球上最伟大的是人,人之中最伟大的是心灵}{威廉·汉密尔顿}

读过金庸武侠小说的读者一定对“华山论剑”津津乐道。天下武林中的顶级高手约定在华山比武论剑。大家各自身怀秘不外传的绝世武功,通过比武确定谁是真正的天下武功第一。十六世纪的意大利也上演了精彩的华山论剑。只不过没有刀光剑影、没有拳脚身法,而是一场关于数学与荣誉的挑战。1535年2月13日深夜,塔尔塔利亚在威尼斯的家中踱来踱去,为即将到来的挑战苦思冥想。文艺复兴时期的意大利,学着间经常进行骑士般的公开挑战。这种挑战并不使用刀剑或者火枪进行决斗,双方各自向对方提出同等数目的数学题目,并在规定的时间内作答。正确答出多的人获胜。胜利的一方赢得荣誉或一定数目的奖金。这种一对一的“单挑”发生在知名学者之间,通常在大教堂这类城市公共场所进行,引起众人的围观。有时甚至市长或者贵族也会前来观看,因而使得胜负具有非常的意义,影响双方的社会地位、学术评价甚至职业发展。

例如1225年,神圣罗马帝国皇帝腓特烈二世在西西里行宫接见了斐波那契。宫廷学者看不起商人出身的数学家,于是向斐波那契发起了挑战。斐波那契成功解决了全部问题,为自己赢得了荣誉。这些题目中包含了一道一元三次方程\footnote{当时还没有代数符号,是用文字描述的,如:找到一个数,其立方加上其平方的两倍再加上其十倍是二十。}:\[x^3 + 2x^2 + 10x = 20\]

斐波那契使用古巴比伦的60进制数值解法给出了正确答案:
\[
1^022^{I}7^{II}42^{III}33^{IV}4^{V}40^{VI} = 1 + \frac{22}{60} + \frac{7}{60^2} + \frac{42}{60^3} + \dotsb
\]

相当于十进制小数1.3688081075,精确到了小数点后9位。这种公开挑战也造成了一种奇特现象:学者们把自己的研究成果视为高度机密。因为这样可以使得他在与别人的挑战中获得优势。一旦泄露,对方就能解出自己提出的问题。学者们在生前不发表自己的成果,而是在死前传给自己信任的门生。这在“不发表就发霉”的今天是很难理解的\cite{HanXueTao2012}。

塔尔塔利亚面临的挑战就是关于三次方程的,对手名叫费奥尔\footnote{安东尼奥・马里亚・费奥尔,Antonio Maria Fior}。尽管古巴比伦人在公元前2500左右就掌握了一元二次方程的解法,但人们在接下来的4000多年一直没有突破一般三次方程的解法。人们并不满足于斐波那契的数值解法,而希望得到如二次方程那样的求根公式。所谓\underdot{一般}是指形如$ax^3 + bx^2 + cx + d = 0$的三次方程,简单的特殊三次方程,如$x^3 = a$自然容易解出一个\footnote{另外两个根是复数:$\dfrac{-1 \pm i\sqrt{3}}{2}\sqrt[3]{a}$}根$x = \sqrt[3]{a}$。塔尔塔利亚经过自己的努力,独立发现了形如$x^3 + ax^2 = b$这种特殊三次方程的解。

但费奥尔不断向世人吹嘘他会解三次方程,是意大利最好的学者,并向塔尔塔利亚发起挑战。双方约定各出30道题目,用火漆封好,在教堂决出谁更优秀。塔尔塔利亚心中忐忑不安。从费奥尔向别人炫耀的题目中,他隐隐感觉到费奥尔所能解出的三次方程不是$x^3 + ax^2 = b$,而是另一种形式:$x^3 + ax = b$,例如:“找到一个数,把它的立方加到自身上等于6”(相当于$x^3 + x = 6$)。但塔尔塔利亚却不知道如何解出这种方程\cite{MacTour-Tartaglia-Cardan}。他为此辗转反侧,茶饭不思。他思绪万千,童年时的一幕幕不断闪现在脑海中。

\index{数学家!塔尔塔利亚}
塔尔塔利亚原名尼科洛・丰塔纳,1499年或1500年,他出生于意大利的布雷西亚。父亲米凯莱·丰塔纳是一名邮递员,在布雷西亚周边的山区送邮件。家中有两男一女三个孩子,生活贫苦。尼科洛6岁时,父亲在送信的路上被人谋杀。失去了顶梁柱,全家陷入了极度贫困。更可怕的灾难还在后面。1512年,法军进攻布雷西亚\footnote{1494~1559年爆发了意大利战争。法国国王路易十二企图征服意大利,这遭到了西班牙哈布斯堡王朝等欧洲列强的反对,引发了战争。},为了报复布雷西亚人的坚决抵抗,法军在攻占后杀死了4.6万人。母亲带着尼科洛和妹妹想躲进教堂避难,但尼科洛还是在混乱中被一名法国士兵砍伤了面部,嘴巴上有两道触目惊心的伤痕。母亲找到了奄奄一息的孩子,但却没有钱请医生,她只好每天给他舔舐伤口。尼科洛奇迹般地活了下来,但是却因伤终身口吃。为此他得到了一个外号“小结巴”,塔尔塔利亚就是意大利语结巴的意思。他年长后一直留着胡子以遮掩脸上的伤疤。

尽管遭遇了诸多不幸,塔尔塔利亚却展现出了学习的天赋。他基本靠自学,买不起纸就用墓碑当作石板来代替。后来母亲终于找到了一位好心人资助他去帕多瓦学习。学成后塔尔塔利亚在维罗纳成了一名小学数学教师。1534年他搬到威尼斯,在圣扎诺波洛教堂教数学。此后他通过赢得一系列的公开挑战越来越有名气\cite{MacTour-Tartaglia}。往事如烟,现在塔尔塔利亚必须集中精神,尽快解决不含有二次项的三次方程。这个晚上,奇迹发生了,他终于找到了解法。当双方撕开火漆密封的题目,其实胜负已分:塔尔塔利亚的30道题目包含了两种特殊的三次方程,既有不含一次项的,也有不含二次项的;而费奥尔的只有不含二次项的一种。塔尔塔利亚只用了两个小时就正确解出了全部题目,但费奥尔被不含一次项的方程困住了。

\index{数额家!卡尔达诺}
塔尔塔利亚取得了胜利,他拒绝了奖金,只接受了荣誉。这次“华山论剑”不胫而走,传到了意大利米兰,引起了一位名叫卡尔达诺\footnote{拉丁文Cardano,也有人根据英文Cardan译作卡尔丹。}的人的注意。吉罗拉莫·卡尔达诺(1501~1576)是一个私生子。他的父亲法齐奥·卡尔达诺是米兰的一位律师,同时也是数学家,在帕维亚大学和米兰大学任教。据说达·芬奇曾向法齐奥请教几何问题。

\begin{figure}[htbp]
  \centering
  \includegraphics[scale=0.33]{img/churchmilan}
  \caption{米兰圣母玛丽亚感恩大教堂}
 \label{fig:church-Milan}
\end{figure}

由于私生子的身份,卡尔达诺生活贫困,体弱多病。经过争取,法齐奥同意送他去帕维亚大学学习医学。意大利战争期间,学校被迫关闭,卡尔达诺只好去帕多瓦大学完成学业。他的父亲不久也去世了。在乱世中,卡尔达诺很快花光了父亲的一小笔遗产,并且染上了赌博的恶习。他很有数学头脑,逐渐悟出了概率原理(一个世纪后帕斯卡和费马才创立概率论),因此经常击败其他赌徒。卡尔达诺争强斗狠,当他怀疑对方出老千时,会毫不犹豫拔出匕首斗殴。这使得他臭名昭著,以至于1525年获得医学博士后,米兰市拒绝向卡尔达诺颁发行医许可。他只得一边私下偷偷行医一边继续赌博。很快输光了老婆的嫁妆和家当,全家不得不搬到米兰的救济院。走投无路的卡尔达诺从父亲生前的大学获得了一个数学教职。他的医学才能逐渐显示了出来,治好了很多名人的疾病,米兰总督甚至当地的医生也私下找他看病。终于在1539年他迎来了人生转机:米兰市迫于卡尔达诺治好的那些著名患者的压力,向他颁发了行医执照。对于卡尔达诺的私生子身份,米兰医学会认为法齐奥最终与其生母结婚,因此出身“合法”。卡尔达诺春风得意,在这一年出版了两部数学书。此外他还做一件事:在过去四年,他尝试自己找出塔尔塔利亚解三次方程的方法,但是失败了。他通过一位往返米兰和威尼斯之间的书商询问塔尔塔利亚,是否可以告知三次方程的解法,卡尔达诺许诺把塔尔塔利亚的结果加入他即将出版的数学书中。但他等来的是冷冰冰的拒绝。塔尔塔利亚说他自己将来会把三次方程的解法著书出版。卡尔达诺不死心,再次托人询问能否告知具体解法并承诺保密。但还是被拒绝了。

于是赌场老江湖卡尔达诺亲自给塔尔塔利亚写了一封信,措辞技巧极高。一方面,他责怪塔尔塔利亚不识好歹,暗示要与他进行一场公开挑战;另一方面,说他和自己治好的患者——米兰总督阿方索·德·阿瓦洛斯,瓦斯托侯爵——讨论了塔尔塔利亚的才能……塔尔塔利亚上钩了。这位出身卑微的小学教师渴望更高的社会地位。他给卡尔达诺回信,询问能否介绍他和总督认识,并向总督亲自展示自己的才能。卡尔达诺于是邀请塔尔塔利亚拜访他米兰的住所,并许诺安排和德·阿瓦洛斯侯爵见面。1539年3月,塔尔塔利亚走进了卡尔达诺的大门。主人殷勤招待,有求必应。但有个小小的遗憾:侯爵大人临时有事,事发突然,不能赴约。在卡尔达诺一碗一碗的迷魂汤下,塔尔塔利亚终于答应吐露三次方程的解法。但他要求卡尔达诺发下重誓:第一,至死不向任何人透露;第二,只能用暗语记录以防别人识破。然后他读出了一首诗\footnote{我们省略了中间部分。塔尔塔利亚的诗包含三种特殊的三次方程,并指出第三种可以转化为第二种。}:

\begin{verse}
当立方加上一些事物,\\
等于某个整数,\\
找出两个数,其差与此数相同,\\
此后你要按惯例考虑,\\
它们的乘积始终等于,\\
这些事物的立方的三分之一,\\
那么一般的余数,\\
从它们的立方根中准确地减去,\\
将是未知量的值。\\
…… \\
我发现这些,并非缓慢达成,\\
在一千五百三十四年,\\
有着非常坚实和稳固的基础,\\
在那座被大海环绕的城市。
\end{verse}

屋子里鸦雀无声,除了卡尔达诺和塔尔塔利亚,只有一个十八岁的年轻仆人,名叫费拉里。

%% del Ferro:德尔・费罗(意大利数学家尼科洛・德尔・费罗,Niccolò del Ferro,15 世纪末至 16 世纪初,首次解出一元三次方程的一类特殊情况)
%% Ferrari:费拉里(意大利数学家洛多维科・费拉里,Lodovico Ferrari,16 世纪,在卡尔达诺指导下首次解出一元四次方程)

\section{三次方程}

\begin{figure}[htbp]
  \centering
  \includegraphics[scale=0.33]{img/plimpton322}
  \caption{编号普林普顿322的古巴比伦泥板,内容为勾股数表,约公元前2020年。藏于哥伦比亚大学。二十世纪二十年代由乔治·亚瑟、普林普顿捐赠给哥伦比亚大学。}
 \label{fig:plimpton322}
\end{figure}

解三次方程的历史

\section{佚名数学家}
\subsection{复数的几何意义和运算法则}
\subsection{匿名作者阿尔冈}
\subsection{代数基本定理}

\section{e的传奇}
\subsection{e的诞生}
\subsection{最美公式}

\section{新世界的大门}
\subsection{小试牛刀}
麦钦公式、正五边形作图
\subsection{费马大定理与高斯整数}
\subsection{抽象的数}
理想数
\subsection{新数的构造-代数数}

\ifx\wholebook\relax \else
\section{参考答案}
\shipoutAnswer

\begin{thebibliography}{99}
\subimport{inc/}{bib-zh-cn}
\end{thebibliography}

\expandafter\enddocument
\fi
