\ifx\wholebook\relax \else

\documentclass[b5paper]{ctexart}
\usepackage[nomarginpar
  %, margin=.5in
]{geometry}

\addtolength{\oddsidemargin}{-0.05in}
\addtolength{\evensidemargin}{-0.05in}
\addtolength{\textwidth}{0.1in}

\usepackage[cn]{../prelude}

\setcounter{page}{1}

\begin{document}

\title{复数}

\author{刘新宇
\thanks{{\bfseries 刘新宇} \newline
  Email: liuxinyu99@hotmail.com \newline}
  }

\maketitle
\fi

\markboth{复数}{数的旅程}

\ifx\wholebook\relax
\chapter{复数}
\fi

%% On earth there is nothing great but man; and in man there is nothing great but mind.
\epigraph{地球上最伟大的是人,人之中最伟大的是心灵}{威廉·汉密尔顿}

读过金庸武侠小说的读者一定对“华山论剑”津津乐道。天下武林中的顶级高手约定在华山比武论剑。大家各自身怀秘不外传的绝世武功,通过比武确定谁是真正的天下武功第一。十六世纪的意大利也上演了精彩的华山论剑。只不过没有刀光剑影、没有拳脚身法,而是一场关于数学与荣誉的挑战。1535年2月13日深夜,塔尔塔利亚在威尼斯的家中踱来踱去,为即将到来的挑战苦思冥想。文艺复兴时期的意大利,学着间经常进行骑士般的公开挑战。这种挑战并不使用刀剑或者火枪进行决斗,双方各自向对方提出同等数目的数学题目,并在规定的时间内作答。正确答出多的人获胜。胜利的一方赢得荣誉或一定数目的奖金。这种一对一的“单挑”发生在知名学者之间,通常在大教堂这类城市公共场所进行,引起众人的围观。有时甚至市长或者贵族也会前来观看,因而使得胜负具有非常的意义,影响双方的社会地位、学术评价甚至职业发展。

例如1225年,神圣罗马帝国皇帝腓特烈二世在西西里行宫接见了斐波那契。宫廷学者看不起商人出身的数学家,于是向斐波那契发起了挑战。斐波那契成功解决了全部问题,为自己赢得了荣誉。这些题目中包含了一道一元三次方程\footnote{当时还没有代数符号,是用文字描述的,如:找到一个数,其立方加上其平方的两倍再加上其十倍是二十。}:\[x^3 + 2x^2 + 10x = 20\]

斐波那契使用古巴比伦的60进制数值解法给出了正确答案:
\[
1^022^{I}7^{II}42^{III}33^{IV}4^{V}40^{VI} = 1 + \frac{22}{60} + \frac{7}{60^2} + \frac{42}{60^3} + \dotsb
\]

相当于十进制小数1.3688081075,精确到了小数点后9位。这种公开挑战也造成了一种奇特现象:学者们把自己的研究成果视为高度机密。因为这样可以使得他在与别人的挑战中获得优势。一旦泄露,对方就能解出自己提出的问题。学者们在生前不发表自己的成果,而是在死前传给自己信任的门生。这在“不发表就发霉”的今天是很难理解的\cite{HanXueTao2012}。

塔尔塔利亚面临的挑战就是关于三次方程的,对手名叫费奥尔\footnote{安东尼奥・马里亚・费奥尔,Antonio Maria Fior}。尽管古巴比伦人在公元前2000左右就掌握了一元二次方程的解法,但人们在接下来的4000多年一直没有突破一般三次方程的解法。人们并不满足于斐波那契的数值解法,而希望得到如二次方程那样的求根公式。所谓\underdot{一般}是指形如$ax^3 + bx^2 + cx + d = 0$的三次方程,简单的特殊三次方程,如$x^3 = a$自然容易解出一个\footnote{另外两个根是复数:$\dfrac{-1 \pm i\sqrt{3}}{2}\sqrt[3]{a}$}根$x = \sqrt[3]{a}$。塔尔塔利亚经过自己的努力,独立发现了形如$x^3 + ax^2 = b$这种特殊三次方程的解。

但费奥尔不断向世人吹嘘他会解三次方程,是意大利最好的学者,并向塔尔塔利亚发起挑战。双方约定各出30道题目,用火漆封好,在教堂决出谁更优秀。塔尔塔利亚心中忐忑不安。从费奥尔向别人炫耀的题目中,他隐隐感觉到费奥尔所能解出的三次方程不是$x^3 + ax^2 = b$,而是另一种形式:$x^3 + ax = b$,例如:“找到一个数,把它的立方加到自身上等于6”(相当于$x^3 + x = 6$)。但塔尔塔利亚却不知道如何解出这种方程\cite{MacTour-Tartaglia-Cardan}。他为此辗转反侧,茶饭不思。他思绪万千,童年时的一幕幕不断闪现在脑海中。

\index{数学家!塔尔塔利亚}
塔尔塔利亚原名尼科洛・丰塔纳,1499年或1500年,他出生于意大利的布雷西亚。父亲米凯莱·丰塔纳是一名邮递员,在布雷西亚周边的山区送邮件。家中有两男一女三个孩子,生活贫苦。尼科洛6岁时,父亲在送信的路上被人谋杀。失去了顶梁柱,全家陷入了极度贫困。更可怕的灾难还在后面。1512年,法军进攻布雷西亚\footnote{1494~1559年爆发了意大利战争。法国国王路易十二企图征服意大利,这遭到了西班牙哈布斯堡王朝等欧洲列强的反对,引发了战争。},为了报复布雷西亚人的坚决抵抗,法军在攻占后杀死了4.6万人。母亲带着尼科洛和妹妹想躲进教堂避难,但尼科洛还是在混乱中被一名法国士兵砍伤了面部,嘴巴上有两道触目惊心的伤痕。母亲找到了奄奄一息的孩子,但却没有钱请医生,她只好每天给他舔舐伤口。尼科洛奇迹般地活了下来,但是却因伤终身口吃。为此他得到了一个外号“小结巴”,塔尔塔利亚就是意大利语结巴的意思。他年长后一直留着胡子以遮掩脸上的伤疤。

尽管遭遇了诸多不幸,塔尔塔利亚却展现出了学习的天赋。他基本靠自学,买不起纸就用墓碑当作石板来代替。后来母亲终于找到了一位好心人资助他去帕多瓦学习。学成后塔尔塔利亚在维罗纳成了一名小学数学教师。1534年他搬到威尼斯,在圣扎诺波洛教堂教数学。此后他通过赢得一系列的公开挑战越来越有名气\cite{MacTour-Tartaglia}。往事如烟,现在塔尔塔利亚必须集中精神,尽快解决不含有二次项的三次方程。这个晚上,奇迹发生了,他终于找到了解法。当双方撕开火漆密封的题目,其实胜负已分:塔尔塔利亚的30道题目包含了两种特殊的三次方程,既有不含一次项的,也有不含二次项的;而费奥尔的只有不含二次项的一种。塔尔塔利亚只用了两个小时就正确解出了全部题目,但费奥尔被不含一次项的方程困住了。

\index{数学家!卡尔达诺}
塔尔塔利亚取得了胜利,他拒绝了奖金,只接受了荣誉。这次“华山论剑”不胫而走,传到了意大利米兰,引起了一位名叫卡尔达诺\footnote{拉丁文Cardano,也有人根据英文Cardan译作卡尔丹。}的人的注意。吉罗拉莫·卡尔达诺(1501~1576)是一个私生子。他的父亲法齐奥·卡尔达诺是米兰的一位律师,同时也是数学家,在帕维亚大学和米兰大学任教。据说达·芬奇曾向法齐奥请教几何问题。

由于私生子的身份,卡尔达诺生活贫困,体弱多病。经过争取,法齐奥同意送他去帕维亚大学学习医学。意大利战争期间,学校被迫关闭,卡尔达诺只好去帕多瓦大学完成学业。他的父亲不久也去世了。在乱世中,卡尔达诺很快花光了父亲的一小笔遗产,并且染上了赌博的恶习。他很有数学头脑,逐渐悟出了概率原理(一个世纪后帕斯卡和费马才创立概率论),因此经常击败其他赌徒。卡尔达诺争强斗狠,当他怀疑对方出老千时,会毫不犹豫拔出匕首斗殴。这使得他臭名昭著,以至于1525年获得医学博士后,米兰市拒绝向卡尔达诺颁发行医许可。他只得一边私下偷偷行医一边继续赌博。很快输光了老婆的嫁妆和家当,全家不得不搬到米兰的救济院。走投无路的卡尔达诺从父亲生前的大学获得了一个数学教职。他的医学才能逐渐显示了出来,治好了很多名人的疾病,米兰总督甚至当地的医生也私下找他看病。终于在1539年他迎来了人生转机:米兰市迫于卡尔达诺治好的那些著名患者的压力,向他颁发了行医执照。对于卡尔达诺的私生子身份,米兰医学会认为法齐奥最终与其生母结婚,因此出身“合法”。卡尔达诺春风得意,在这一年出版了两部数学书。此外他还做一件事:在过去四年,他尝试自己找出塔尔塔利亚解三次方程的方法,但是失败了。他通过一位往返米兰和威尼斯之间的书商询问塔尔塔利亚,是否可以告知三次方程的解法,卡尔达诺许诺把塔尔塔利亚的结果加入他即将出版的数学书中。但他等来的是冷冰冰的拒绝。塔尔塔利亚说他自己将来会把三次方程的解法著书出版。卡尔达诺不死心,再次托人询问能否告知具体解法并承诺保密。但还是被拒绝了。

于是赌场老江湖卡尔达诺亲自给塔尔塔利亚写了一封信,措辞技巧极高。一方面,他责怪塔尔塔利亚不识好歹,暗示要与他进行一场公开挑战;另一方面,说他和自己治好的患者——米兰总督阿方索·德·阿瓦洛斯,瓦斯托侯爵——讨论了塔尔塔利亚的才能……塔尔塔利亚上钩了。这位出身卑微的小学教师渴望更高的社会地位。他给卡尔达诺回信,询问能否介绍他和总督认识,并向总督亲自展示自己的才能。卡尔达诺于是邀请塔尔塔利亚拜访他米兰的住所,并许诺安排和德·阿瓦洛斯侯爵见面。1539年3月,塔尔塔利亚走进了卡尔达诺的大门。主人殷勤招待,有求必应。但有个小小的遗憾:侯爵大人临时有事,事发突然,不能赴约。在卡尔达诺一碗一碗的迷魂汤下,塔尔塔利亚终于答应吐露三次方程的解法。但他要求卡尔达诺发下重誓:第一,至死不向任何人透露;第二,只能用暗语记录以防别人识破。然后他读出了一首诗\footnote{我们省略了中间部分。塔尔塔利亚的诗包含三种特殊的三次方程,并指出第三种可以转化为第二种。}:

\begin{verse}
当立方加上一些事物,\\
等于某个整数,\\
找出两个数,其差与此数相同,\\
此后你要按惯例考虑,\\
它们的乘积始终等于,\\
这些事物的立方的三分之一,\\
那么一般的余数,\\
从它们的立方根中准确地减去,\\
将是未知量的值。\\
…… \\
我发现这些,并非缓慢达成,\\
在一千五百三十四年,\\
有着非常坚实和稳固的基础,\\
在那座被大海环绕的城市。
\end{verse}

屋子里鸦雀无声,除了卡尔达诺和塔尔塔利亚,只有一个十八岁的年轻仆人,名叫费拉里(Lodovico Ferrari)。塔尔塔利亚不记得自己是怎样走出卡尔达诺家的大门的。他手里拿着一封卡尔达诺给公爵的介绍信。一阵风吹来,他打了个寒战。他没有继续留在米兰求见公爵,而是匆匆返回了威尼斯。他后悔了。这一年卡尔达诺出版了两本书,塔尔塔利亚逐一检查了每一本,内容是关于赌博概率的计算以及数学讲义。卡尔达诺似乎信守了诺言,没有只字透露三次方程的解法。

卡尔达诺和他的仆人费拉里彻底研究了塔尔塔利亚的解法,并继续解决了所有形式的三次方程。本质上,主仆二人找到了一般三次方程的解法。并且费拉里在主人的指导下一举攻克了四次方程。但毕竟发过重誓,他们不能发表这些结果,而只能把这一秘密用于和别人“华山论剑”。转机发生在1543年,卡尔达诺和费拉里前往博洛尼亚旅行。从一位名叫德拉·纳韦的人那里得知,在很早以前当地的学者德尔·费罗(Scipione del Ferro,1465~1926)就解出了三次方程。费罗从1496年起任数学教授,而1535年被塔尔塔利亚击败的费奥尔正是费罗的学生!卡尔达诺立刻意识到,既然第一个解出三次方程的人是费罗而非塔尔塔利亚,他可以把誓言丢到九霄云外了。1545年,卡尔达诺出版了数学史上意义非凡的一本书《大术》(Ars magna),公开了完整的三次和四次方程的解法,并在第8页写道:

\begin{quotation}
在我们这个时代,博洛尼亚的德尔·费罗解决了无二次项的特殊三次方程,这是一项极为优雅且令人钦佩的成就。由于这门技艺超越了人类所有的精妙构思以及凡人天赋的敏锐洞察力,它是一种真正来自上天的恩赐,也是对人类思维能力的一次明确考验,任何致力于此的人都会相信,没有什么是他无法理解的。受他的激励,我的朋友,布雷西亚的尼科洛·塔尔塔利亚,不甘示弱,在与费罗的学生费奥尔的一场竞赛中,也解决了同样的问题。在我多次恳请下,他将解法告诉了我……得到塔尔塔利亚的解法后,我寻求它的证明,进而明白还有许多相关内容有待发掘。秉持着这个想法并满怀信心,我陆续发现了这些成果。它们部分是我独自完成的,部分来自我的学生费拉里。
\end{quotation}

卡尔达诺声名鹊起,但塔尔塔利亚被激怒了。他觉得卡尔达诺背叛了誓言,泄露了秘密。第二年,塔尔塔利亚出版了一本小册子《新问题新发明》,试图给出故事的真想,揭露卡尔达诺背信弃义的恶行。但卡尔达诺再次展现了老江湖的斗争策略。他退到幕后保持沉默,而放任仆人费拉里公开和塔尔塔利亚唇枪舌剑、互相攻击。这在外人看来,小学教师塔尔塔利亚似乎只能和卡尔达诺的仆人论战,因而和仆人在同一水平,而低于著名学者、数学家、名医卡尔达诺。费拉里公开向塔尔塔利亚发起挑战,但塔尔塔利亚真正想挑战的人是卡尔达诺,但后者始终对他不予理睬。塔尔塔利亚和费拉里互相谩骂攻击了一年后,发生了戏剧性的一幕。布雷西亚大学邀请塔尔塔利亚回家乡任教,但条件是他必须去米兰参加和费拉里的“华山论剑”。于是1548年8月10日,第二次华山论剑在米兰的圣母玛利亚感恩大教堂(见\cref{fig:church-Milan})上演了。日落时,费拉里占尽了优势,他不仅掌握了三次方程的解法,还用四次方程挑战对手。塔尔塔利亚在当晚不辞而别,他输了。

\begin{figure}[htbp]
  \centering
  \includegraphics[scale=0.33]{img/churchmilan}
  \caption{米兰圣母玛丽亚感恩大教堂}
 \label{fig:church-Milan}
\end{figure}

塔尔塔利亚失去了布雷西亚的教授工作,他在贫穷和愤恨中走完了自己的一生。卡尔达诺的誓言似乎冥冥中笼罩着他。1560年他的长子被控毒杀不忠的妻子被判处死刑。卡尔达诺彻底被丧子之痛击倒了,他从米兰搬到布雷西亚住。1565年,卡尔达诺再次遭受打击,得意门生费拉里被他姐姐毒杀。卡尔丹诺沉迷赌博和占星术,他因推算出了耶稣的星象,被教会判处大逆不道,于1570年锒铛入狱。出狱后他移居罗马,在生命的最后一年出版了一本自传。在自传中,他不断重复丧子之痛,吹嘘自己的成就,但只用一句话轻描淡写地谈到了塔尔塔利亚。我们今天对这段历史的还原主要来自塔尔塔利亚的《新问题新发明》,以及他和中间人书商、卡尔达诺、费拉里往来的信件。

\begin{figure}[htbp]
 \centering
 \subcaptionbox{塔尔塔利亚\label{fig:tartaglia}}{\includegraphics[scale=0.35]{img/tartaglia-coin}}
 \subcaptionbox{卡尔达诺\label{fig:cardano}}{\includegraphics[scale=0.3]{img/Cardano}}
 \caption{塔尔塔利亚与卡尔达诺}
\end{figure}

今天,人们把三次方程的求根公式叫做卡尔达诺——塔尔塔利亚公式,简称卡尔达诺公式。此外塔尔塔利亚还在1546年发表过弹道学著作,第一个给出了弹道表格。1543年他第一个把欧几里得《原本》翻译成意大利语出版,此外他把阿基米德的一些著作翻译成意大利语。除了明显的缺陷外,卡尔达诺是一名百科全书式的学者。他是第一个对斑疹伤寒做出临床描述的人,数学上的贡献涵盖代数、概率论。他还发明了万向轴,被称作卡尔达诺轴。莱布尼茨评价到:“尽管有缺点,他是个伟人;但如果没有这些缺点,他将是无与伦比的。”

\section{三次方程}

与通常的观点不同,复数\underdot{不是}在解二次方程的过程中,而是在解三次方程的过程中被意外发现的。高中数学课上会给出卡尔达诺公式。本节将利用代数与几何两种方法推导这个公式。

\begin{figure}[htbp]
  \centering
  \includegraphics[scale=0.33]{img/plimpton322}
  \caption{编号普林普顿322的古巴比伦泥板,内容为勾股数表,约公元前2020年。藏于哥伦比亚大学。二十世纪二十年代由乔治·亚瑟、普林普顿捐赠给哥伦比亚大学。}
 \label{fig:plimpton322}
\end{figure}

%% https://personal.math.ubc.ca/~cass/courses/m446-03/pl322/pl322.html
我们是从古巴比伦泥板上了解到二次方程的早期历史的。如\cref{fig:plimpton322}所示,这是藏于哥伦比亚大学的一块泥板,编号为普林普顿322。这块泥板上的楔形文字内容很有特点,德国科学史家诺伊格鲍尔和萨克斯将发现它竟然是一张勾股数表。根据第1章给出的古巴比伦计数系统,我们很容易看出最后一列是编号$1, 2, 3, \dotsc, 15$,其中5,6,15三个数字损毁了。同样右边第二列也是编号。左边第二列的表头注明是直角三角形的底,从上到下转换为十进制依次是:$119, 3367, 4601, \dotsc, 1771, 56$,其中最后的56损毁了,并且有两处错误。左边第三列的表头注明是直角三角形的斜边,从上到下转换成十进制依次为:$169, 4825, 6649, \dotsc, 3229, 106$,同样最后的数字有错并且损毁了。左边第一列的数字最难破译,古巴比伦此时还没有使用零,也没有小数点。科学家们将其破译并转换为十进制数字,依次为:$1.9834\dotso, 1.94916\dotso, 1.9188\dotso$直到$1.43024\dotso, 1.38716\dotso$。60进制小数转换为10进制通常会除不尽(见\cref{thm:finite-decimal}),因此我们增加了省略号。如果我们记直角三角形的底为$w$,斜边为$d$,高为$h$,科学家们发现第一列的数字是$\dfrac{d^2}{h^2}$,根据勾股定理它等于$\dfrac{d^2}{d^2 - w^2}$。例如第一行:$d = 169, w = 119$,计算可得:

\[
\frac{d^2}{d^2 - w^2} = \frac{169^2}{169^2 - 119^2} = 1.9843\dotso
\]

%% https://engelsbergideas.com/notebook/the-brilliance-of-babylonian-mathematics/
正好是第一行中的第一个数字。这个勾股数表证实古巴比伦人知晓勾股定理(尽管没有定理的证明),并将其应用于计算。随着更多楔形文字泥板的出土,科学家们还发现了勾股定理的“配套练习题”。其中一道题目说:“已知长方形的面积和对角线,求长方形的两个边长。”这道典型的土地丈量问题相当于解方程:

\[
\begin{cases}
  ab = s \\
  a^2 + b^2 = d^2
\end{cases}
\]

其中$s$是面积,$d$是对角线。将$b = \dfrac{s}{a}$代入第二式,就得到关于$a^2$二次方程$(a^2)^2 - d^2 a^2 + s^2 = 0$。当时还没有代数符号系统,古巴比伦人解二次方程的过程是通过文字描述的。初中课堂上通常会在介绍求根公式前,学习用“配方法”(complete square)解二次方程。这本质上也是一种代数解法。阿拉伯数学家花拉子密在他的《代数学》中还遵循古希腊传统给出了二次方程的几何解释\footnote{即Hisab al-jabr w'al-muqabala,见第\ref{sec:Khwarizmi}节}。如果$a \ne 0$,我们可以把一般二次方程$ax^2 + bx + c = 0$的二次项系数化成1,然后把常数项移到右侧,得到$x^2 + \dfrac{b}{a}x = -\dfrac{c}{a}$。令$p = \dfrac{b}{a}, q= -\dfrac{c}{a}$,我们要通过几何方法解方程$x^2 + px = q$。如果$p > 0$,\cref{fig:quadratic1}给出了方程左侧的几何意义。$q$相当于一个正方形的面积$x^2$加上一个长$p$宽$x$的矩形面积。

\begin{figure}[htbp]
  \centering
  \includegraphics[scale=0.5]{img/quadratic1}
  \caption{$x^2 + px$的几何意义:正方形与长方形的面积和}
 \label{fig:quadratic1}
\end{figure}

配方法的几何意义是把矩形切分成两个同样的长$\dfrac{p}{2}$宽$x$的小矩形(见\cref{fig:quadratic2}左侧),然后把其中一个移动到正方形的下方,再补上一个边长为$\dfrac{p}{2}$的小正方形(见\cref{fig:quadratic2}右侧的黑色正方形)拼成一个边长为$x + \dfrac{p}{2}$的大正方形。左右两边图形中的白色面积相等,即:

\begin{align*}
 (x + \frac{p}{2})^2 &= q + (\frac{p}{2})^2 &&\text{大正方形面积等于白色部分加黑色部分} \\
 x + \frac{p}{2} &= \pm \sqrt{q + (\frac{p}{2})^2} &&\text{两边开方} \\
 x &= -\frac{p}{2} \pm \sqrt{q + \frac{p^2}{4}} \\
 x &= -\frac{b}{2a} \pm \sqrt{-\frac{c}{a} + \frac{b^2}{4a^2}} &&\text{代入}p = \frac{b}{a}, q= -\frac{c}{a} \\
 x &= \frac{-b \pm \sqrt{b^2 - 4ac}}{2a}
\end{align*}

\begin{figure}[htbp]
  \centering
  \includegraphics[scale=0.5]{img/quadratic2}
  \caption{$x^2 + px$的几何意义:正方形与长方形的面积和}
 \label{fig:quadratic2}
\end{figure}

这就是常见的一元二次方程求根公式。但在花拉子密时代负数还没有被接受,因此当一次项为负时就需要给出$x^2 - px = q$的几何解释;当常数项为负时需要给出$x^2 - q = px$的几何解释(见\cref{qn:geometric-quadratic})。卡尔达诺在《大术》中遵循了这样的传统,同时给出了三次方程的代数解法和几何解释。我们先介绍代数解法。对于一般三次方程$a_3x^3 + a_2x^2 + a_1x + a_0 = 0, a_3 \ne 0$,首先两边同时除以$a_3$把三次项系数化为1,即$x^3 + \dfrac{a_2}{a_3} x^2 + \dfrac{a_1}{a_3} x + \dfrac{a_0}{a_3}$。重新命名系数为$x^3 + ax^2 + bx + c = 0$。大约在十四世纪末,意大利佛罗伦萨的数学家就发现可以通过变量替换$x = y - \dfrac{a}{3}$把二次项消去:

\begin{align*}
(y - \frac{a}{3})^3 + a(y - \frac{a}{3})^2 + b(y - \frac{a}{3}) + c &= 0  \\
y^3 -\cancel{ay^2} + \frac{a^2}{3}y - \frac{a^3}{27} + \cancel{ay^2} - \frac{2a^2}{3}y + \frac{a^3}{9} + by - \frac{ab}{3} + c &= 0 && \text{二项式展开} \\
y^3 + (b - \frac{a^2}{3})y + \frac{2a^3}{27} - \frac{ab}{3} + c &= 0 \\
y^3 + py + q = 0
\end{align*}

其中$p = b - \dfrac{a^2}{3}, q = c + \dfrac{2a^2}{27} - \dfrac{ab}{3}$。由于当时人们不接受负数,根据$p, q$的正负,这种“缺二次项”的三次方程共有三种形式:

\begin{align}
y^3 + py &= q \label{eq:cubic-linear} \\
     y^3 &= py + q \label{eq:cubic-complex} \\
y^3 + q &= py
\end{align}

其中费罗独立发现的是\cref{eq:cubic-linear}形式的解,而\cref{eq:cubic-complex}形式的解直接导致了复数的发现。我们不妨以$y^3 = py + q$为例继续推导求根公式。令$y = u + v$,则方程左侧化为:

\begin{align*}
y^3 = (u + v)^3 &= u^3 + 3u^2v + 3uv^2 + v^3 && \text{二项式展开} \\
  &= u^3 + v^3 + 3uv(u + v) \\
  &= u^3 + v^3 + 3uvy && \text{代入} y = u + v \\
  &= (3uv)y + (u^3 + v^3) = py + q && \text{左边等于右边}
\end{align*}

由待定系数法知:

\[
\begin{cases}
  3uv = p \\
  u^3 + v^3 = q
\end{cases}
\]

这相当于关于$u, v$的二元方程组。从第一个方程中得到:$v = \dfrac{p}{3u}$,代入第二个方程得:

\[
u^3 + (\frac{p}{3u})^3 = q
\]

这实际上是一个关于$u^3$的二次方程:$(u^3)^2 - qu^3 + (\dfrac{p}{3})^3 = 0$,利用二次方程求根公式得到:

\[
u^3 = \frac{q}{2} \pm \sqrt{(\frac{q}{2})^2 - (\frac{p}{3})^3}
\]

注意到$u, v$在上面的二元方程组中是完全对称的,$v^3$必然具有同样的形式。我们不妨让它们分别取这两个根:

\[
\begin{cases}
u^3 = \dfrac{q}{2} + \sqrt{(\dfrac{q}{2})^2 - (\dfrac{p}{3})^3} \\
v^3 = \dfrac{q}{2} - \sqrt{(\dfrac{q}{2})^2 - (\dfrac{p}{3})^3}
\end{cases}
\]

这样它们满足$u^3 + v^3 = q$。最后由$y = u + v$,我们得到方程的根:

\[
y = u + v = \sqrt[3]{\frac{q}{2} + \sqrt{(\frac{q}{2})^2 - (\frac{p}{3})^3}} + \sqrt[3]{\frac{q}{2} - \sqrt{(\frac{q}{2})^2 - (\frac{p}{3})^3}}
\]

这就是高中数学课上介绍的卡尔达诺公式。进一步我们可以用$x = y - \dfrac{a}{3}$求出未知数$x$。

\section{佚名数学家}
\subsection{复数的几何意义和运算法则}
\subsection{匿名作者阿尔冈}
\subsection{代数基本定理}

\section{e的传奇}
\subsection{e的诞生}
\subsection{最美公式}

\section{新世界的大门}
\subsection{小试牛刀}
麦钦公式、正五边形作图
\subsection{费马大定理与高斯整数}
\subsection{抽象的数}
理想数
\subsection{新数的构造-代数数}

\begin{Exercise}[label={ex:complex}]
\Question{给出解二次方程$x^2 - px = q$的几何解释。\label{qn:geometric-quadratic}}
\end{Exercise}

\begin{Answer}[ref={ex:complex}]
\Question{给出解二次方程$x^2 - px = q$的几何解释。

有两种情况:$q \geq 0$和$q < 0$
}
\end{Answer}

\ifx\wholebook\relax \else
\section{参考答案}
\shipoutAnswer

\begin{thebibliography}{99}
\subimport{inc/}{bib-zh-cn}
\end{thebibliography}

\expandafter\enddocument
\fi
