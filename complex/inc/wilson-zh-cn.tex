\section{威尔逊定理}
\phantomsection
\label[appendix]{app:wilsons-theorem}

\begin{theorem}[威尔逊定理]
$p$是素数,当且仅当阶乘$(p - 1)! \equiv -1 \pmod p$。
\end{theorem}

其中阶乘$(p - 1)! = 1 \times 2 \times 3 \dotsm (p - 1)$。这个阶乘的值加1可以被$p$整除。我们首先证明一个引理。

\begin{lemma}
对小于素数$p$的正整数$a$,存在唯一小于$p$的正整数$b$,使得$ab \equiv 1 \pmod p$。我们称$b$为$a$的模$p$乘法逆元。
\end{lemma}

\begin{proof}
$a$与$p$互素,$(a, p) = 1$。根据贝祖等式,存在$x, y$使得$ax + py = 1$。所以$p$整除$ax - 1$,即$ax \equiv 1 \pmod p$。令$b = x + pk$,调整整数$k$使得$0 < b < p$即可。接下来证明唯一。假设存在正整数$b' < p$也满足$ab' \equiv 1 \pmod p$,两式相减:

\[
ab - ab' \equiv a(b - b') \equiv 1 - 1 \equiv 0 \pmod p
\]
即$p$整除$a(b - b')$。由于$p$不整除$a$,所以$p$整除$b - b'$,但正整数$b, b'$都小于$p$,所以$b = b'$。
\end{proof}

接下来我们利用这一引理证明威尔逊定理,我们的思路是把阶乘$(p - 1)!$中的因子重新排列,把模$p$的乘法逆元两两放在一起。

\begin{proof}
由模$p$逆元引理,对每个小于$p$的正整数$a$,存在唯一逆元$b$,使得$ab \equiv 1 \pmod p$。重新排列阶乘顺序:

\begin{align*}
(p - 1)! &= 1 \times 2 \times 3 \dotsm (p - 1)  \\
  &= 1 (a_1 b_1) (a_2 b_2) \dotsm (a_k b_k) (p - 1) && p - 1\text{是偶数} \\
  &\equiv 1 \cdot 1 \cdot 1 \dotsm 1 \cdot (p - 1) \pmod p && \text{每个} a_ib_i \equiv 1 \pmod p \\
  &\equiv -1 \pmod p
\end{align*}

反之,如果$p$是合数,我们分三种情况证明。情况(1) $p = 4, (p - 1)! = 3! = 6$,除以4余2。情况(2) 如果$p = q^2$是素数的平方,当$q \geq 3$时:

\begin{align*}
(q - 1)^2 &\geq (3 - 1)^2 = 2 \\
q^2 - 2q + 1 &\geq 2 \\
2q &\leq q^2 - 1 = p - 1
\end{align*}
所以$q, 2q$都是阶乘$(p - 1)!$的因子,即$2q^2 = 2p$是因子,所以$p$能整除这个阶乘而不余-1。情况(3) $p = ab$,$a, b$都大于1且小于p,因此是阶乘$(p - 1)!$的因子。所以$p = ab$整除阶乘,而不余-1。综合三种情况,$p$不可能是合数,只能是素数。
\end{proof}
