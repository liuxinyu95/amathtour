\ifx\wholebook\relax \else

\documentclass[b5paper]{ctexart}
\usepackage[nomarginpar
  %, margin=.5in
]{geometry}

\addtolength{\oddsidemargin}{-0.05in}
\addtolength{\evensidemargin}{-0.05in}
\addtolength{\textwidth}{0.1in}

\usepackage[cn]{../prelude}

\setcounter{page}{1}

\begin{document}

\title{零}

\author{刘新宇
\thanks{{\bfseries 刘新宇} \newline
  Email: liuxinyu99@hotmail.com \newline}
  }

\maketitle
\fi

\markboth{零}{数的旅程}

\ifx\wholebook\relax
\chapter{零}
\fi

\epigraph{因此“有”与“无”的真理,就是两者的统一。}{黑格尔《小逻辑》}

尽管数有超过5000年的历史,零却只有1200年的历史,是个“新生事物”。零诞生于印度,经由阿拉伯人引入欧洲。但它仅仅具有“占位符”的特殊身份,和1、2、3……其它数比起来是个“二等公民”,甚至面临被“开除数籍”的风险。

\section{质疑与否定}
零一出生就代表“无”、“没有”。印度人给它起名叫sunya,意为“空位”。所以2025中的0表示\underdot{没有}百。在西方,“无\footnote{对应的英文是void,表示虚空。}”在文化传统、宗教哲学上是负面否定的。它往往和黑暗、虚空、死亡等意象关联。而1代表\underdot{有}一个,2代表\underdot{有}两个……人们不说0,而说\underdot{没}有,表示对\underdot{有}、\underdot{生存}、\underdot{实在}的否定。莎士比亚《王子复仇记》中的名句“To be or not to be, that is the question.”中译为“生存还是死亡?”人们说3只羊跑了1只羊,还有两只羊;如果再跑了两只羊,人们说没有羊,而不说有0只羊。

印度-阿拉伯计数系统传入欧洲后,尽管计算方便,人们还是把计算结果转换成罗马数字,而避免使用零。教会宣布0是邪恶的符号,禁止在公开场合使用。僧侣学者们用算盘(Abacus不是中国的珠算,珠算是明代发明的)计算。欧洲的算盘源自古巴比伦的泥板。就是在一块板上用小石子进行演算。算好后抄下结果\footnote{古代中国在进行算筹计算时(见第\ref{counting-rods}节)用空位代表0。但计算完成后就把结果用乘法分组系统表示出来,如一百、两千一十五、一百有二。汉字“零”直到宋、元后才出现。}。



\ifx\wholebook\relax \else
\section{参考答案}
\shipoutAnswer


\begin{thebibliography}{99}

\end{thebibliography}

\expandafter\enddocument
\fi
