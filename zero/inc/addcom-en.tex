\ifx\wholebook\relax \else

\documentclass[UTF8]{article}

\usepackage[en]{../../prelude}

\setcounter{page}{1}

\begin{document}

\title{Natural numbers}

\author{LIU Xinyu
\thanks{{\bfseries LIU Xinyu} \newline
  Email: liuxinyu99@hotmail.com \newline}
  }

\maketitle
\fi

\markboth{Natural numbers}{Mathematics of Programming}

\chapter{Proof of commutativity of addition}
\phantomsection  % so hyperref creates bookmarks
\label[appendix]{app:add-commutativity}

To prove the commutativity of addition $a + b = b + a$, we prepare three things. The first states that, for any natural number $a$, we have:

\be
0 + a = a
\label{eq:left-zero}
\ee

It means the zero on the left side can be cancelled for addition. When $a = 0$, according to the first rule of addition, it holds:

\[
0 + 0 = 0
\]

As the induction, suppose $0 + a = a$ holds, we are going to show that $0 + a' = a'$.

\begin{align*}
0 + a' & = (0 + a)' && \text{second rule of addition} \\
       & = a'       && \text{induction assumption}
\end{align*}

Next we define the successor of 0 be 1, and prove the second fact:

\be
a' = a + 1
\label{eq:one-succ}
\ee

It means the successor of any natural number is this number plus 1. This is because:

\begin{align*}
a' & = (a + 0)' && \text{first rule of addition} \\
   & = a + 0'   && \text{second rule of addition} \\
   & = a + 1    && \text{1 is defined as the successor of 0}
\end{align*}

The third thing we are going to prove is the starting case:

\be
a + 1 = 1 + a
\label{eq:one-commu}
\ee

When $a = 0$, we have:

\begin{align*}
0 + 1 & = 1     && \text{We proved the left 0 can be cancelled} \\
      & = 1 + 0 && \text{first rule of addition}
\end{align*}

For induction case, suppose $a + 1 = 1 + a$ holds, we are going to show $a' + 1 = 1 + a'$.

\begin{align*}
a' + 1 & = a' + 0'        && \text{1 is the successor of 0} \\
       & = (a' + 0)'      && \text{first rule of addition} \\
       & = ((a + 1) + 0)' && \text{second result we proved. \cref{eq:one-succ}} \\
       & = (a + 1)'       && \text{second rule of addition} \\
       & = (1 + a)'       && \text{induction assumption} \\
       & = 1 + a'         && \text{second rule of addition}
\end{align*}

On top of these three results, we can prove the commutative law of addition. We first show that when $b = 0$ it holds. According to the first rule of addition, we have $a + 0 = a$; While from the first result we proved, $0 + a = a$ holds too. Hence $a + 0 = 0 + a$. Then we prove the induction case. Suppose $a + b = b + a$ holds, we are going to show $a + b' = b' + a$.

\begin{align*}
a + b' & = (a + b)' && \text{second rule of addition} \\
       & = (b + a)' && \text{induction assumption} \\
       & = b + a'   && \text{second rule of addition} \\
       & = b + a + 1 && \text{second result we proved. \cref{eq:one-succ}} \\
       & = b + 1 + a && \text{third result we proved. \cref{eq:one-commu}} \\
       & = (b + 1) + a && \text{associative law proved in chapter 1} \\
       & = b' + a  && \text{third result we proved. \cref{eq:one-commu}}
\end{align*}

Therefore we proved commutative law of addition with Peano's axioms(\cite{StepanovRose15}, p147 - 148).

\ifx\wholebook\relax \else
\begin{thebibliography}{99}

\bibitem{StepanovRose15}
Alexander A. Stepanov, Daniel E. Rose ``From Mathematics to Generic Programming''. Addison-Wesley Professional; 1 edition (November 17, 2014) ISBN-13: 978-0321942043

\end{thebibliography}

\expandafter\enddocument
%\end{document}

\fi
