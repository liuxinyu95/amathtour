\ifx\wholebook\relax \else

\documentclass[b5paper]{ctexart}
\usepackage[nomarginpar
  %, margin=.5in
]{geometry}

\addtolength{\oddsidemargin}{-0.05in}
\addtolength{\evensidemargin}{-0.05in}
\addtolength{\textwidth}{0.1in}

\usepackage[cn]{../prelude}

\setcounter{page}{1}

\begin{document}

\title{无理数}

\author{刘新宇
\thanks{{\bfseries 刘新宇} \newline
  Email: liuxinyu99@hotmail.com \newline}
  }

\maketitle
\fi

\markboth{无理数}{数的旅程}

\ifx\wholebook\relax
\chapter{无理数}
\fi

\epigraph{数学是上帝用来书写宇宙的文字}{伽利略}

几何的威力——埃拉托色特尼计算地球的直径

\section{万物皆数}
音乐与数似乎毫无关联,但毕达哥拉斯却发现了它们背后竟有奇妙的规律。他从中得到启发并大胆推测“万物皆数\footnote{英文All is number}”——世间万物都可以用数或数的比例来理解。

All is number.

\section{勾股定理}
Pythagoras theorem and varies of proofs.

\section{数论}
Euclid's proof to the Pythagoras triples.
Euclid and Elements
Perfect numbers and Even perfect numbers theorem (Euclid), Fermat numbers and Mersenne primes.

\section{无理数}
Discovery of irrational number, square root of 2 and n.
Compass and ruler construction I
Golden ration and Fibonacci series.

\section{欧几里得算法}
Euclidean algorithm,
It's extension and Bezout identity.
Geometric meaning of Euclidean algorithm.
Continue fraction and Euclidean algorithm.
Uncommensurable and irrational numbers, (Elements)

\section{更多的无理数}
Compass and ruler construction II,

polygon construction (pentagon, 17-gon)

Solving polygon construction with number theory, Fermat numbers, Perfect numbers, and Wantzel theorem.

\section{圆周率}
pi as an irrational number

\section{思想之剑}
Dedekind cut

\ifx\wholebook\relax \else
\section{参考答案}
\shipoutAnswer

\begin{thebibliography}{99}
\subimport{inc/}{bib-zh-cn}
\end{thebibliography}

\expandafter\enddocument
\fi
