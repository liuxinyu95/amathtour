\label[appendix]{app:gauss-wantzel-theorem}

%% https://math.stackexchange.com/questions/4102110/proof-of-gauss-wantzel-theorem
%% https://www.raahilmullick.com/wp-content/uploads/2024/10/SUMaC_Research_Project.pdf
%% https://mp.weixin.qq.com/s/vyDQ7wMUTOuvsomFLWcgqg

高斯——旺策尔定理的证明梗概需要复数的知识,建议读者先阅读第5章。这一定理给出了正$n$边形可用尺规作出的\underdot{充分必要}条件。即$n=2^k p_1 p_2 \dotsm p_m$其中$p_i$是彼此不同的费马素数。

\subsection{域}
我们首先引入域的概念。这是抽象代数中的一个重要概念,可以帮助我们把所有尺规可作出数的范围严格地界定出来。在历史上,第一个明确使用这个概念的人是法国数学天才伽罗瓦,但是他没有给这个概念命名。1871年,德国数学家戴德金给出了域的德语名字körper\footnote{中文译作“体”,例如范·德·瓦尔登《代数学》的中译本。},其对应的英文为field。

\begin{definition}
一个集合$\mathbf{F}$叫做域,如果它上面定义了加法和乘法两种运算,并且满足如下条件:
\begin{enumerate}[(1)]
\item 结合律。加法、乘法都满足结合律,即:
  \begin{align*}
    a + b + c &= a + (b + c) \\
    abc &= a(bc)
  \end{align*}

\item 交换律。加法、乘法都满足交换律,即:
  \begin{align*}
    a + b &= b + a \\
    ab &= ba
  \end{align*}

\item 分配律,即:$(a + b)c = ac + bc$。由于满足交换律,所以$c(a + b) = ca + cb$也成立。
\item 加法有单位元0,乘法有单位元1(见第\ref{sec:unit}节),并且$0 \ne 1$,即:
\[
a + 0 = 0 + a = a \qquad a1 = 1a = a
\]

\item 加法存在逆,即对$\mathbf{F}$中任何元素$a$,存在$b$使得:
\[
a + b = b + a = 0
\]
这个条件本质上定义了减法。

\item 对任何非零元素,乘法存在逆,即对任何$\mathbf{F}$中的元素$a \ne 0$,存在$b$使得:
\[
ab = ba = 1
\]
这个条件本质上定义了除法。
\end{enumerate}
\end{definition}

读者也许看出来了:域就是可以做加减乘除四则运算的集合。例如, 全体有理数$\mathbb{Q}$是一个域;全体实数$\mathbb{R}$是一个域;全体复数$\mathbb{C}$是一个域。我们还能找到更精细的域么?例如比有理数域大,但比实数域小?为此我们引入记号$\mathbb{Q}(\sqrt{a})$,其中$a$为不含平方因子的正整数。我们用例子$\mathbb{Q}(\sqrt{2})$来解释一下这个符号。它是所有形如

\be
x = a_0 + a_1 x + a_2 x^2 + \dotsb + a_n x^n
\ee
的多项式,当$x = \sqrt{2}$时得出的数构成的集合。其中$a_0, a_1, a_2, \dotsc, a_n$都是有理数。代入$\sqrt{2}$就是:

\be
x = a_0 + a_1 \sqrt{2} + a_2 (\sqrt{2})^2 + \dotsb + a_n (\sqrt{2})^n
\label{eq:polynomial-sqrt2}
\ee

这个数看似吓人,其实是个“纸老虎”。因为$(\sqrt{2})^2 = 2$是有理数,$(\sqrt{2})^4, (\sqrt{2})^6, \dotsc$都是有理数;$(\sqrt{2})^3 = 2\sqrt{2}$是$a\sqrt{2}$的形式,$(\sqrt{2})^5 = 4\sqrt{2}$是$a\sqrt{2}$的形式……所以\ref{eq:polynomial-sqrt2}的所有奇数项都是有理数$p_i$,所有的偶数项都是$q_i\sqrt{2}$,合并同类项后得:

\[
x = p + q\sqrt{2}
\]

其中$p, q$都是有理数,也就是说$\mathbb{Q}(\sqrt{2})$含有所有形如$p + q\sqrt{2}$的数,并且:

\begin{proposition}
所有形如$p + q\sqrt{2}$的数经过加减乘除四则运算后仍然在$\mathbb{Q}(\sqrt{2})$内,其中$p, q$是有理数。
\end{proposition}

\begin{proof}
加减法的结果为$a + b\sqrt{2} \pm (c + d\sqrt{2}) = (a \pm c) + (b \pm d)\sqrt{2}$,仍然是$p + q\sqrt{2}$的形式,在$\mathbb{Q}(\sqrt{2})$内。

乘法的结果为$(a + b\sqrt{2})(c + d\sqrt{2}) = (ac + 2bd) + (bc + ad)\sqrt{2}$,仍然是$p + q\sqrt{2}$的形式,在$\mathbb{Q}(\sqrt{2})$内。

除法的结果为

\begin{align*}
\frac{a + b\sqrt{2}}{c + d\sqrt{2}} &= \frac{(a + b\sqrt{2})(c - d\sqrt{2})}{(c + d\sqrt{2})(c - d\sqrt{2})} &&\text{分子分母同乘}c - d\sqrt{2} \\
&= \frac{(ac - 2bd) + (bc - ad)\sqrt{2}}{c^2 - 2d^2} &&\text{平方差公式} \\
&= \frac{ac - 2bd}{c^2 - 2d^2} + \frac{bc - ad}{c^2 - 2d^2}\sqrt{2}
\end{align*}
仍然是$p + q\sqrt{2}$的形式,在$\mathbb{Q}(\sqrt{2})$内。
\end{proof}

这就证明了$\mathbb{Q}(\sqrt{2})$是一个域。这个域有什么特殊之处呢?我们知道方程$x^2 - 2 = 0$在有理数域上无解,即不存在任何有理数$q$满足这个方程。但是它却在$\mathbb{Q}(\sqrt{2})$上有解$x_{1, 2} = \pm \sqrt{2}$。把无理数$\sqrt{2}$“加入”有理数$\mathbb{Q}$后,域扩大了。这说明域有大小,并且:

\[
\mathbb{Q} \subset \mathbb{Q}(\sqrt{2}) \subset \mathbb{R} \subset \mathbb{C}
\]

用尺规作出长为$\sqrt{2}$的线段(边长为1的正方形的对角线)后,再用尺规进行加减乘除运算(见第\ref{sec:geometric-arthimetic}节)能作出的所有长度就是集合$\mathbb{Q}(\sqrt{2})$。给定长度$a$,由于我们能用尺规作出$\sqrt{a}$长的线段(见命题\ref{thm:sqrt-a}),所以能获得域$\mathbb{Q}(\sqrt{a})$。

第二个重要的意义在于,把无理数$\sqrt{2}$加入有理数$\mathbb{Q}$后,“维度”变了。有理数只有一个维度:$\{q | q \in \mathbb{Q}\}$,而$\mathbb{Q}(\sqrt{2})$有两个维度:$\{p + q \sqrt{2} | p, q \in \mathbb{Q}\}$,我们需要两个变量$p, q$才能确定一个值,维度扩大了2倍。我们用符号:

\[
[\mathbb{Q}(\sqrt{2}) : \mathbb{Q}] = 2
\]

来定量表示扩大了多少。不难想象,我们用尺规作出$\sqrt{3}$,把它加入$\mathbb{Q}(\sqrt{2})$后,域$\mathbb{Q}(\sqrt{2})(\sqrt{3})$是这样的数组成的集合:

\begin{align*}
  \mathbb{Q}(\sqrt{2})(\sqrt{3}) &= \{a + b \sqrt{3} | a, b \in \mathbb{Q}(\sqrt{2}) \} \\
  &= \{p + q \sqrt{2} + (r + s\sqrt{2})\sqrt{3} \} && a, b \text{都形如} p + q\sqrt{2} \\
  &= \{p + q \sqrt{2} + r\sqrt{3} + s\sqrt{6} \}
\end{align*}

有4个变量,维度是4。我们看到维度又扩大了2倍。它相当于把两个无理数$\sqrt{2}, \sqrt{3}$加入有理数域的结果,简写成$\mathbb{Q}(\sqrt{2}, \sqrt{3})$。我们把这样的过程叫做\underdot{扩域}。注意到尺规作图的过程,就是一轮一轮扩域的过程:不断解二次方程,把$\sqrt{a}, \sqrt{b}, \dotsc$加入,构成一个扩域链:

\be
\mathbb{Q} \subset K_1 \subset K_2 \subset \dotsb \subset K_n
\label{eq:tower-of-geometric-field-ext}
\ee

但不管怎样,尺规作图每次只能把维度扩大2倍,总有$[K_{i+1} : K_i] = 2$,所以$[K_n : \mathbb{Q}] = 2^n$。读者朋友们,从维度的角度出发,$\mathbb{Q}(\sqrt[3]{2})$在这个扩域链中么?你能就此推翻倍立方问题么?

\subsection{分圆方程}
高斯将正$n$边形尺规作图的问题转化为把单位圆的圆周$n$等分问题,再利用复数进一步转化为求方程$x^n - 1 = 0$的全部复数根的问题,这种特殊的方程名叫“分圆方程”(cyclotomic equation),$n$次分圆方程有$n$个复数根:$\zeta_0 = 1, \zeta_1, \dotsc, \zeta_{n-1}$,任何$\zeta_k$满足$(\zeta_k)^n = 1$,所以又叫做$n$次单位根。根据高中学习的棣莫佛定理,可以用三角函数表示单位根:

\be
\zeta_k = e^{\frac{2k\pi}{n}} = \cos(\frac{2k\pi}{n}) + i\sin(\frac{2k\pi}{n})
\ee

从这个角度我们就能理解为何这$n$个根均匀分布在单位圆周上,并且第一根$\zeta_0 = \cos(0) + i\sin(0) = 1$。如果每个单位根都能用尺规作出(从0到$\zeta_k$代表的线段),就能作出对应的正$n$边形(例如作出$\zeta_1 - 1$,以此长度在单位圆上连续截取$n$次)。这样问题就进一步转化为判断$n$次单位根是否都在扩域链\cref{eq:tower-of-geometric-field-ext}中的问题。由于1是一个根,我们知道多项式$x^n - 1$必然有一个因子$x - 1$。下表列出了前6个多项式$x^n - 1$在有理数域上的因式分解,我们把每个首次出现在有理数域上不可进一步分解的多项式叫做$n$次分圆多项式,记作$\Phi_n(x)$。

\begin{align*}
x - 1 &= x - 1 & \Phi_1(x) &= x - 1 \\
x^2 - 1 &= (x - 1)(x + 1) & \Phi_2(x) &= x + 1 \\
x^3 - 1 &= (x - 1)(x^2 + x + 1) & \Phi_3(x) &= x^2 + x + 1 \\
x^4 - 1 &= (x^2 - 1)(x^2 + 1) = (x - 1)(x + 1)(x^2 + 1) & \Phi_4(x) &= x^2 + 1 \\
x^5 - 1 &= (x - 1)(x^4 + x^3 + x^2 + 1) & \Phi_5(x) &= x^4 + x^3 + x^2 + 1 \\
x^6 - 1 &= (x - 1)^2(x^2 + x + 1)(x^2 - x + 1) & \Phi_6(x) &= x^2 - x + 1 \\
\dotso
\end{align*}

1772年,欧拉首先发现了$x^n - 1$在有理数上的因式分解方法:

\be
x^n - 1 = \Phi_1(x)\Phi_a(x)\Phi_b(x) \dotsm \Phi_d(x)
\ee

其中$a, b, \dotsc, d$是所有$n$的真因子;$\Phi_i(x)$是第$i$个分圆多项式。例如$x^{18} - 1$,由于18的真因子有1, 2, 3, 6, 9,所以$x^{18} - 1 = \Phi_1(x)\Phi_2(x)\Phi_3(x)\Phi_6(x)\Phi_9(x)$。
