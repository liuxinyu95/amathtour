\section{Even perfect number theorem}
\phantomsection
\label[appendix]{app:even-perfect-number}

Euler proves that any even perfect number is an Euclidean perfect number. To do this, he developed a powerful tool, $\sigma$ function. For any whole number $n$, the $\sigma(n)$ function is defined as the sum of all its factors. Note the factors need be proper, hence $n$ itself is included in this sum. For example, $\sigma(1) = 1$, $\sigma(2) = 1 + 2 = 3$, $\sigma(6) = 1 + 2 + 3 + 6 = 12$, $\sigma(9) = 1 + 3 + 9 = 13$, $\sigma(28) = 1 + 2 + 4 + 7 + 14 + 28 = 56$. There are some interesting properties of the $\sigma$ function, for example, for any prime number $p$, the value $\sigma(p) = 1 + p$. By the definition of perfect number, $\sigma(n) = 2n$ if $n$ is perfect, this is because the sum of all proper factors is $n$, plus $n$ itself equals $2n$. Euler also discovered that $\sigma$ function is multiplicative in some sense:

\begin{lemma}
If $a, b$ are coprime, then $\sigma(ab) = \sigma(a)\sigma(b)$.
\end{lemma}

We need a proposition in order to proof this lemma:

\begin{proposition}
If $a, b$ are coprime, then any factor of $ab$ is a product of some factor of $a$ and some factor of $b$.
\end{proposition}

\begin{proof}
By the fundamental theorem of arithmetic, represent $a, b$ as product of prime numbers: $a = p_1^{a_1} p_2^{a_2} \dotsm p_n^{a_n}, b = q_1^{b_1} q_2^{b_2} \dotsm q_m^{b_m}$, where $p_i, q_j$ are prime numbers. Since $a, b$ are coprime, 1 is the only common divisor of them, hence, $p_i \ne q_j$. For their product $ab = p_1^{a_1} p_2^{a_2} \dotsm p_n^{a_n} q_1^{b_1} q_2^{b_2} \dotsm q_m^{b_m}$, any factor of $ab$ is,

\begin{align*}
d &= p_1^{c_1} p_2^{c_2} \dotsm p_n^{c_n} q_1^{d_1} q_2^{d_2} \dotsm q_m^{d_m}  & \text{where }0 \leq c_i \leq a_i, 0 \leq d_j \leq b_j \\
  &= (p_1^{c_1} p_2^{c_2} \dotsm p_n^{c_n}) (q_1^{d_1} q_2^{d_2} \dotsm q_m^{d_m}) \\
  &= d_a d_b & d_a\text{ is a factor of }a, d_b\text{is a factor of }b & \qedhere
\end{align*}
\end{proof}

We are going to show Euler's $\sigma$ function is multiplicative.

\begin{proof}
Denote all factors of $a$ by $1, a_1, a_2, \dotsc, a$, and their sum is $\sigma(a)$ by definition; denote all factors of $b$ by $1, b_1, b_2, \dotsc, b$, and their sum is $\sigma(b)$. Since $a, b$ are coprime, any factor of $ab$ is the product of some factor of $a$ and some factor of $b$. We list all factors of $ab$ in below order:

\begin{align*}
\sigma(ab)
  &= 1 + a_1 + a_2 + \dotsb + a + \\
  &\quad b_1 + b_1a_1 + b_1a_2 + \dotsb + b_1a + \\
  &\quad b_2 + b_2a_1 + b_2a_2 + \dotsb + b_2a + \\
  &\quad \dotso + \\
  &\quad b + ba_1 + ba_2 + \dotsb + ba \\
  &= \sigma(a) + b_1\sigma(a) + b_2\sigma(a) + \dotsb + b\sigma(a) & \text{substitute }\sigma(a) = 1 + a_1 + a_2 + \dotsb + a \\
  &= \sigma(a)(1 + b_1 + b_2 + \dotsb + b) \\
  &= \sigma(a)\sigma(b) && \qedhere
\end{align*}
\end{proof}

We shall show Euler's even perfect number theorem with this lemma. \index{Euler's even perfect number theorem}

\begin{theorem}[Euler's even perfect number]
Any even perfect number is an Euclidean perfect number, which is in the form of $2^n(2^{n+1}-1)$, where $2^{n+1}-1$ is prime.
\end{theorem}

\begin{proof}
For any even perfect number $N$, extract all factors of 2, such that $N = 2^nb$. $b$ must be odd, hence $2^n$ and $b$ are coprime. All factors of $2^n$ are: $1, 2, 4, \dotsc, 2^n$, their sum is $\sigma(2^n)$, which equals,

\begin{align*}
\sigma(2^n) &= 1 + 2 + 4 + \dotsb + 2^n = 2^{n+1} - 1 && \text{sum of geometric series}
\end{align*}

By the multiplicative property of $\sigma$ function,

\[
\sigma(N) = \sigma(2^n)\sigma(b) = (2^{n+1}-1)\sigma(b)
\]

Since $N$ is perfect,

\begin{align*}
\sigma(N) &= 2N  & \text{sum of factors of a perfect number} \\
          &= 2 \times 2^n b & \text{substitute }N = 2^n b \\
          &= 2^{n+1} b
\end{align*}

Compare the above equations:
\begin{align*}
(2^{n+1}-1)\sigma(b) &= \sigma(N) = 2^{n+1}b \\
\frac{b}{\sigma(b)} &= \frac{2^{n+1} - 1}{2^{n+1}} \\
\end{align*}
For the fraction on the right, their nominator and denominator differ by 1, hence is irreducible. Therefor, for the fraction on the left, their nominator and denominator are some multiples of the right side:

\[
b = (2^{n+1} - 1)c, \quad \sigma(b) = 2^{n+1}c
\]

where $c$ is some integer. If $c > 1$, then the factor of $b$ contains at least $1, b, c$, hence the sum of factors:

\begin{align*}
\sigma(b) &\geq 1 + b + c   & \text{contains at least }1, b, c \\
          &= 1 + (2^{n+1} -1)c + c &\text{substitute } b = (2^{n+1} - 1)c \\
          &= 1 + 2^{n+1}c - \cancel{c} + \cancel{c} > 2^{n+1}c = \sigma(b) &\text{substitute } \sigma(b) = 2^{n+1}c
\end{align*}
The result $\sigma(b) > \sigma(b)$ is obviously absurd, hence $c > 1$ can't be hold, it implies $c=1$, and $b = 2^{n+1} - 1$, it follows that the perfect number $N = 2^nb = 2^n(2^{n+1}-1)$. We shall next show $2^{n+1}-1$ is prime. Note:

\begin{align*}
\sigma(2^{n+1} - 1) &= \sigma(b) \\
          &= 2^{n+1}c = 2^{n+1} & \text{由}c = 1 \\
          &=1 + (2^{n + 1} - 1)
\end{align*}
It turns out that $2^{n+1}-1$ only has two factors: 1 and $2^{n+1} - 1$ itself, therefor, is prime. This proves any even perfect number is an Euclidean perfect number.
\end{proof}
