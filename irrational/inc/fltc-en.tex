\section{A combinatorial proof of Fermat's little theorem}
\phantomsection
\label[appendix]{app:fermat-little-theorem}

\index{Fermat's little theorem}
We now introduce a combinatorial proof of Fermat's little theorem.%% \cite{Wiki-FLT-proof}

\begin{proof}
Consider $a$ different colors of beads, and we want to string a necklace of length $p$, where $p$ is a prime number. Obviously, there are $a^p$ different strings, because each position in the string can be chosen from $a$ colors of beads, and there are $p$ positions to choose.

For example, if we have two colors of beads $A$ red and $B$ green, and we want to string a necklace of length 5, that is, $a = 2, p = 5$, there are $2^5 = 32$ different strings:

\begin{verbatim}
  AAAAA, AAAAB, AAABA, ..., BBBBA, BBBBB.
\end{verbatim}

Which corresponds to: Red Red Red Red Red, Red Red Red Red Green, Red Red Red Green Red, ..., Green Green Green Green Red, Green Green Green Green Green.

We are going to prove that among these $a^p$ strings of beads, if we remove $a$ strings that are all the same color (in the above example, the strings \texttt{AAAAA} and \texttt{BBBBB}), the remaining $a^p - a$ strings can be partitioned into several groups, each has exactly $p$ strings. In other words, $p$ divides $a^p - a$.

Next, consider the necklaces made by connecting the ends of the strings together. Some different strings may become the same necklace. If a string can be transformed into another by rotation, these two strings must make the same necklace. For example, the following 5 strings of beads can make the same necklace:

\begin{verbatim}
  AAAAB, AAABA, AABAA, ABAAA, BAAAA.
\end{verbatim}

\begin{figure}[htbp]
  \centering
  \subcaptionbox{A necklace consists of 3 colors represents 7 distinct strings: \texttt{ABCBAAC}, \texttt{BCBAACA}, \texttt{CBAACAB}, \texttt{BAACABC}, \texttt{AACABCB}, \texttt{ACABCBA}, \texttt{CABCBAA}}[0.45\linewidth]{\includegraphics[scale=0.2]{../img/bracelet-rotate}} \quad
  \subcaptionbox{A necklace in one color represents only one string: \texttt{AAAAAAA}}[0.45\linewidth]{\includegraphics[scale=0.2]{../img/bracelet-identical}}
  \caption{Grouping strings by necklaces.}
  \label{fig:bracelet}
\end{figure}

By this way, the 32 strings of beads in the above example can be grouped into 5 different necklaces with different colors and 2 necklaces with the same color:

\begin{verbatim}
  [AAABB, AABBA, ABBAA, BBAAA, BAAAB];
  [AABAB, ABABA, BABAA, ABAAB, BAABA];
  [AABBB, ABBBA, BBBAA, BBAAB, BAABB];
  [ABABB, BABBA, ABBAB, BBABA, BABAB];
  [ABBBB, BBBBA, BBBAB, BBABB, BABBB];
  [AAAAA];
  [BBBBB].
\end{verbatim}

How many different strings can a necklace represent? If a string $S$ is made by connecting several identical substrings $T$, and $T$ cannot be further split into identical substrings, then the necklace made by $S$ can represent $|T|$ different strings, where $|T|$ denotes the length of the string $T$. For example, the string $S=$\texttt{ABBABBABBABB} is made by connecting several identical substrings $T=$\texttt{ABB}, if we rotate one bead at a time, we can get 3 different strings:

\begin{verbatim}
  ABBABBABBABB,
  BBABBABBABBA,
  BABBABBABBAB.
\end{verbatim}

Except these three strings, no other string can be obtained by rotating the string $S$. the length of the substring \texttt{ABB} is 3, further rotation will just give the same three strings again. In summary, all the $a^p$ strings of beads can be divided into two categories: one contains the strings in one same color; the other contains the strings with different colors, but since the length $p$ of these strings is a prime number, they cannot be formed by connecting several identical substrings together. Therefore, any necklace made by such a string represents $p$ different strings. There are $a^p - a$ such strings, and they can be partitioned into several groups by connecting them into necklaces, each group has exactly $p$ strings, representing one different necklace. Therefore, $a^p - a = a(a^{p-1} - 1)$ must be divisible by $p$. Since $a$ and $p$ are coprime, $p$ must divide $a^{p-1} - 1$.
\end{proof}

This combinatorial proof does not require much mathematical knowledge. The core idea is that counting the same thing in two different ways must yield the same result.
