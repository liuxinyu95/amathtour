\section{Proof outline of Gauss-Wantzel theorem}
\phantomsection
\label[appendix]{app:gauss-wantzel-theorem}

%% https://math.stackexchange.com/questions/4102110/proof-of-gauss-wantzel-theorem
%% https://www.raahilmullick.com/wp-content/uploads/2024/10/SUMaC_Research_Project.pdf
%% https://mp.weixin.qq.com/s/vyDQ7wMUTOuvsomFLWcgqg

We need the basic knowledge of complex numbers to understand this proof outline, readers may read Chapter 6 first. This theorem states that a regular polygon with $n$ sides can be constructed by ruler and compass if and only if $n = 2^k p_1 p_2 \dotsm p_m$ where $p_i$ are distinct Fermat primes.

\subsection{Field} \index{field}
We need the concept of field, an important concept in abstract algebra, to rigorously define the scope of all numbers that can be constructed by ruler and compass. Galois was the first one who consciously used this concept, but he did not give it a name. In 1871, Dedekind gave it the name of körper which corresponds to the English term \enquote{field}. 

\begin{definition}
A set $\mathbf{F}$ is called a field if it has two operations, addition and multiplication, and satisfies the following axioms:
\begin{enumerate}[(1)]
\item Associativity. Both addition and multiplication are associative, i.e.:
  \begin{align*}
    a + b + c &= a + (b + c) \\
    abc &= a(bc)
  \end{align*}

\item Commutativity. Both addition and multiplication are commutative, i.e.:
  \begin{align*}
    a + b &= b + a \\
    ab &= ba
  \end{align*}

\item Distributivity. Multiplication distributes over addition, that is, $(a + b)c = ac + bc$. And by the commutativity, hence, $c(a + b) = ca + cb$ also holds.
\item 0 is the unit element for addition, 1 is the unit element for multiplication (see Section \ref{sec:unit}), and $0 \ne 1$, i.e.:
\[
a + 0 = 0 + a = a \qquad a1 = 1a = a
\]

\item The inverse of addition exists, i.e. for any element $a \in \mathbf{F}$, there exists $b \in \mathbf{F}$ such that:
\[
a + b = b + a = 0
\]
This condition essentially defines subtraction.

\item The inverse of multiplication exists for any non-zero element, i.e., for any non-zero $a \in \mathbf{F}$, there exists $b \in \mathbf{F}$ such that:
\[
ab = ba = 1
\]
This condition essentially defines division.
\end{enumerate}
\end{definition}

You may have noticed that a field is merely a set that allows arithmetic operations of $+, -, \times$, and $\div$. For example, the set $\mathbb{Q}$ of rational numbers, $\mathbb{R}$ of real numbers, and $\mathbb{C}$ of complex numbers are all fields. Are there any more refined fields? For example, larger than $\mathbb{Q}$ but smaller than $\mathbb{R}$? To this end, we introduce the notation $\mathbb{Q}(\sqrt{a})$, where $a$ is a positive and square free integer. Let us use $\mathbb{Q}(\sqrt{2})$ for example. It is the set of all numbers that can be obtained by evaluating polynomials,

\be
x = a_0 + a_1 x + a_2 x^2 + \dotsb + a_n x^n
\ee

for $x = \sqrt{2}$, where the coefficients $a_0, a_1, a_2, \dotsc, a_n$ are rationals. Substituting $\sqrt{2}$ into the above polynomial gives:

\be
x = a_0 + a_1 \sqrt{2} + a_2 (\sqrt{2})^2 + \dotsb + a_n (\sqrt{2})^n
\label{eq:polynomial-sqrt2}
\ee

Don't be scared by the above expression. Since $(\sqrt{2})^2 = 2$ is rational, $(\sqrt{2})^4, (\sqrt{2})^6, \dotsc$ are all rational; since $(\sqrt{2})^3 = 2\sqrt{2}$ is of the form $a\sqrt{2}$, $(\sqrt{2})^5 = 4\sqrt{2}$ is of the form $a\sqrt{2}$, etc., all odd terms in \ref{eq:polynomial-sqrt2} are rationals $p_i$, and all even terms are $q_i \sqrt{2}$. Collecting terms, it simplifies to:

\[
x = p + q\sqrt{2}
\]

where $p, q$ are rationals, in other words, $\mathbb{Q}(\sqrt{2})$ contains all numbers of the form $p + q\sqrt{2}$, and:

\begin{proposition}
All numbers of the form $p + q\sqrt{2}$, after $+, -, \times$, and $\div$ operations, still remain in $\mathbb{Q}(\sqrt{2})$, where $p, q$ are rationals.
\end{proposition}

\begin{proof}
The sum and difference are in form of $(a + b\sqrt{2}) \pm (c + d\sqrt{2}) = (a \pm c) + (b \pm d)\sqrt{2}$, which is still in the form of $p + q\sqrt{2}$, hence in $\mathbb{Q}(\sqrt{2})$.

The product is in form of $(a + b\sqrt{2})(c + d\sqrt{2}) = (ac + 2bd) + (bc + ad)\sqrt{2}$, which is still in the form of $p + q\sqrt{2}$, hence in $\mathbb{Q}(\sqrt{2})$.

The result of division is,

\begin{align*}
\frac{a + b\sqrt{2}}{c + d\sqrt{2}} &= \frac{(a + b\sqrt{2})(c - d\sqrt{2})}{(c + d\sqrt{2})(c - d\sqrt{2})} &&\text{multiply the nominator and denominator by }c - d\sqrt{2} \\
&= \frac{(ac - 2bd) + (bc - ad)\sqrt{2}}{c^2 - 2d^2} &&\text{difference of squares} \\
&= \frac{ac - 2bd}{c^2 - 2d^2} + \frac{bc - ad}{c^2 - 2d^2}\sqrt{2}
\end{align*}
It is still in the form of $p + q\sqrt{2}$, hence in $\mathbb{Q}(\sqrt{2})$.
\end{proof}

This proves that $\mathbb{Q}(\sqrt{2})$ is a field. What is special about this field? We know that the equation $x^2 - 2 = 0$ has no solution in the rational field, i.e., there is no rational number $q$ such that $q^2 = 2$. However, it does have solutions in $\mathbb{Q}(\sqrt{2})$, which are $x_{1, 2} = \pm \sqrt{2}$. After \emph{adjoin} the irrational number $\sqrt{2}$ to $\mathbb{Q}$, the field is extended (enlarged). This shows that a field can be contained in another field, and there can be ordering among fields,for example:

\[
\mathbb{Q} \subset \mathbb{Q}(\sqrt{2}) \subset \mathbb{R} \subset \mathbb{C}
\]

After construct the length of $\sqrt{2}$ (the diagonal of a unit square) by ruler and compass, all lengths that can be constructed by ruler and compass through arithmetic operations (see \cref{sec:geometric-arthimetic}) are exactly the set $\mathbb{Q}(\sqrt{2})$. Given a length $a$, since we can construct a line segment of length $\sqrt{a}$ (see proposition \ref{thm:sqrt-a}), we can obtain the field $\mathbb{Q}(\sqrt{a})$.

The second important thing of $\mathbb{Q}(\sqrt{2})$ is that the \enquote{dimension} changes after adjoining $\sqrt{2}$ to $\mathbb{Q}$. The rational numbers have only one dimension: $\{q | q \in \mathbb{Q}\}$, while $\mathbb{Q}(\sqrt{2})$ has two: $\{p + q \sqrt{2} | p, q \in \mathbb{Q}\}$, we need two variables $p, q$ to determine a value, the dimension is doubled. We use the notation:

\[
[\mathbb{Q}(\sqrt{2}) : \mathbb{Q}] = 2
\]

to qualify the dimension change. It is not hard to imagine that if we construct $\sqrt{3}$ by ruler and compass, and adjoin it to $\mathbb{Q}(\sqrt{2})$, the field $\mathbb{Q}(\sqrt{2})(\sqrt{3})$ is the set of all numbers of the form:

\begin{align*}
  \mathbb{Q}(\sqrt{2})(\sqrt{3}) &= \{a + b \sqrt{3} | a, b \in \mathbb{Q}(\sqrt{2}) \} \\
  &= \{p + q \sqrt{2} + (r + s\sqrt{2})\sqrt{3} \} && a, b \text{ are in the form of } p + q\sqrt{2} \\
  &= \{p + q \sqrt{2} + r\sqrt{3} + s\sqrt{6} \}
\end{align*}

\index{field extension}
There are 4 variables $p, q, r, s$ in the above expression; the dimension changes to 4, being doubled again. It is equivalent to adjoining two irrationals $\sqrt{2}, \sqrt{3}$ to the rational field, denoted by $\mathbb{Q}(\sqrt{2}, \sqrt{3})$. We call this process \emph{field extension}. It means the ruler and compass construction is a process of field extension: continuously solving quadratic equations and adjoining $\sqrt{a}, \sqrt{b}, \dotsc$ to form a chain of field extensions:

\be
\mathbb{Q} \subset K_1 \subset K_2 \subset \dotsb \subset K_m
\label{eq:tower-of-geometric-field-ext}
\ee

Whatever constructed length being adjoined, since ruler and compass can only solve quadratic equations, which means the dimension can only be doubled, i.e., $[K_{i+1} : K_i] = 2$. It follows that $[K_m : \mathbb{Q}] = 2^m$. Based on this, can you figure out whether $\mathbb{Q}(\sqrt[3]{2})$ is in this chain of field extensions? Can you use this to disprove the doubling cube problem? (see \cref{fig:double-cube})

\begin{figure}[htbp]
 \centering
 \includegraphics[scale=0.5]{../img/double-cube}
 \caption{The polynomial $a_0 + a_1\sqrt[3]{2} + a_2(\sqrt[3]{2})^2 + \dotsb + a_n(\sqrt[3]{2})^n$, by collecting terms, can be simplified to the form $p + q\sqrt[3]{2} + r(\sqrt[3]{2})^2$. Hence $[\mathbb{Q}(\sqrt[3]{2}) : \mathbb{Q}] = 3$, which is not in the chain of field extensions in ruler and compass construction.}
 \label{fig:double-cube}
\end{figure}

\subsection{Cyclotomic equations} \index{cyclotomic equations}
Gauss transformed the problem of constructing a regular polygon with $n$ sides by ruler and compass into the problem of dividing the circumference of a unit circle into $n$ equal parts, and further transformed it into the problem of finding all complex roots of the equation $x^n - 1 = 0$ by using complex numbers. This special equation is called a \emph{cyclotomic equation}, which has $n$ complex roots: $\zeta_0 = 1, \zeta_1, \dotsc, \zeta_{n-1}$, any $\zeta_k$ satisfies $(\zeta_k)^n = 1$, hence also called \emph{$n$-th roots of unity}. According to Euler's formula and De Moivre's theorem learned in high school, the roots of unity can be expressed using trigonometric functions:

\be
\zeta_k = e^{i\frac{2k\pi}{n}} = \cos(\frac{2k\pi}{n}) + i\sin(\frac{2k\pi}{n})
\ee

It explains why the roots of unity are $n$ points uniformly distributed on the unit circle, and $\zeta_0 = \cos(0) + i\sin(0) = 1$ is the first root. If every root of unity can be constructed by ruler and compass (the line segment from 0 to $\zeta_k$), then the corresponding regular $n$-gon can be constructed (for example, construct $\zeta_1 - 1$, and then use this length to continuously cut off $n$ segments on the unit circle). Thus the problem is further transformed into the problem of whether all $n$-th roots of unity are in the chain of field extensions \cref{eq:tower-of-geometric-field-ext}. Since 1 is a root, $x - 1$ is a factor of the polynomial $x^n - 1$. The following table lists the factorization of the first 6 polynomials $x^n - 1$ over $\mathbb{Q}$ (the rationals), we call each irreducible polynomial that first appears in $\mathbb{Q}$ as an $n$-th cyclotomic polynomial, denoted by $\Phi_n(x)$.

\index{cyclotomic polynomial}
\begin{align*}
x - 1 &= x - 1 & \Phi_1(x) &= x - 1 \\
x^2 - 1 &= (x - 1)(x + 1) & \Phi_2(x) &= x + 1 \\
x^3 - 1 &= (x - 1)(x^2 + x + 1) & \Phi_3(x) &= x^2 + x + 1 \\
x^4 - 1 &= (x^2 - 1)(x^2 + 1) = (x - 1)(x + 1)(x^2 + 1) & \Phi_4(x) &= x^2 + 1 \\
x^5 - 1 &= (x - 1)(x^4 + x^3 + x^2 + 1) & \Phi_5(x) &= x^4 + x^3 + x^2 + 1 \\
x^6 - 1 &= (x - 1)^2(x^2 + x + 1)(x^2 - x + 1) & \Phi_6(x) &= x^2 - x + 1 \\
\dotso
\end{align*}

In 1772, Euler first discovered the method of factorization of $x^n - 1$ over rationals\cite{LinKailiang-2025}:

\be
x^n - 1 = \Phi_1(x)\Phi_a(x)\Phi_b(x) \dotsm \Phi_d(x)
\ee

where $a, b, \dotsc, d$ run through all factors of $n$, and $\Phi_i(x)$ is the $i$-th cyclotomic polynomial. For example, the factors of 4 are 1, 2, 4, so $x^4 - 1 = \Phi_1(x)\Phi_2(x)\Phi_4(x)$; the factors of 18 are 1, 2, 3, 6, 9, 18, so $x^{18} - 1 = \Phi_1(x)\Phi_2(x)\Phi_3(x)\Phi_6(x)\Phi_9(x)\Phi_{18}(x)$. The most special case happens when $n$ is a prime $p$, in which case Gauss proved in 1801 that the $p$-th cyclotomic polynomial $\Phi_p(x) = x^{p-1} + \dotsb + x + 1$ is irreducible (cannot be factored over rationals). Since a prime $p$ has only two factors $1$ and $p$, we have $x^p - 1 = \Phi_1(x)\Phi_p(x) = (x - 1)(x^{p-1} + \dotsb + x + 1)$.

\subsection{Euler totient function}
Let us return to the original problem: to determine whether a regular $n$-gon can be constructed by ruler and compass, we only need to determine whether all $n$-th roots of unity are in the chain of field extensions \cref{eq:tower-of-geometric-field-ext}. It means that for any $n$-th root of unity $\zeta_i$, there exists some extension $K_m = \mathbb{Q}(\zeta_i)$. Then $\zeta_i$ can be expressed as a solution of a series of quadratic equations, hence can be constructed by ruler and compass. Since $[K_m : \mathbb{Q}] = 2^m$, i.e., $[\mathbb{Q}(\zeta_i) : \mathbb{Q}] = 2^m$, we only need to determine whether the dimension expands by $2^m$ for some integer $m$ after adjoining the root $\zeta_i$ to $\mathbb{Q}$. Acutally, we needn't check $\zeta_0, \zeta_1, \dotsc, \zeta_{n-1}$ one by one, for example:

\begin{enumerate}[(1)]
\item For equation $x^3 - 1 = 0$, there are three roots: $1, \dfrac{-1 \pm i\sqrt{3}}{2}$. The root $\zeta_0 = 1$ is rational. We may skip the root $\zeta_2 = \dfrac{-1 - i\sqrt{3}}{2}$ because after adjoin $\zeta_1 = \dfrac{-1 + i\sqrt{3}}{2}$ to $\mathbb{Q}$, we have $\zeta_2 \in \mathbb{Q}(\zeta_1)$ (because any $a + b \zeta_2$ can be written as $(a - b) - b \zeta_1$, and one may further verify that $\zeta_2 = \zeta_1^2$). Among these three roots, only two are independent. We add $\{1, \zeta_1\}$ to the rational field $\mathbb{Q}$ to form an extension:

\begin{align*}
[\mathbb{Q}(1, \zeta_1) : \mathbb{Q}] &= [\mathbb{Q}(\zeta_1) : \mathbb{Q}] = 2  && \text{Note that }\mathbb{Q}(1) = \mathbb{Q}
\end{align*}

Therefore, the equilateral triangle can be constructed by ruler and compass.

\item For equation $x^4 - 1 = 0$, there are four roots: $\pm 1, \pm i$. But we only need to check two of them, such as $1, i$. (alteratively, we may choose $1, -i$ or $-1, i$ or $-1, -i$, but not $\pm 1$ or $\pm i$). Essentially, there are only two independent roots out of these four. We extend the rational field $\mathbb{Q}$ by adding two independent roots:

\begin{align*}
[\mathbb{Q}(1, i) : \mathbb{Q}] &= [\mathbb{Q}(i) : \mathbb{Q}] = 2 && \text{Two variables of } \{a + bi\}
\end{align*}

Therefore, the square can be constructed by ruler and compass.

\item For equation $x^6 - 1 = 0$, there are six roots: $1, \dfrac{-1 \pm i\sqrt{3}}{2}, -1, \dfrac{1 \pm i\sqrt{3}}{2}$. We choose and adjoin two independent ones among them, namely, $\{\zeta_0 = 1, \zeta_5 = \dfrac{-1 - i\sqrt{3}}{2}\}$ to the rational field $\mathbb{Q}$ to form an extension:

\begin{align*}
[\mathbb{Q}(1, \zeta_5) : \mathbb{Q}] &= [\mathbb{Q}(\zeta_5) : \mathbb{Q}] = 2 && \text{By the result of (1)}
\end{align*}

Therefore, the regular hexagon can be constructed by ruler and compass.
\end{enumerate}

Don't you find a pattern? we needn't check all $n$ roots, but only the independent ones. A root is independent if it cannot be expressed in terms of other roots. Returning to De Moivre's theorem, these $n$ roots are:

\[
\zeta_0 = e^0 = 1, \zeta_1 = e^{1\frac{2\pi}{n}i}, \dotsb, \zeta_{n-1} = e^{(n - 1)\frac{2\pi}{n}i}
\]

If an integer $b$ divides $a$, that is $a = bc$, then,

\[
\zeta_a = e^{a\frac{2\pi}{n}i} = e^{bc\frac{2\pi}{n}i} = e^{(b\frac{2\pi}{n}i)c} = (\zeta_b)^c
\]

\index{Euler totient function} \index{Euler function}
It follows that $\zeta_a$ is not independent, but can be expressed in terms of $\zeta_b$. How many independent integers are there among $1, 2, \dotsc, n-1$? Euler defined a function $\phi(n)$, called the Euler totient function (or simply Euler function), which counts the number of integers from $1$ to $n-1$ that are coprime to $n$, and $\phi(n)$ is exactly the number of independent roots of unity. This is because if $n$ and some integer $a$ have a common factor $b$ greater than 1 (not coprime), then $a = bc$, and by the above result, $\zeta_a$ is not independent. We adjoin these $\phi(n)$ independent roots to $\mathbb{Q}$ to get the field $K_m$ where all roots of the equation $x^n - 1 = 0$ lie, any number in this field can be expressed as: $a_1 \zeta_a + a_2\zeta_b + \dotsb + a_{\phi(n)}\zeta_z$.

\be
[K_m : \mathbb{Q}] = [\mathbb{Q}(\zeta_a, \zeta_b, \dotsc, \zeta_z) : \mathbb{Q}] = \phi(n)
\ee

On the other hand, if a regular $n$-gon can be constructed by ruler and compass, by \cref{eq:tower-of-geometric-field-ext}, then $[K_m : \mathbb{Q}] = 2^m$. Thus the problem of constructing a regular $n$-gon is transformed into the problem of whether the value of Euler function $\phi(n)$ is a power of 2, that is, if

\be
\phi(n) = 2^m
\label{eq:euler-function-as-power-of-2}
\ee

where $m$ is some positive integer, then the regular $n$-gon can be constructed by ruler and compass; otherwise, it cannot. Given $n$, how to calculate $\phi(n)$? For small $n$, we can directly count the numbers for those among $1, 2, \dotsc, n - 1$ bing coprime to $n$. Particularly, if $n = p$ is prime, then all integers from 1 to $p-1$ are coprime to $p$, hence $\phi(p) = p - 1$. Merely with this, we are able to claim it's impossible to construct the regular heptagon, in which $n = 7$, is a prime number. All integers from 1 to 6 are coprime to 7, so $\phi(7) = 6$, but 6 is not a power of 2, therefore, the regular heptagon is not constructible. For regular hexagon $n = 6$, only 1 and 5 are coprime to 6; $\phi(6) = 2 = 2^1$, hence is constructible. For regular heptadecagon, $n = 17$ is prime; $\phi(17) = 16 = 2^4$, it is constructible as what Gauss proved. For regular octadecagon $n = 18$, the numbers coprime to 18 are: 1, 5, 7, 11, 13, 17, hence $\phi(18) = 6$, which is not a power of 2; it is not constructible.

But for big $n$, it's impractical to count the coprime numbers one by one. Consider $\phi(p^m)$ for a prime $p$ raised to some power of $m$. To find the numbers from 1 to $p^m - 1$ that are coprime to $p^m$, we need remove the multiples of $p$, namely, $p, 2p, 3p, \dotsc, p^m - p$; If divide each by $p$, we get the sequence of $1, 2, 3, ..., p^{m-1} - 1$, in total $p^{m-1} - 1$ numbers. Therefore, the value of Euler function for $p^m$ is:

\begin{align*}
\phi(p^m) &= (p^m - 1) - (p^{m-1} - 1) \\
            &= p^m - p^{m-1} \\
            &= p^m(1-\dfrac{1}{p})
\end{align*}

Next consider the product of two different prime powers, $n = p^u q^v$. We first remove all multiples of $p$ from 1 to $n-1$, then remove all multiples of $q$, but some integers are multiples of both $p$ and $q$, therefore, we need add back those multiples of $pq$ (the inclusion-exclusion principle). 

\begin{align*}
\phi(p^uq^v) &=  (n - 1) - (\dfrac{n}{p} - 1) - (\dfrac{n}{q} - 1) + (\dfrac{n}{pq} - 1) \\
          &=  n(1 - \dfrac{1}{p})(1 - \dfrac{1}{q}) \\[5pt]
          &=  p^u(1 - \dfrac{1}{p})q^v(1 - \dfrac{1}{q}) \\[5pt]
          &=  \phi(p^u)\phi(q^v)
\end{align*}

In particular, if both $u$ and $v$ are 1, then $\phi(pq) = \phi(p)\phi(q)$. We can further extend to multiple prime powers: $n = p_1^{e_1} p_2^{e_2} \dotsm p_k^{e_k}$, in which case, the value of Euler function is,

\begin{align*}
\phi(n) &= n(1-\frac{1}{p_1}) (1-\frac{1}{p_2}) \dotsm (1-\frac{1}{p_k}) \\
    &= \phi(p_1^{e_1}) \phi(p_2^{e_2}) \dotsm \phi(p_k^{e_k})
\end{align*}

We say Euler function $\phi(n)$ is multiplicative.

\subsection{Euler function and power of 2}
Now we have all the pieces of the puzzle, let's put them together. For any regular $n$-gon that can be constructed by ruler and compass, we can factor out all the powers of 2 and other odd prime factors from $n$ using the fundamental theorem of arithmetic:

\be
n = 2^a p_1^{e_1} p_2^{e_2} \dotsm p_k^{e_k}
\ee

where $a \ge 0$, $p_1, p_2, \dotsc, p_k$ are distinct odd primes, and each exponent $e_i \ge 1$. For the special case $n = 2^a$, which represents a point, a line segment, regular polygons with $4, 8, 16, \dotsc$ sides, all can be constructed by ruler and compass. Take the value of Euler function for $n$:

\begin{align*}
\phi(n) &= \phi(2^a p_1^{e_1} p_2^{e_2} \dotsm p_k^{e_k})  \\
        &= \phi(2^a) \phi(p_1^{e_1}) \phi(p_2^{e_2}) \dotsm \phi(p_k^{e_k}) && \text{multiplicative} \\
        &= 2^a(1 - \frac{1}{2}) p_1^{e_1}(1 - \frac{1}{p_1}) p_2^{e_2}(1 - \frac{1}{p_2})\dotsm p_k^{e_k}(1 - \frac{1}{p_k}) \\
        &= 2^{a-1} p_1^{e_1 - 1}(p_1 - 1) p_2^{e_2 - 1}(p_2 - 1)\dotsm p_k^{e_k - 1}(p_k - 1) \\
        &= 2^{a-1} p_1^{e_1 - 1} p_2^{e_2 - 1} \dotsm p_k^{e_k - 1} (p_1 - 1)(p_2 - 1) \dotsm (p_k - 1) \\
        &= 2^m &&\text{by the condition of \cref{eq:euler-function-as-power-of-2}}
\end{align*}

Since $p_i$ are all odd primes, but $\phi(n) = 2^m$ is even, it implies each $p_i^{e_i - 1}$ equals 1, that is, each $e_i = 1$. Thus the above expression can be further simplified to:

\begin{align*}
2^{a-1} (p_1 - 1)(p_2 - 1) \dotsm (p_k - 1) & = 2^m  && \text{every }p_i^{e_i - 1} = 1 \\
(p_1 - 1)(p_2 - 1) \dotsm (p_k - 1) &= 2^{m - a + 1} && \text{divide both sides by }2^{a-1}
\end{align*}

It turns out that each $p_i - 1$ is a power of 2. Denote $p_i - 1 = 2^{n_i}$ for some integer $n_i$. Then each different prime $p_i = 2^{n_i} + 1$ is a Fermat prime, because:

\begin{proposition}
If $p = 2^m + 1$ is prime, then $m$ has no odd factors, that is, $m = 2^n$, and $p = 2^{2^n} + 1$.
\end{proposition}

\begin{proof}
Assume for contradiction that $m$ has an odd factor $b$, that is, $m = ab$. Consider the equation $x^b + 1 = 0$. When $b$ is odd, $x = -1$ is a solution of this equation, because $(-1)^b + 1 = 0$. Therefore, $x + 1$ must be a factor of $x^b + 1$. Using polynomial long division, we can find the factorization of $x^b + 1$:

\[
x^b + 1 = (x + 1)(x^{b-1} - x^{b-2} + x^{b-3} - \dotsb + 1)
\]

With this result, we have:

\begin{align*}
2^m + 1 &= 2^{ab} + 1 = (2^a)^b + 1 \\
  &= (2^a + 1)[(2^a)^{b-1} - (2^a)^{b-2} + (2^a)^{b-3} - \dotsb + 1]
\end{align*}

Which conflicts with the assumption that $p = 2^m + 1$ is prime. Therefore, $m$ has no odd factors, it must be a power of 2, denoted by $m = 2^n$, hence $p = 2^{2^n} + 1$.
\end{proof}

Substituting $p_i = 2^{n_i} + 1$ into the expression of $n$, the number of sides of the regular polygon, 

\[
n = 2^a p_1 p_2 \dotsm p_k
\]

where $p_i$ are distinct Fermat primes. This proves the necessity of the Gauss-Wantzel theorem. Next we prove the sufficiency. If $n = 2^a p_1 p_2 \dotsm p_k$, where each $p_i = 2^{n_i} + 1$ is a Fermat prime, we calculate the value of Euler function for $n$:

\begin{align*}
\phi(n) &= \phi(2^a) \phi(p_1) \phi(p_2) \dotsm \phi(p_k)  \\
   &= 2^{a - 1} (p_1 - 1) (p_2 - 1) \dotsm (p_k - 1) \\
   &= 2^{a-1} 2^{n_1} 2^{n_2} \dotsm 2^{n_k} = 2^{a - 1 + n_1 + n_2 + \dotsb + n_k}
\end{align*}

Therefore, the value of Euler function $\phi(n)$ is a power of 2, and such a regular $n$-gon is constructible by ruler and compass. This proves the Gauss-Wantzel theorem.

%% TODO: Edward Landau's proof of the impossibility of doubling the cube by ruler and compass.
