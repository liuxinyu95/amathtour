\section{Proof outline of Gauss-Wantzel theorem}
\phantomsection
\label[appendix]{app:gauss-wantzel-theorem}

%% https://math.stackexchange.com/questions/4102110/proof-of-gauss-wantzel-theorem
%% https://www.raahilmullick.com/wp-content/uploads/2024/10/SUMaC_Research_Project.pdf
%% https://mp.weixin.qq.com/s/vyDQ7wMUTOuvsomFLWcgqg

We need the basic knowledge of complex numbers to understand this proof outline, readers may read Chapter 6 first. This theorem states that a regular polygon with $n$ sides can be constructed by ruler and compass if and only if $n = 2^k p_1 p_2 \dotsm p_m$ where $p_i$ are distinct Fermat primes.

\subsection{Field} \index{field}
We need the concept of field, an important concept in abstract algebra, to rigorously define the scope of all numbers that can be constructed by ruler and compass. Galois was the first one who consciously used this concept, but he did not give it a name. In 1871, Dedekind gave it the name of körper which corresponds to the English term \enquote{field}. 

\begin{definition}
A set $\mathbf{F}$ is called a field if it has two operations, addition and multiplication, and satisfies the following axioms:
\begin{enumerate}[(1)]
\item Associativity. Both addition and multiplication are associative, i.e.:
  \begin{align*}
    a + b + c &= a + (b + c) \\
    abc &= a(bc)
  \end{align*}

\item Commutativity. Both addition and multiplication are commutative, i.e.:
  \begin{align*}
    a + b &= b + a \\
    ab &= ba
  \end{align*}

\item Distributivity. Multiplication distributes over addition, that is, $(a + b)c = ac + bc$. And by the commutativity, hence, $c(a + b) = ca + cb$ also holds.
\item 0 is the unit element for addition, 1 is the unit element for multiplication (see Section \ref{sec:unit}), and $0 \ne 1$, i.e.:
\[
a + 0 = 0 + a = a \qquad a1 = 1a = a
\]

\item The inverse of addition exists, i.e. for any element $a \in \mathbf{F}$, there exists $b \in \mathbf{F}$ such that:
\[
a + b = b + a = 0
\]
This condition essentially defines subtraction.

\item The inverse of multiplication exists for any non-zero element, i.e., for any non-zero $a \in \mathbf{F}$, there exists $b \in \mathbf{F}$ such that:
\[
ab = ba = 1
\]
This condition essentially defines division.
\end{enumerate}
\end{definition}

You may have noticed that a field is merely a set that allows arithmetic operations of $+, -, \times$, and $\div$. For example, the set $\mathbb{Q}$ of rational numbers, $\mathbb{R}$ of real numbers, and $\mathbb{C}$ of complex numbers are all fields. Are there any more refined fields? For example, larger than $\mathbb{Q}$ but smaller than $\mathbb{R}$? To this end, we introduce the notation $\mathbb{Q}(\sqrt{a})$, where $a$ is a positive and square free integer. Let us use $\mathbb{Q}(\sqrt{2})$ for example. It is the set of all numbers that can be obtained by evaluating polynomials,

\be
x = a_0 + a_1 x + a_2 x^2 + \dotsb + a_n x^n
\ee

for $x = \sqrt{2}$, where the coefficients $a_0, a_1, a_2, \dotsc, a_n$ are rationals. Substituting $\sqrt{2}$ into the above polynomial gives:

\be
x = a_0 + a_1 \sqrt{2} + a_2 (\sqrt{2})^2 + \dotsb + a_n (\sqrt{2})^n
\label{eq:polynomial-sqrt2}
\ee

Don't be scared by the above expression. Since $(\sqrt{2})^2 = 2$ is rational, $(\sqrt{2})^4, (\sqrt{2})^6, \dotsc$ are all rational; since $(\sqrt{2})^3 = 2\sqrt{2}$ is of the form $a\sqrt{2}$, $(\sqrt{2})^5 = 4\sqrt{2}$ is of the form $a\sqrt{2}$, etc., all odd terms in \ref{eq:polynomial-sqrt2} are rationals $p_i$, and all even terms are $q_i \sqrt{2}$. Collecting terms, it simplifies to:

\[
x = p + q\sqrt{2}
\]

where $p, q$ are rationals, in other words, $\mathbb{Q}(\sqrt{2})$ contains all numbers of the form $p + q\sqrt{2}$, and:

\begin{proposition}
All numbers of the form $p + q\sqrt{2}$, after $+, -, \times$, and $\div$ operations, still remain in $\mathbb{Q}(\sqrt{2})$, where $p, q$ are rationals.
\end{proposition}

\begin{proof}
The sum and difference are in form of $(a + b\sqrt{2}) \pm (c + d\sqrt{2}) = (a \pm c) + (b \pm d)\sqrt{2}$, which is still in the form of $p + q\sqrt{2}$, hence in $\mathbb{Q}(\sqrt{2})$.

The product is in form of $(a + b\sqrt{2})(c + d\sqrt{2}) = (ac + 2bd) + (bc + ad)\sqrt{2}$, which is still in the form of $p + q\sqrt{2}$, hence in $\mathbb{Q}(\sqrt{2})$.

The result of division is,

\begin{align*}
\frac{a + b\sqrt{2}}{c + d\sqrt{2}} &= \frac{(a + b\sqrt{2})(c - d\sqrt{2})}{(c + d\sqrt{2})(c - d\sqrt{2})} &&\text{multiply the nominator and denominator by }c - d\sqrt{2} \\
&= \frac{(ac - 2bd) + (bc - ad)\sqrt{2}}{c^2 - 2d^2} &&\text{difference of squares} \\
&= \frac{ac - 2bd}{c^2 - 2d^2} + \frac{bc - ad}{c^2 - 2d^2}\sqrt{2}
\end{align*}
It is still in the form of $p + q\sqrt{2}$, hence in $\mathbb{Q}(\sqrt{2})$.
\end{proof}

This proves that $\mathbb{Q}(\sqrt{2})$ is a field. What is special about this field? We know that the equation $x^2 - 2 = 0$ has no solution in the rational field, i.e., there is no rational number $q$ such that $q^2 = 2$. However, it does have solutions in $\mathbb{Q}(\sqrt{2})$, which are $x_{1, 2} = \pm \sqrt{2}$. After \emph{adjoin} the irrational number $\sqrt{2}$ to the rational numbers $\mathbb{Q}$, the field is extended (enlarged). This shows that a field can be contained in another field, for example:

\[
\mathbb{Q} \subset \mathbb{Q}(\sqrt{2}) \subset \mathbb{R} \subset \mathbb{C}
\]

After construct the length of $\sqrt{2}$ (the diagonal of a unit square) by ruler and compass, all lengths that can be constructed by ruler and compass through arithmetic operations (see \cref{sec:geometric-arthimetic}) are exactly the set $\mathbb{Q}(\sqrt{2})$. Given a length $a$, since we can construct a line segment of length $\sqrt{a}$ (see proposition \ref{thm:sqrt-a}), we can obtain the field $\mathbb{Q}(\sqrt{a})$.

The second important thing of $\mathbb{Q}(\sqrt{2})$ is that the \enquote{dimension} changes after adjoining $\sqrt{2}$ to $\mathbb{Q}$. The rational numbers have only one dimension: $\{q | q \in \mathbb{Q}\}$, while $\mathbb{Q}(\sqrt{2})$ has two: $\{p + q \sqrt{2} | p, q \in \mathbb{Q}\}$, we need two variables $p, q$ to determine a value, the dimension is doubled. We use the notation:

\[
[\mathbb{Q}(\sqrt{2}) : \mathbb{Q}] = 2
\]

to qualify the dimension change. It is not hard to imagine that if we construct $\sqrt{3}$ by ruler and compass, and adjoin it to $\mathbb{Q}(\sqrt{2})$, the field $\mathbb{Q}(\sqrt{2})(\sqrt{3})$ is the set of all numbers of the form:

\begin{align*}
  \mathbb{Q}(\sqrt{2})(\sqrt{3}) &= \{a + b \sqrt{3} | a, b \in \mathbb{Q}(\sqrt{2}) \} \\
  &= \{p + q \sqrt{2} + (r + s\sqrt{2})\sqrt{3} \} && a, b \text{ are in the form of } p + q\sqrt{2} \\
  &= \{p + q \sqrt{2} + r\sqrt{3} + s\sqrt{6} \}
\end{align*}

\index{field extension}
There are 4 variables $p, q, r, s$ in the above expression; the dimension changes to 4, being doubled again. It is equivalent to adjoining two irrationals $\sqrt{2}, \sqrt{3}$ to the rational field, denoted as $\mathbb{Q}(\sqrt{2}, \sqrt{3})$. We call this process \emph{field extension}. It means the ruler and compass construction is a process of field extension: continuously solving quadratic equations and adjoining $\sqrt{a}, \sqrt{b}, \dotsc$ to form a chain of field extensions:

\be
\mathbb{Q} \subset K_1 \subset K_2 \subset \dotsb \subset K_m
\label{eq:tower-of-geometric-field-ext}
\ee

Whatever constructed length being adjoined, since ruler and compass can only solve quadratic equations, which means the dimension can only be doubled, i.e., $[K_{i+1} : K_i] = 2$. It follows that $[K_m : \mathbb{Q}] = 2^m$. Based on this, can you figure out whether $\mathbb{Q}(\sqrt[3]{2})$ is in this chain of field extensions? Can you use this to disprove the doubling cube problem? (see \cref{fig:double-cube})

\begin{figure}[htbp]
 \centering
 \includegraphics[scale=0.5]{../img/double-cube}
 \caption{The polynomial $a_0 + a_1\sqrt[3]{2} + a_2(\sqrt[3]{2})^2 + \dotsb + a_n(\sqrt[3]{2})^n$, by collecting terms, can be simplified to the form $p + q\sqrt[3]{2} + r(\sqrt[3]{2})^2$. Hence $[\mathbb{Q}(\sqrt[3]{2}) : \mathbb{Q}] = 3$, which is not in the chain of field extensions in ruler and compass construction.}
 \label{fig:double-cube}
\end{figure}

\subsection{Cyclotomic equations} \index{cyclotomic equations}
Gauss transformed the problem of constructing a regular polygon with $n$ sides by ruler and compass into the problem of dividing the circumference of a unit circle into $n$ equal parts, and further transformed it into the problem of finding all complex roots of the equation $x^n - 1 = 0$ by using complex numbers. This special equation is called a \emph{cyclotomic equation}, which has $n$ complex roots: $\zeta_0 = 1, \zeta_1, \dotsc, \zeta_{n-1}$, any $\zeta_k$ satisfies $(\zeta_k)^n = 1$, hence also called \emph{$n$th roots of unity}. According to Euler's formula and De Moivre's theorem learned in high school, the roots of unity can be expressed using trigonometric functions:

\be
\zeta_k = e^{i\frac{2k\pi}{n}} = \cos(\frac{2k\pi}{n}) + i\sin(\frac{2k\pi}{n})
\ee

从这个角度我们就能理解为何这$n$个根均匀分布在单位圆周上,并且第一根$\zeta_0 = \cos(0) + i\sin(0) = 1$。如果每个单位根都能用尺规作出(从0到$\zeta_k$代表的线段),就能作出对应的正$n$边形(例如作出$\zeta_1 - 1$,以此长度在单位圆上连续截取$n$次)。这样问题就进一步转化为判断$n$次单位根是否都在扩域链\cref{eq:tower-of-geometric-field-ext}中的问题。由于1是一个根,我们知道多项式$x^n - 1$必然有一个因子$x - 1$。下表列出了前6个多项式$x^n - 1$在有理数域上的因式分解,我们把每个首次出现在有理数域上不可进一步分解的多项式叫做$n$次分圆多项式,记作$\Phi_n(x)$。

\index{分圆多项式}
\begin{align*}
x - 1 &= x - 1 & \Phi_1(x) &= x - 1 \\
x^2 - 1 &= (x - 1)(x + 1) & \Phi_2(x) &= x + 1 \\
x^3 - 1 &= (x - 1)(x^2 + x + 1) & \Phi_3(x) &= x^2 + x + 1 \\
x^4 - 1 &= (x^2 - 1)(x^2 + 1) = (x - 1)(x + 1)(x^2 + 1) & \Phi_4(x) &= x^2 + 1 \\
x^5 - 1 &= (x - 1)(x^4 + x^3 + x^2 + 1) & \Phi_5(x) &= x^4 + x^3 + x^2 + 1 \\
x^6 - 1 &= (x - 1)^2(x^2 + x + 1)(x^2 - x + 1) & \Phi_6(x) &= x^2 - x + 1 \\
\dotso
\end{align*}

1772年,欧拉首先发现了$x^n - 1$在有理数上的因式分解方法\cite{LinKailiang-2025}:

\be
x^n - 1 = \Phi_1(x)\Phi_a(x)\Phi_b(x) \dotsm \Phi_d(x)
\ee

其中$a, b, \dotsc, d$是所有$n$的因子,$\Phi_i(x)$是第$i$个分圆多项式。例如4的因子有1, 2, 4,所以$x^4 - 1 = \Phi_1(x)\Phi_2(x)\Phi_4(x)$;18的因子有1, 2, 3, 6, 9,18,所以$x^{18} - 1 = \Phi_1(x)\Phi_2(x)\Phi_3(x)\Phi_6(x)\Phi_9(x)\Phi_{18}(x)$。最特殊的情况发生在$n$为素数$p$时,1801年高斯证明了$p$次分圆多项式$\Phi_p(x) = x^{p-1} + \dotsb + x + 1$不可约(不能在有理数域上因式分解)。而素数$p$只有两个因子$1, p$,所以$x^p - 1 = \Phi_1(x)\Phi_p(x) = (x - 1)(x^{p-1} + \dotsb + x + 1)$。

\subsection{欧拉总计函数}
我们回到要解决的问题:判断正$n$边形能否尺规作出,只需要判断$n$次单位根是否都在扩域链\cref{eq:tower-of-geometric-field-ext}中。这意味着对任何$n$次单位根$\zeta_i$,存在某个扩域$K_m = \mathbb{Q}(\zeta_i)$。这样$\zeta_i$就可以表示成一系列二次方程的解,从而可以尺规作出。由于$[K_m : \mathbb{Q}] = 2^m$,即$[\mathbb{Q}(\zeta_i) : \mathbb(Q)] = 2^m$,所以我们只要判断把根$\zeta_i$加入有理数域$\mathbb{Q}$后,维度是否扩大了2的整数次幂就可以了。好消息是,我们无需逐一检查$\zeta_0, \zeta_1, \dotsc, \zeta_{n-1}$,例如:

\begin{enumerate}[(1)]
\item 方程$x^3 - 1 = 0$有三个根$1, \dfrac{-1 \pm i\sqrt{3}}{2}$。根$\zeta_0 = 1$是有理数。我们可以跳过根$\zeta_2 = \dfrac{-1 - i\sqrt{3}}{2}$,因为一旦把$\zeta_1 = \dfrac{-1 + i\sqrt{3}}{2}$加入有理数域$\mathbb{Q}$后,就有$\zeta_2 \in \mathbb{Q}(\zeta_1)$了(因为任何$a + b \zeta_2$都可以表示成$(a - b) - b \zeta_1$,读者可以进一步验证$\zeta_2 = \zeta_1^2$)。这三个根中,只有两个是独立的,我们把$\{1, \zeta_1\}$加入有理数域进行扩域:

\begin{align*}
[\mathbb{Q}(1, \zeta_1) : \mathbb{Q}] &= [\mathbb{Q}(\zeta_1) : \mathbb{Q}] = 2  && \text{注意到}\mathbb{Q}(1) = \mathbb{Q}
\end{align*}

因此正三角形可以尺规作出。

\item 方程$x^4 - 1 = 0$有4个根:$\pm 1, \pm i$,但我们只需要检查4个根中的2个,如$1, i$就可以了(我们也可以选择$1, -i$或$-1, i$或$-1, -i$,但不能选择$\pm 1$或$\pm i$),这4个根本质上只有两个是独立的。把它们扩入有理数域:

\begin{align*}
[\mathbb{Q}(1, i) : \mathbb{Q}] &= [\mathbb{Q}(i) : \mathbb{Q}] = 2 && \{a + bi\}\text{的维度是2}
\end{align*}

所以正方形也可以尺规作出。

\item 方程$x^6 - 1 = 0$有6个根,包括$x^3 - 1 = 0$的全部3个根和这3个根的相反数。所以这6个根本质上只有2个是独立的。我们跳过4个根,选择2个根:$\{\zeta_0 = 1, \zeta_5 = \dfrac{-1 - i\sqrt{3}}{2}\}$加入有理数域。根据(1)的结果,扩域后维度为2,所以正六边形可以尺规作出。
\end{enumerate}

你发现规律了么?我们无需检查全部$n$个根,而只需检查其中独立的根。所谓独立,就是说一个根不能用其它的根表示出来。我们再次回到棣莫弗定理,这$n$个根是:

\[
\zeta_0 = e^0 = 1, \zeta_1 = e^{1\frac{2\pi}{n}i}, \dotsb, \zeta_{n-1} = e^{(n - 1)\frac{2\pi}{n}i}
\]

如果整数$a$能被$b$整除,例如$a = bc$,则:

\[
\zeta_a = e^{a\frac{2\pi}{n}i} = e^{bc\frac{2\pi}{n}i} = e^{(b\frac{2\pi}{n}i)c} = (\zeta_b)^c
\]

\index{欧拉总计函数} \index{欧拉函数}
这样$\zeta_a$就不是独立的,而可以用$\zeta_b$表示出来。那么$1, 2, \dotsc, n-1$中有多少个独立的数呢?欧拉定义了一个函数$\phi(n)$,名叫欧拉总计函数(简称欧拉函数),它的值是$1, 2, \dotsc, n-1$中和$n$彼此互素的数的个数,并且$\phi(n)$就是独立根的个数。这是因为如果$n$和某个$a$有大于1的公因子$b$(不互素),那么$a = bc$,根据上面的分析$\zeta_a$就不是独立的。我们把这独立的$\phi(n)$个根扩入有理数域,得到方程$x^n -1 = 0$的所有根所在的数域$K_m$,这个数域中的任何数都可以表示为:$a_1 \zeta_a + a_2\zeta_b + \dotsb + a_{\phi(n)}\zeta_z$。

\be
[K_m : \mathbb{Q}] = [\mathbb{Q}(\zeta_a, \zeta_b, \dotsc, \zeta_z) : \mathbb{Q}] = \phi(n)
\ee

另一方面,如果正$n$边形能够尺规作出,根据\cref{eq:tower-of-geometric-field-ext}必然有$[K_m : \mathbb{Q}] = 2^m$。这样正$n$边形作图的问题,就转换为欧拉函数$\phi(n)$是否等于2的方幂的问题,如果:

\be
\phi(n) = 2^m
\label{eq:euler-function-as-power-of-2}
\ee

其中$m$是正整数,则正$n$边形可以尺规作出,否则不可作出。给定$n$,怎样求$\phi(n)$的值呢?对于较小的$n$,我们可以逐一检查$1, 2, \dotsc, n - 1$是否和$n$互素。特别地,如果$n$是素数,那么1到$p-1$全都和$p$互素,所以$\phi(p) = p - 1$。这已经足够解释一些结论了,例如:正7边形$n = 7$,7是是素数,从1到6都和7互素,所以$\phi(7) = 6$,但6不是2的方幂,所以正7边形不可尺规作出。正六边形$n = 6$,和6互素的数只有1和5。$\phi(6) = 2 = 2^1$,所以正六边形可以用尺规作出。正17边形$n = 17$是素数,$\phi(17) = 16 = 2^4$,所以正17边形可以用尺规作出。而正18边形$n = 18$,和18互素的数有:1, 5, 7, 11, 13, 17,所以$\phi(18) = 6$,不是2的方幂,所以不可尺规作出。

但是对于更大的$n$,我们需要再深入分析一下欧拉函数$\phi(n)$的特性,从而彻底得到高斯——旺策尔定理。首先分析素数$p$的$m$次幂,$\phi(p^m)$如何计算。我们要找出从1到$p^m-1$中和$p^m$互素的数。我们需要把$p$的倍数除去,这些数是:$p, 2p, 3p, \dotsc, p^m - p$。把它们分别除以$p$,就可以得到自然数序列:$1, 2, 3, ..., p^{m-1} - 1$,显然共有$p^{m-1} - 1$个。于是$p^m$的欧拉函数值为:

\begin{align*}
\phi(p^m) &= (p^m - 1) - (p^{m-1} - 1) \\
            &= p^m - p^{m-1} \\
            &= p^m(1-\dfrac{1}{p})
\end{align*}

接下来我们考虑$n = p^uq^v$,也就是两个不同素数幂的积。我们首先从1到$n-1$中,减去$p$的所有倍数,然后再减去$q$的所有倍数,但是有些整数既是$p$的倍数,也是$q$的倍数,所以最后要把这些$pq$的倍数再加回来(组合数学中的容斥原理)。这样有:

\begin{align*}
\phi(p^uq^v) &=  (n - 1) - (\dfrac{n}{p} - 1) - (\dfrac{n}{q} - 1) + (\dfrac{n}{pq} - 1) \\
          &=  n(1 - \dfrac{1}{p})(1 - \dfrac{1}{q}) \\[5pt]
          &=  p^u(1 - \dfrac{1}{p})q^v(1 - \dfrac{1}{q}) \\[5pt]
          &=  \phi(p^u)\phi(q^v)
\end{align*}

特别地,当指数$u$, $v$都是1的时候,有$\phi(pq) = \phi(p)\phi(q)$。并且我们可以把这个结果推广到多个素数幂的情况,如果$n = p_1^{e_1} p_2^{e_2} \dotsm p_k^{e_k}$,则其欧拉函数的值为:

\begin{align*}
\phi(n) &= n(1-\frac{1}{p_1}) (1-\frac{1}{p_2}) \dotsm (1-\frac{1}{p_k}) \\
    &= \phi(p_1^{e_1}) \phi(p_2^{e_2}) \dotsm \phi(p_k^{e_k})
\end{align*}

我们把这个性质叫做欧拉函数的乘法性质。

\subsection{欧拉函数值与2次方幂}
现在我们可以拼上最后一块拼图了。对任何\underdot{尺规可作出}的正$n$边形,利用算术基本定理把$n$中所有2的因子和其它奇素因子分解出来:

\be
n = 2^a p_1^{e_1} p_2^{e_2} \dotsm p_k^{e_k}
\ee

其中$a \ge 0$,$p_1, p_2, \dotsc, p_k$是\underdot{各不相同}的奇素数,每个指数$e_i \ge 1$。特殊情况是$n = 2^a$,即点,单位线段,正$4, 8, 16, \dotsc$边形都可以尺规作出。取$n$的欧拉函数值:

\begin{align*}
\phi(n) &= \phi(2^a p_1^{e_1} p_2^{e_2} \dotsm p_k^{e_k})  \\
        &= \phi(2^a) \phi(p_1^{e_1}) \phi(p_2^{e_2}) \dotsm \phi(p_k^{e_k}) && \text{由欧拉函数的乘法性质} \\
        &= 2^a(1 - \frac{1}{2}) p_1^{e_1}(1 - \frac{1}{p_1}) p_2^{e_2}(1 - \frac{1}{p_2})\dotsm p_k^{e_k}(1 - \frac{1}{p_k}) \\
        &= 2^{a-1} p_1^{e_1 - 1}(p_1 - 1) p_2^{e_2 - 1}(p_2 - 1)\dotsm p_k^{e_k - 1}(p_k - 1) \\
        &= 2^{a-1} p_1^{e_1 - 1} p_2^{e_2 - 1} \dotsm p_k^{e_k - 1} (p_1 - 1)(p_2 - 1) \dotsm (p_k - 1) \\
        &= 2^m &&\text{尺规可作条件\cref{eq:euler-function-as-power-of-2}}
\end{align*}

由于$p_i$都是奇素数,而最终$\phi(n) = 2^m$又必须是偶数,所以每个$p_i^{e_i - 1}$必须都等于1,也就是每个$e_i = 1$。这样上式就进一步化为:

\begin{align*}
2^{a-1} (p_1 - 1)(p_2 - 1) \dotsm (p_k - 1) & = 2^m  && \text{每个}p_i^{e_i - 1} = 1 \\
(p_1 - 1)(p_2 - 1) \dotsm (p_k - 1) &= 2^{m - a + 1} && \text{左右除以}2^{a-1}
\end{align*}

因此每个$p_i - 1$都是2的方幂,不妨记$p_i - 1 = 2^{n_i}$,其中$n_i$是整数。这样每个不同的素数$p_i = 2^{n_i} + 1$,而这种形式的素数就是费马数,因为:

\begin{proposition}
若$p = 2^m + 1$是素数,则$m$不含有任何奇数因子,即$m = 2^n$,$p = 2^{2^n} + 1$
\end{proposition}

\begin{proof}
用反证法,假设$m$有奇数因子$b$,即$m = ab$。考虑方程$x^b + 1 = 0$,$b$为奇数时,$x = -1$是这个方程的一个解:$(-1)^b + 1 = 0$。因此$x + 1$必定是$x^b + 1$的一个因式。利用多项式长除求得$x^b + 1$的因式分解为:

\[
x^b + 1 = (x + 1)(x^{b-1} - x^{b-2} + x^{b-3} - \dotsb + 1)
\]

利用这个结果有:

\begin{align*}
2^m + 1 &= 2^{ab} + 1 = (2^a)^b + 1 \\
  &= (2^a + 1)[(2^a)^{b-1} - (2^a)^{b-2} + (2^a)^{b-3} - \dotsb + 1]
\end{align*}

这与$p = 2^m + 1$是素数矛盾。因此$m$不含任何奇数因子,它必定是2的方幂,可表示成$m = 2^n$,即$p = 2^{2^n} + 1$。
\end{proof}

把这个结果带回正$n$边形$n$的表达式得:

\[
n = 2^a p_1 p_2 \dotsm p_k
\]

其中$p_i$是彼此不同的费马素数。这样就证明了高斯——旺策尔定理的必要性,我们接下来证明充分性。若$n = 2^a p_1 p_2 \dotsm p_k$,其中每个$p_i = 2^{n_i} + 1$是费马素数,我们求$n$的欧拉函数值:

\begin{align*}
\phi(n) &= \phi(2^a) \phi(p_1) \phi(p_2) \dotsm \phi(p_k)  \\
   &= 2^{a - 1} (p_1 - 1) (p_2 - 1) \dotsm (p_k - 1) \\
   &= 2^{a-1} 2^{n_1} 2^{n_2} \dotsm 2^{n_k} = 2^{a - 1 + n_1 + n_2 + \dotsb + n_k}
\end{align*}

因此欧拉函数值$\phi(n)$是2的方幂,这样的正$n$边形可尺规作出。这样我们就证明了高斯——旺策尔定理。

%% TODO: 爱德华·朗道关于倍立方尺规不可作的证明。
