\section{A limit related to the sine function}
\phantomsection
\label[appendix]{app:calculus}

Observe the sine function $\sin x$ and the \underdot{radian} $x$ on the unit circle (as shown in \cref{fig:unit-circle-trigonometric}), the arc length PQ is equal to $x$, when it approaches 0, the $\sin x$ approaches the arc length, that is:

\begin{figure}[htbp]
  \centering
  \begin{tikzpicture}[>=stealth,scale=3]

    % unit circle
    \draw[gray!50, dashed] (0,0) circle (1);

    \draw[->,thick] (-1.2,0) -- (1.2,0) node[right] {}; % x-axis
    \draw[->,thick] (0,-1.2) -- (0,1.2) node[above] {}; % y-axis
    \node[below left] at (0,0) {$O$};

    \def\theta{15} % degree
    \draw[dashed] (0,0) -- (\theta:1.2); % radius at \theta

    \draw[->] (1,0) arc (0:\theta:1) node[midway,right] {$x \quad \tan x$};

    % P(x = cos x, y = sin x)
    \fill[black] (\theta:1) circle (0.02) node[above] {$P$};

    \draw[red, thick] (\theta:1 |- 0,0) -- (0,0) node[midway,below] {$\cos x$};
    \draw[blue, thick] (\theta:1) -- (\theta:1 |- 0,0) node[midway,left] {$\sin x$};

    \draw[dashed] (1, -0.5) -- (1, 1);  % tangent at (1, 0)

    \fill[black] (1,0) circle (0.02) node[below right] {$Q(1,0)$};
  \end{tikzpicture}
  \caption{Unit circle and trigonometric functions.}
  \label{fig:unit-circle-trigonometric}
\end{figure}

\be
\lim_{x \to 0} \frac{\sin x}{x} = 1
\ee

\index{squeeze theorem} \index{sandwich theorem}
\begin{proof}
We are going to use the famous \underdot{Squeeze Theorem} (also called the Sandwich Theorem) to prove this limit. Draw a tangent line at point $Q$ in \cref{fig:unit-circle-trigonometric} that intersects the extension of $OP$, thus forming a right triangle with the outer arc $PQ$. Its two legs are 1 and $\tan x$ respectively. Now consider the inscribed right triangle formed by the arc $PQ$ (its two legs are $\cos x$ and $\sin x$ respectively), the area of the sector formed by the arc $PQ$, denoted by $S$, and the area of the circumscribed right triangle formed by the arc $PQ$, denoted by $S_o$:

\begin{align*}
  S_i &< S < S_o \\
  \frac{1}{2}\cos x \sin x &< \frac{1}{2}x < \frac{1}{2}1\cdot \tan x \\
  \cos x \sin x &< x < \tan x = \frac{\sin x}{\cos x} \\
  \cos x &< \frac{x}{\sin x} < \frac{1}{\cos x} &&\text{divided both sides by }\sin x \\
  \frac{1}{\cos x} &> \frac{\sin x}{x} > \cos x && \text{reciprocal of all three sides} \\
\end{align*}
Suppose $x > 0$, now take the limit $x \to 0^+$. The limits of $\cos x$ and its reciprocal are both 1. So the limit of $\dfrac{\sin x}{x}$, which is squeezed in between, must also be 1. One may also prove the case when $x < 0$, approaching from the left side.
\end{proof}

\section{Tangent function in continued fraction}
\phantomsection
\label[appendix]{app:tan-as-cfrac}

Tangent function is the quotient of sine function and cosine function, substituting the expansions of $\sin x$ and $\cos x$ gives:

\begin{align*}
\tan x & = \frac{\sin x}{\cos x} = \frac{x - \frac{x^3}{3!} + \frac{x^5}{5!} - \frac{x^7}{7!} + \dotsb}{1 - \frac{x^2}{2!} + \frac{x^4}{4!} - \frac{x^6}{6!} + \dotsb} \\
  &= x \frac{1 - \frac{x^2}{3!} + \frac{x^4}{5!} - \frac{x^6}{7!} + \dotsb}{1 - \frac{x^2}{2!} + \frac{x^4}{4!} - \frac{x^6}{6!} + \dotsb} && \text{extract }x \text{ from the nominator} \\
  &= x \frac{B(x)}{A(x)} = \frac{x}{\frac{A(x)}{B(x)}}
\end{align*}

Where the series $A(x) = 1 - \frac{x^2}{2!} + \frac{x^4}{4!} - \frac{x^6}{6!} + \dotsb$,$B(x) = 1 - \frac{x^2}{3!} + \frac{x^4}{5!} - \frac{x^6}{7!} + \dotsb$. Next we are going to apply the division algorithm to the two series $A(x)$ and $B(x)$:

\begin{align*}
A(x) &= 1 \cdot B(x) + R_1(x) && \text{quotient is 1, remainder is }R_1(x) \\
R_1(x) &= A(x) - B(x) && \text{the quotient 1 can cancel the constant term} \\
  &= -(\frac{1}{2!} - \frac{1}{3!})x^2 + (\frac{1}{4!} - \frac{1}{5!})x^4 -(\frac{1}{6!} - \frac{1}{7!})x^6 + \dotsb \\
  &= - \frac{2}{3!}x^2 + \frac{4}{5!}x^4 - \frac{6}{7!}x^6 + \dotsb \\
  &= - \frac{1}{3}\frac{1}{1!}x^2 + \frac{1}{5}\frac{1}{3!}x^4 - \frac{1}{7}\frac{1}{5!}x^6 + \dotsb \\
  &= -x^2(\frac{1}{3} - \frac{1}{5}\frac{1}{3!}x^2 + \frac{1}{7}\frac{1}{5!}x^4 - \dotsb) = -x^2 R_1'(x) && \text{extract }-x^2
\end{align*}

Here we use the relation $\frac{2n}{(2n + 1)!} = \frac{1}{2n+1}\frac{1}{(2n-1)!}$ to simplify the coefficients. Now the tangent function can be written as a continued fraction:

\[
\tan x = \frac{x}{\frac{A(x)}{B(x)}} = \cfrac{x}{1 + \frac{R_1(x)}{B(x)}}
 = \cfrac{x}{1 + \frac{-x^2 R_1'(x)}{B(x)}} = \cfrac{x}{1 - \cfrac{x^2}{\frac{B(x)}{R_1'(x)}}}
\]

where $R_1'(x) = \frac{1}{3} - \frac{1}{5}\frac{1}{3!}x^2 + \frac{1}{7}\frac{1}{5!}x^4 - \dotsb$ is the remaining part of $R_1(x)$ after extracting $-x^2$. Next we are going to iteratively convert $\frac{B(x)}{R_1'(x)}$ into a continued fraction. Again using the division algorithm to divide $B(x)$ by $R_1'(x)$, note that the quotient 3 can cancel the constant term:

\begin{align*}
B(x) &= 3R_1'(x) + R_2(x)  \\
R_2(x) &= B - 3R_1'(x) \\
   &= - \frac{2}{5}\frac{1}{3!}x^2 + \frac{4}{7}\frac{1}{5!}x^4 - \frac{6}{9}\frac{1}{7!}x^6 + \dotsb \\
   &= -x^2(\frac{1}{3 \cdot 5} - \frac{1}{5 \cdot 7}\frac{1}{3!}x^2 + \frac{1}{7 \cdot 9}\frac{1}{5!}x^4 - \dotsb) = -x^2 R_2'(x)
\end{align*}

Here we use the relation $\frac{2n}{2n+3}\frac{1}{(2n+1)!} = \frac{1}{(2n + 1)(2n + 3)}\frac{1}{(2n - 1)!}$ to simplify the coefficients. Now the tangent function can be written as:

\[
\tan x = \cfrac{x}{1 - \cfrac{x^2}{\frac{B(x)}{R_1'(x)}}}
 = \cfrac{x}{1 - \cfrac{x^2}{3 + \frac{-x^2R_2'(x)}{R_1'(x)}}}
 =\cfrac{x}{1 - \cfrac{x^2}{3 - \cfrac{x^2}{\frac{R_1'(x)}{R_2'(x)}}}}
\]

where $R_2'(x) = \frac{1}{3 \cdot 5} - \frac{1}{5 \cdot 7}\frac{1}{3!}x^2 + \frac{1}{7 \cdot 9}\frac{1}{5!}x^4 - \dotsb$ is the remaining part of $R_2(x)$ after extracting $-x^2$. Next we are going to iteratively convert $\frac{R_1'(x)}{R_2'(x)}$ into a continued fraction. Again using the division algorithm to divide $R_1'(x)$ by $R_2'(x)$, note that the quotient 5 can cancel the constant term:

\begin{align*}
R_1'(x) &= 5R_2'(x) + R_3(x)  \\
R_3(x) &= R_1'(x) - 5R_2'(x) \\
   &= - \frac{2}{5 \cdot 7}\frac{1}{3!}x^2 + \frac{4}{7 \cdot 9}\frac{1}{5!}x^4 - \frac{6}{9 \cdot 11}\frac{1}{7!}x^6 + \dotsb \\
   &= -x^2(\frac{1}{3 \cdot 5 \cdot 7} - \frac{1}{5 \cdot 7 \cdot 9}\frac{1}{3!}x^2 + \frac{1}{7 \cdot 9 \cdot 11}\frac{1}{5!}x^4 - \dotsb) = -x^2 R_3'(x)
\end{align*}

The continued fraction of tangent function can be written as:

\[
\tan x = \cfrac{x}{1 - \cfrac{x^2}{3 - \cfrac{x^2}{\frac{R_1'(x)}{R_2'(x)}}}} =
  \cfrac{x}{1 - \cfrac{x^2}{3 - \cfrac{x^2}{5 - \cfrac{x^2}{\frac{R_2'(x)}{R_3'(x)}}}}}
\]

\index{tangent function in continued fraction}
Repeat this process indefinitely to get the continued fraction of the tangent function:

\begin{align*}
\tan x  &= \cfrac{x}{1 - \cfrac{x^2}{3 - \cfrac{x^2}{5 - \cfrac{x^2}{7 - \cfrac{x^2}{\dotso}}}}}
\end{align*}
