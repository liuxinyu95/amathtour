\label[appendix]{app:even-perfect-number}

欧拉证明了任何偶完美数必定是欧几里得完美数。这得益于欧拉引入的一个强大的工具:$\sigma$函数。对于任何整数$n$,函数$\sigma(n)$等于$n$的所有因子的和。注意,这里没有要求是真因子,所以也包含$n$本身这个因子。我们看几个例子:$\sigma(1) = 1$,$\sigma(2) = 1 + 2 = 3$, $\sigma(6) = 1 + 2 + 3 + 6 = 12$, $\sigma(9) = 1 + 3 + 9 = 13$, $\sigma(28) = 1 + 2 + 4 + 7 + 14 + 28 = 56$。
