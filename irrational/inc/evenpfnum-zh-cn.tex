\label[appendix]{app:even-perfect-number}

欧拉证明了任何偶完美数必定是欧几里得完美数。这得益于欧拉引入的一个强大的工具:$\sigma$函数。对于任何整数$n$,函数$\sigma(n)$等于$n$的所有因子的和。注意,这里没有要求必须是真因子,所以也包含$n$本身这个因子。我们看几个例子:$\sigma(1) = 1$,$\sigma(2) = 1 + 2 = 3$, $\sigma(6) = 1 + 2 + 3 + 6 = 12$, $\sigma(9) = 1 + 3 + 9 = 13$, $\sigma(28) = 1 + 2 + 4 + 7 + 14 + 28 = 56$。$\sigma$函数有一些有趣的特性,例如对于任意素数$p$,$\sigma(p) = 1 + p$。根据完美数的定义,有$\sigma(n) = 2n$。这是因为所有真因子之和等于$n$,再加上$n$本身这个因子得$2n$。欧拉还发现$\sigma$函数具有乘法性质:

\begin{lemma}
若$a, b$互素,则$\sigma(ab) = \sigma(a)\sigma(b)$。
\end{lemma}

为了证明这一结论,我们需要先证明:

\begin{proposition}
如果$a, b$互素,则任何$ab$的因子必然是某个$a$的因子与某个$b$的因子的积。
\end{proposition}

\begin{proof}
利用算术基本定理,把$a, b$各自表示为素数幂的积:$a = p_1^{a_1} p_2^{a_2} \dotsm p_n^{a_n}, b = q_1^{b_1} q_2^{b_2} \dotsm q_m^{b_m}$,其中$p_i, q_j$都是素数。由于$a, b$互素,它们的公因数只有1,故$p_i \ne q_j$。它们的积$ab = p_1^{a_1} p_2^{a_2} \dotsm p_n^{a_n} q_1^{b_1} q_2^{b_2} \dotsm q_m^{b_m}$。任何$ab$的因数可以表示为:

\begin{align*}
d &= p_1^{c_1} p_2^{c_2} \dotsm p_n^{c_n} q_1^{d_1} q_2^{d_2} \dotsm q_m^{d_m}  & \text{其中}0 \leq c_i \leq a_i, 0 \leq d_j \leq b_j \\
  &= (p_1^{c_1} p_2^{c_2} \dotsm p_n^{c_n}) (q_1^{d_1} q_2^{d_2} \dotsm q_m^{d_m}) \\
  &= d_a d_b & d_a\text{是}a\text{的因子,}d_b\text{是}b\text{的因子} & \qedhere
\end{align*}
\end{proof}

我们接下来证明欧拉$\sigma$函数具有乘法性质。

\begin{proof}
令$a$的所有因子为:$1, a_1, a_2, \dotsc, a$,根据定义它们的和等于$\sigma(a)$;$b$的所有因子为$1, b_1, b_2, \dotsc, b$,它们的和等于$\sigma(b)$。由于$a, b$互素,因此任何$ab$的因子都是$a$的某个因子乘以$b$的某个因子的积。我们按照如下顺序列出$ab$的所有因子:

\begin{align*}
\sigma(ab)
  &= 1 + a_1 + a_2 + \dotsb + a + \\
  &\quad b_1 + b_1a_1 + b_1a_2 + \dotsb + b_1a + \\
  &\quad b_2 + b_2a_1 + b_2a_2 + \dotsb + b_2a + \\
  &\quad \dotso + \\
  &\quad b + ba_1 + ba_2 + \dotsb + ba \\
  &= \sigma(a) + b_1\sigma(a) + b_2\sigma(a) + \dotsb + b\sigma(a) & \text{代入}\sigma(a) = 1 + a_1 + a_2 + \dotsb + a \\
  &= \sigma(a)(1 + b_1 + b_2 + \dotsb + b) \\
  &= \sigma(a)\sigma(b) && \qedhere
\end{align*}
\end{proof}

有了这一引理,我们就可以证明欧拉的偶完美数定理了。

\begin{theorem}[欧拉偶完美数]
任何偶完美数一定是欧几里得完美数,具有$2^n(2^{n+1}-1)$的形式,其中$2^{n+1}-1$是素数。
\end{theorem}

\begin{proof}
我们把偶完美数$N$中的所有2的因子抽出来,写成$N = 2^nb$的形式。这样$b$一定是奇数,并且$2^n$和$b$互素。$2^n$的全部因子为:$1, 2, 4, \dotsc, 2^n$。这些因子和就是$\sigma(2^n)$,它等于:

\begin{align*}
\sigma(2^n) &= 1 + 2 + 4 + \dotsb + 2^n = 2^{n+1} - 1 && \text{等比数列求和}
\end{align*}

根据$\sigma$函数的乘法性质,有:

\[
\sigma(N) = \sigma(2^n)\sigma(b) = (2^{n+1}-1)\sigma(b)
\]

由于$N$是完美数,所以:

\begin{align*}
\sigma(N) &= 2N  & \text{完美数的因子和} \\
          &= 2 \times 2^n b & \text{代入}N = 2^n b \\
          &= 2^{n+1} b
\end{align*}

比较上面两式有:
\begin{align*}
(2^{n+1}-1)\sigma(b) &= \sigma(N) = 2^{n+1}b \\
\frac{b}{\sigma(b)} &= \frac{2^{n+1} - 1}{2^{n+1}} \\
\end{align*}
右侧的分数分子分母相差1,必定是即约分数。所以左侧分数的分子分母是右侧分子分母的某个倍数:

\[
b = (2^{n+1} - 1)c, \quad \sigma(b) = 2^{n+1}c
\]

其中$c$是某个整数。如果$c > 1$,则$b$至少含有因子$1, b, c$,这样因子和:

\begin{align*}
\sigma(b) &\geq 1 + b + c   & \text{至少有因子}1, b, c \\
          &= 1 + (2^{n+1} -1)c + c &\text{代入} b = (2^{n+1} - 1)c \\
          &= 1 + 2^{n+1}c - \cancel{c} + \cancel{c} > 2^{n+1}c = \sigma(b) &\text{代入} \sigma(b) = 2^{n+1}c
\end{align*}
这个结果:$\sigma(b) > \sigma(b)$显然矛盾,所以不可能$c > 1$,只能是$c=1$。这样$b = 2^{n+1} - 1$,而完美数$N = 2^nb = 2^n(2^{n+1}-1)$。我们接下来证明$2^{n+1}-1$是素数。注意到:

\begin{align*}
\sigma(2^{n+1} - 1) &= \sigma(b) \\
          &= 2^{n+1}c = 2^{n+1} & \text{由}c = 1 \\
          &=1 + (2^{n + 1} - 1)
\end{align*}
这说明$2^{n+1}-1$只有两个因子1和$2^{n+1} - 1$本身,所以它是素数。这就证明了任何偶完美数一定是欧几里得完美数。
\end{proof}
