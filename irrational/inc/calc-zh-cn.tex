\label[appendix]{app:calculus}

通过观察单位圆上的正弦$\sin x$和\underdot{弧度}$x$(如\cref{fig:unit-circle-trigonometric}),弧长PQ等于$x$,它趋近于0时,正弦$\sin x$趋近于弧长,即:

\begin{figure}[htbp]
  \centering
  \begin{tikzpicture}[>=stealth,scale=3] % 箭头样式和缩放比例

    % 绘制单位圆
    \draw[gray!50, dashed] (0,0) circle (1);
    % 绘制坐标轴
    \draw[->,thick] (-1.2,0) -- (1.2,0) node[right] {}; % x轴
    \draw[->,thick] (0,-1.2) -- (0,1.2) node[above] {}; % y轴
    % 标记原点
    \node[below left] at (0,0) {$O$};

    % 绘制角度θ
    \def\theta{15} % 角度值(单位:度)
    \draw[dashed] (0,0) -- (\theta:1); % 半径(终边)

    % 标记角度θ
    \draw[->] (1,0) arc (0:\theta:1) node[midway,right] {$x$};

    % 单位圆上的点P(x=cosθ, y=sinθ)
    \fill[black] (\theta:1) circle (0.02) node[right] {$P(\cos x, \sin x)$};

    % 绘制余弦线(x坐标)
    \draw[red, thick] (\theta:1 |- 0,0) -- (0,0) node[midway,below] {$\cos x$};
    % 绘制正弦线(y坐标)0.5 = sin(\theta)
    \draw[blue, thick] (\theta:1) -- (\theta:1 |- 0,0) node[midway,left] {$\sin x$};

    % 标记x=1处的点
    \fill[black] (1,0) circle (0.02) node[below right] {$Q(1,0)$};
  \end{tikzpicture}
  \caption{单位圆上的三角函数}
  \label{fig:unit-circle-trigonometric}
\end{figure}

\be
\lim_{x \to 0} \frac{\sin x}{x} = 1
\ee

\begin{proof}
我们用著名的“夹逼定理”(也叫做三明治定理)来证明这一极限。过\cref{fig:unit-circle-trigonometric}中的Q点作圆的切线交OP的延长线,这样就构成了外切弧PQ的直角三角形。它的两个直角边分别为1和$\tan x$。现在考虑弧PQ内接直角三角形(它的两个直角边分别为$\cos x$和$\sin x$)的面积$S_i$,弧PQ所在的扇形面积$S$,弧PQ外切直角三角形的面积$S_o$:

\begin{align*}
  S_i &< S < S_o \\
  \frac{1}{2}\cos x \sin x &< \frac{1}{2}x < \frac{1}{2}1\cdot \tan x \\
  \cos x \sin x &< x < \tan x \\
  \cos x &< \frac{x}{\sin x} < \frac{1}{\cos x} &&\text{两边除以}\sin x \\
  \frac{1}{\cos x} &> \frac{\sin x}{x} > \cos x &&{取倒数} \\
\end{align*}
注意这里我们限制了$x > 0$,现在对不等式取$x \to 0^+$极限。$\cos x$和它的倒数的极限都是1。所以中间所夹的$\dfrac{\sin x}{x}$的极限也是1。读者可以证明$x < 0$,从左侧逼近时的情况。
\end{proof}
