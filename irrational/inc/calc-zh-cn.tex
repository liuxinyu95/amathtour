\section{常用极限和导数}
\phantomsection
\label[appendix]{app:calculus}

通过观察单位圆上的正弦$\sin x$和\underdot{弧度}$x$(如\cref{fig:unit-circle-trigonometric}),弧长PQ等于$x$,它趋近于0时,正弦$\sin x$趋近于弧长,即:

\begin{figure}[htbp]
  \centering
  \begin{tikzpicture}[>=stealth,scale=3]

    % unit circle
    \draw[gray!50, dashed] (0,0) circle (1);

    \draw[->,thick] (-1.2,0) -- (1.2,0) node[right] {}; % x-axis
    \draw[->,thick] (0,-1.2) -- (0,1.2) node[above] {}; % y-axis
    \node[below left] at (0,0) {$O$};

    \def\theta{15} % degree
    \draw[dashed] (0,0) -- (\theta:1); % radius at \theta

    \draw[->] (1,0) arc (0:\theta:1) node[midway,right] {$x$};

    % P(x = cos x, y = sin x)
    \fill[black] (\theta:1) circle (0.02) node[right] {$P(\cos x, \sin x)$};

    \draw[red, thick] (\theta:1 |- 0,0) -- (0,0) node[midway,below] {$\cos x$};
    \draw[blue, thick] (\theta:1) -- (\theta:1 |- 0,0) node[midway,left] {$\sin x$};

    \fill[black] (1,0) circle (0.02) node[below right] {$Q(1,0)$};
  \end{tikzpicture}
  \caption{单位圆上的三角函数}
  \label{fig:unit-circle-trigonometric}
\end{figure}

\be
\lim_{x \to 0} \frac{\sin x}{x} = 1
\ee

\begin{proof}
我们用著名的“夹逼定理”(也叫做三明治定理)来证明这一极限。过\cref{fig:unit-circle-trigonometric}中的Q点作圆的切线交OP的延长线,这样就构成了外切弧PQ的直角三角形。它的两个直角边分别为1和$\tan x$。现在考虑弧PQ内接直角三角形(它的两个直角边分别为$\cos x$和$\sin x$)的面积$S_i$,弧PQ所在的扇形面积$S$,弧PQ外切直角三角形的面积$S_o$:

\begin{align*}
  S_i &< S < S_o \\
  \frac{1}{2}\cos x \sin x &< \frac{1}{2}x < \frac{1}{2}1\cdot \tan x \\
  \cos x \sin x &< x < \tan x = \frac{\sin x}{\cos x} \\
  \cos x &< \frac{x}{\sin x} < \frac{1}{\cos x} &&\text{两边除以}\sin x \\
  \frac{1}{\cos x} &> \frac{\sin x}{x} > \cos x &&{取倒数} \\
\end{align*}
注意这里我们限制了$x > 0$,现在对不等式取$x \to 0^+$极限。$\cos x$和它的倒数的极限都是1。所以中间所夹的$\dfrac{\sin x}{x}$的极限也是1。读者可以证明$x < 0$,从左侧逼近时的情况。
\end{proof}

\section{正切函数的连分数表示}
\phantomsection
\label[appendix]{app:tan-as-cfrac}

正切函数是正弦函数与余弦函数的商,代入$\sin x, \cos x$的展开式得:

\begin{align*}
\tan x & = \frac{\sin x}{\cos x} = \frac{x - \frac{x^3}{3!} + \frac{x^5}{5!} - \frac{x^7}{7!} + \dotsb}{1 - \frac{x^2}{2!} + \frac{x^4}{4!} - \frac{x^6}{6!} + \dotsb} \\
  &= x \frac{1 - \frac{x^2}{3!} + \frac{x^4}{5!} - \frac{x^6}{7!} + \dotsb}{1 - \frac{x^2}{2!} + \frac{x^4}{4!} - \frac{x^6}{6!} + \dotsb} && \text{分子提出}x \\
  &= x \frac{B(x)}{A(x)} = \frac{x}{\frac{A(x)}{B(x)}}
\end{align*}

其中级数$A(x) = 1 - \frac{x^2}{2!} + \frac{x^4}{4!} - \frac{x^6}{6!} + \dotsb$,$B(x) = 1 - \frac{x^2}{3!} + \frac{x^4}{5!} - \frac{x^6}{7!} + \dotsb$。接下来用带余数的除法求$A(x)$除以$B(x)$的商和余式:

\begin{align*}
A(x) &= 1 \cdot B(x) + R_1(x) && \text{商1,余}R_1(x) \\
R_1(x) &= A(x) - B(x) && \text{商1可消去常数项} \\
  &= -(\frac{1}{2!} - \frac{1}{3!})x^2 + (\frac{1}{4!} - \frac{1}{5!})x^4 -(\frac{1}{6!} - \frac{1}{7!})x^6 + \dotsb \\
  &= - \frac{2}{3!}x^2 + \frac{4}{5!}x^4 - \frac{6}{7!}x^6 + \dotsb \\
  &= - \frac{1}{3}\frac{1}{1!}x^2 + \frac{1}{5}\frac{1}{3!}x^4 - \frac{1}{7}\frac{1}{5!}x^6 + \dotsb \\
  &= -x^2(\frac{1}{3} - \frac{1}{5}\frac{1}{3!}x^2 + \frac{1}{7}\frac{1}{5!}x^4 - \dotsb) = -x^2 R_1'(x) && \text{提取出}-x^2
\end{align*}

其中我们利用了关系$\frac{2n}{(2n + 1)!} = \frac{1}{2n+1}\frac{1}{(2n-1)!}$。此时连分数化成了:

\[
\tan x = \frac{x}{\frac{A(x)}{B(x)}} = \cfrac{x}{1 + \frac{R_1(x)}{B(x)}}
 = \cfrac{x}{1 + \frac{-x^2 R_1'(x)}{B(x)}} = \cfrac{x}{1 - \cfrac{x^2}{\frac{B(x)}{R_1'(x)}}}
\]

其中$R_1'(x) = \frac{1}{3} - \frac{1}{5}\frac{1}{3!}x^2 + \frac{1}{7}\frac{1}{5!}x^4 - \dotsb$是$R_1(x)$提取出$-x^2$后剩余的部分。接下来迭代地把$\frac{B(x)}{R_1'(x)}$化为连分数。再次利用带余数除法求$B(x)$除以$R_1'(x)$, 注意到商3可消去常数项:

\begin{align*}
B(x) &= 3R_1'(x) + R_2(x)  \\
R_2(x) &= B - 3R_1'(x) \\
   &= - \frac{2}{5}\frac{1}{3!}x^2 + \frac{4}{7}\frac{1}{5!}x^4 - \frac{6}{9}\frac{1}{7!}x^6 + \dotsb \\
   &= -x^2(\frac{1}{3 \cdot 5} - \frac{1}{5 \cdot 7}\frac{1}{3!}x^2 + \frac{1}{7 \cdot 9}\frac{1}{5!}x^4 - \dotsb) = -x^2 R_2'(x)
\end{align*}

这里我们利用了关系$\frac{2n}{2n+3}\frac{1}{(2n+1)!} = \frac{1}{(2n + 1)(2n + 3)}\frac{1}{(2n - 1)!}$。此时连分数化成了:

\[
\tan x = \cfrac{x}{1 - \cfrac{x^2}{\frac{B(x)}{R_1'(x)}}}
 = \cfrac{x}{1 - \cfrac{x^2}{3 + \frac{-x^2R_2'(x)}{R_1'(x)}}}
 =\cfrac{x}{1 - \cfrac{x^2}{3 - \cfrac{x^2}{\frac{R_1'(x)}{R_2'(x)}}}}
\]

其中$R_2'(x) = \frac{1}{3 \cdot 5} - \frac{1}{5 \cdot 7}\frac{1}{3!}x^2 + \frac{1}{7 \cdot 9}\frac{1}{5!}x^4 - \dotsb$是$R_2(x)$提取出$-x^2$后剩余的部分。接下来用同样方法求$\frac{R_1'(x)}{R_2'(x)}$的连分数表示。再次用带余数除法求$R_1'(x)$除以$R_2'(x)$,注意到商5可以消去常数项:

\begin{align*}
R_1'(x) &= 5R_2'(x) + R_3(x)  \\
R_3(x) &= R_1'(x) - 5R_2'(x) \\
   &= - \frac{2}{5 \cdot 7}\frac{1}{3!}x^2 + \frac{4}{7 \cdot 9}\frac{1}{5!}x^4 - \frac{6}{9 \cdot 11}\frac{1}{7!}x^6 + \dotsb \\
   &= -x^2(\frac{1}{3 \cdot 5 \cdot 7} - \frac{1}{5 \cdot 7 \cdot 9}\frac{1}{3!}x^2 + \frac{1}{7 \cdot 9 \cdot 11}\frac{1}{5!}x^4 - \dotsb) = -x^2 R_3'(x)
\end{align*}

此时连分数化成了:

\[
\tan x = \cfrac{x}{1 - \cfrac{x^2}{3 - \cfrac{x^2}{\frac{R_1'(x)}{R_2'(x)}}}} =
  \cfrac{x}{1 - \cfrac{x^2}{3 - \cfrac{x^2}{5 - \cfrac{x^2}{\frac{R_2'(x)}{R_3'(x)}}}}}
\]

不断重复上述步骤最终得到正切函数的连分数:

\begin{align*}
\tan x  &= \cfrac{x}{1 - \cfrac{x^2}{3 - \cfrac{x^2}{5 - \cfrac{x^2}{7 - \cfrac{x^2}{\dotso}}}}}
\end{align*}
