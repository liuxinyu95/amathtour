\ifx\wholebook\relax \else

\documentclass[b5paper]{ctexart}
\usepackage[nomarginpar
  %, margin=.5in
]{geometry}

\addtolength{\oddsidemargin}{-0.05in}
\addtolength{\evensidemargin}{-0.05in}
\addtolength{\textwidth}{0.1in}

\usepackage[cn]{../prelude}

\setcounter{page}{1}

\begin{document}

\title{附录}

\maketitle
\fi

\chapter{参考答案}
\label{ch:answers}
\phantomsection
\shipoutAnswer

\chapter{证明和算法}
\phantomsection  % so hyperref creates bookmarks

\section{加法交换律的证明}
\phantomsection  % so hyperref creates bookmarks
\subimport{../zero/inc/}{addcom-zh-cn}

\section{埃及分数分解算法}
\phantomsection
\subimport{../fraction/inc/}{decomposition-zh-cn}

\section{偶完美数定理的证明}
\phantomsection
\subimport{../irrational/inc/}{evenpfnum-zh-cn}

\section{高斯——旺策尔定理的证明梗概}
\phantomsection
\subimport{../irrational/inc/}{wantzel-zh-cn}

\section{费马小定理的组合证明}
\phantomsection
\subimport{../irrational/inc/}{fltc-zh-cn}

\chapter{希腊字母表} \label{ch:greek-letters}
\phantomsection
\subimport{../numeral/inc/}{greek-zh-cn}

%% Reference
%% =============================================
\markboth{\bibname}{}
\phantomsection  % so hyperref creates bookmarks
\addcontentsline{toc}{chapter}{\bibname}

\begin{thebibliography}{99}
  \subimport{../numeral/inc/}{bib-zh-cn}
  \subimport{../zero/inc/}{bib-zh-cn}
  \subimport{../fraction/inc/}{bib-zh-cn}
  \subimport{../irrational/inc/}{bib-zh-cn}
\end{thebibliography}

\ifx\wholebook\relax \else
\expandafter\enddocument
\fi
