\ifx\wholebook\relax \else

\documentclass[b5paper]{ctexart}
\usepackage[nomarginpar
  %, margin=.5in
]{geometry}

\addtolength{\oddsidemargin}{-0.05in}
\addtolength{\evensidemargin}{-0.05in}
\addtolength{\textwidth}{0.1in}

\usepackage[cn]{../prelude}

\setcounter{page}{1}

\begin{document}

\title{序}

\author{刘新宇
\thanks{{\bfseries 刘新宇} \newline
  Email: liuxinyu99@hotmail.com \newline}
  }

\maketitle
\fi

\markboth{序}{数的旅程}

\chapter*{序}
\phantomsection  % so hyperref creates bookmarks
\addcontentsline{toc}{chapter}{序}

2022年冬天,我一边远程工作,一边旁听孩子上网课。我由衷地感到幸运,孩子们遇上了这么好的老师,在困难的条件下带病工作,坚守课堂教学的阵地。作为家长,除了多一些理解、支持、配合,我们还能做些什么呢?随着时代的进步,知识是不断积累的。但课堂只有45分钟,并没有随着知识的增长而延长。我们不得不争分夺秒、提高效率,把浓缩的精华传授给孩子们。尽管这些知识的背后有许多人和故事,例如伽利略在比萨斜塔的“高空抛物”,例如业余数学家费马的“秘密研究”,例如意大利数学家们关于解方程的“华山论剑”,例如青年阿尔冈的“匿名投稿”……尽管前人经历了曲折,甚至走了弯路才有了今天严格、清晰、优美的结果,但我们没法占用课堂教学的主阵地告诉孩子们这些历史。这也许是我们家长可以辅助的:在课后周末、节假日,带孩子走进博物馆,接触文物,了解历史,了解人,了解背后的“为什么?”。

数学是关于真、善、美的学问。她追求真理,一是一、二是二,任何事情都要严格证明。这里没有权威,大数学家欧拉、牛顿也会犯错误。数学就是在不断改正缺陷中进步的。她在历史上不惜多次推倒基础重建宏伟的大厦。她追求善,体现着学者风范。当年轻的拉格朗日把变分法的成果告诉欧拉时,尽管正在做着同样的研究,欧拉暂缓了自己的论文,而鼓励拉格朗日先投稿。希尔伯特为遭遇性别歧视的女数学家艾米·诺特发声:“先生们,这里是大学,不是澡堂子!”她追求美,简洁的美,对称的美,清晰的美,抽象的美。克罗内克在博士答辩时坚持认为数学是艺术;埃尔德什一生云游追寻“数学天书”中最美的证明;欧拉公式被称作“最美公式”;黄金分割数指导着艺术家们对于美的创作;迷人的分形图案出现在大自然中。

在课堂教学的主阵地之外摇旗呐喊,带孩子们走进数学博物馆研学,体验数学的历史和真善美,就是我写作本书的目的。这本书不是课本,不和课堂教学冲突;它不是教辅,不拆解例题,不讲解题技巧;它不是刷题手册,尽管每章带有动手实践的习题和完整的答案;它不是通俗科普,不讲传说八卦。这是一本认真的数学科普书,以数的发展为线索,带着青少年读者沿着历史脉络“参观”数学博物馆。回答为什么负负为正?为什么零不能作除数?为什么有的分数化成小数会循环?为什么圆周率在小数点后无穷无尽?……这本书同时回答“怎样做?”的问题,古人怎样计算圆周率?怎样开方?怎样用尺规作出正五边形,数学家怎样找出三次方程的求根公式……我们刨根问底,分数线是谁发明的?小数点是谁发明的?百分号是谁引入的?央行降息的基点是啥意思?为什么自然对数的底叫做e?我们破除谣言,毕达哥拉斯不可能听到铁匠铺敲打出不同音高的和声;书本纸张并非是黄金比的长方形;鹦鹉螺壳和向日葵花并非严格按黄金比生长……

这本书的前一半只涉及义务教育阶段的数学,后一半涉及高中阶段的数学。其中圆周率在祖冲之以后的计算方法涉及了高中的微积分初步知识。此外从第4章开始,加入了初等数论的内容。毕竟这是关于数的科普书,数论是绕不过去的。

本书成书过程中得到了常成的大力帮助,仔细审阅了我的初稿,机械工业出版社的编辑给了我不断的支持和鼓励。家里的大人孩子和很多朋友也成了我的义务审稿人,在这里一并感谢。

\vspace{15mm}

刘新宇

二零二五年深秋于北京和平里

%% 本书的电子版可以在\url{https://github.com/liuxinyu95/amathtour}下载。如果希望获得纸质版,请联系作者:liuxinyu99@hotmail.com

\ifx\wholebook\relax \else

\expandafter\enddocument
%\end{document}

\fi
