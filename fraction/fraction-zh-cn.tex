\ifx\wholebook\relax \else

\documentclass[b5paper]{ctexart}
\usepackage[nomarginpar
  %, margin=.5in
]{geometry}

\addtolength{\oddsidemargin}{-0.05in}
\addtolength{\evensidemargin}{-0.05in}
\addtolength{\textwidth}{0.1in}

\usepackage[cn]{../prelude}

\setcounter{page}{1}

\begin{document}

\title{分数}

\author{刘新宇
\thanks{{\bfseries 刘新宇} \newline
  Email: liuxinyu99@hotmail.com \newline}
  }

\maketitle
\fi

\markboth{分数}{数的旅程}

\ifx\wholebook\relax
\chapter{分数}
\fi

\epigraph{此曲只应天上有,人间能得几回闻。}{杜甫《赠花卿》}

2015年9月14日,美国LIGO\footnote{激光干涉引力波}天文台探测到一个神秘信号GW150914。它来自13亿光年之外的一次惊天动地的奇观:一对双黑洞天体彼此靠近、吸引,旋转着合并到一起(如\cref{fig:gravitational-wave})。它们巨大引力激起的“涟漪”穿越13亿年时空\footnote{根据爱因斯坦的广义相对论,引力波以光速传播。}到达了地球。物理学家们花了几个月的事件进行数据分析,排除了可能的干扰因素,最终于2016年2月11日正式宣布:这是人类首次探测到了引力波,证实了爱因斯坦在百年前通过广义相对论做出的预言。LIGO的物理学家们把引力波转换成声音信号,我们得以首次听到来自宇宙深处的“天籁之音”。

\begin{figure}[htbp]
 \centering
 \includegraphics[scale=0.35]{img/gravitational-wave}
 \caption{双黑洞彼此合并过程中产生的引力波示意。}
 \label{fig:gravitational-wave}
\end{figure}

2500年前,正是追寻天籁之音的过程,使得古希腊的先贤毕达哥拉斯发现了音乐背后的数学秘密。相传有一天毕达哥拉斯经过铁匠铺,听到从里面传出了悦耳的声音。这些声音是铁匠用不同重量的铁锤一起敲打铁砧时产生的。毕达哥拉斯注意到有些声音是和谐的,而另一些不和谐。他进一步发现当铁锤的重量比恰好是12、9、8、6时,敲打的声音是和谐的。毕达哥拉斯敏锐地发现了音乐背后的数字规律:

\begin{itemize}
\item 比例$12:6$(即$2:1$)对应纯8度音;
\item 比例$9:6$(即$3:2$)对应纯5度音;
\item 比例$12:9$(即$4:3$)对应纯4度音;
\item 比例$9:8$对应纯2度音。
\end{itemize}

这个故事非常流行。如\ref{fig:pythagoras-music}所示,这幅木刻连环画展示了毕达哥拉斯先是聆听到铁匠们挥舞铁锤敲打,然后他开始进行\underdot{定量}研究,包括敲打不同重量的钟,敲打盛有不同水的杯子,弹拨坠有不同重量的琴弦,吹奏不同长度的笛子。巴洛克时期的音乐家亨德尔有一部作品叫做《快乐的铁匠》(作品编号:HWV430)。但这个故事\underdot{不是真实}的。第二幅图和第一幅图是矛盾的。我们可以自己动手做一个小实验,用几个同样的杯子盛上不同的水,如\cref{fig:fraction-cups}所示。

\begin{figure}[htbp]
 \centering
 \includegraphics[scale=0.4]{img/cups}
 \caption{盛有不同水的杯子}
 \label{fig:fraction-cups}
\end{figure}

用不同的勺子敲击就会发现音调的高低是由杯子决定的而不是勺子。同理,铁匠铺传出的音调高低是由铁砧而不是锤子决定的。但一把锤子不能同时敲击两个铁砧。其它三幅插图也有问题。钟上、杯子上、坠在琴弦的重物上的数字4、6、8、9、12、16是印度——阿拉伯数字(见第\ref{sec:hindu-arabic-numerals}节),要到十三、十四世纪才传入欧洲。古希腊的毕达哥拉斯不可能用这样的数字。

\begin{figure}[htbp]
 \centering
 \includegraphics[scale=0.1]{img/pythagoras-music}
 \caption{毕达哥拉斯与音乐,出自1492年(或1480年)弗兰奇诺・加富里奥《乐理》中的木刻插页。}
 %% Woodcut showing Pythagoras with bells, a kind of glass harmonica, a monochord and (organ?) pipes in Pythagorean tuning. From Theorica musicae by Franchino Gaffurio, 1492 (1480?)
 %% https://commons.wikimedia.org/wiki/File:Gaffurio_Pythagoras.png
 \label{fig:pythagoras-music}
\end{figure}

古希腊的乐器叫做里尔琴(Lyre),是一种七弦琴(见\cref{fig:lyre})。

\begin{figure}[htbp]
 \centering
 \includegraphics[scale=0.5]{img/apollo-with-lyre}
 \caption{绘有太阳神阿波罗手持里尔琴的盘子,约公元前480~470年,藏于雅典德尔菲考古博物馆。}
 \label{fig:lyre}
\end{figure}


\ifx\wholebook\relax \else
\section{参考答案}
\shipoutAnswer

\begin{thebibliography}{99}
%% \subimport{inc/}{bib-zh-cn}
\end{thebibliography}

\expandafter\enddocument
\fi
