\ifx\wholebook\relax \else

\documentclass[b5paper]{article}
\usepackage[nomarginpar
  %, margin=.5in
]{geometry}

\addtolength{\oddsidemargin}{-0.05in}
\addtolength{\evensidemargin}{-0.05in}
\addtolength{\textwidth}{0.1in}

\usepackage[en]{../prelude}

\setcounter{page}{1}

\begin{document}

\title{Fractions}

\author{Xinyu LIU
\thanks{{\bfseries Xinyu LIU} \newline
  Email: liuxinyu99@hotmail.com \newline}
  }

\maketitle
\fi

\markboth{Fractions}{The Journey of Numbers}

\ifx\wholebook\relax
\chapter{Fraction}
\fi

\epigraph{Learn to be silent. Let your quiet mind listen and absorb.}{Pythagoras}

\index{gravitational wave}
The Laser Interferometer Gravitational-Wave Observatory (LIGO), an astronomical observatory, in USA detected the first gravitational wave signal on Sep 14th, 2015. The signal was named after the date as GW15094. It comes from a spectacular cosmic event happened 1.3 billion light-year away: a pair of black holes spiraled and merged into each other (\cref{fig:gravitational-wave}). The black holes were 36 and 29 times the mass of the Sun and formed a new black hole 62 times the mass of the Sun. The `ripples', namely the gravitational wave, in spacetime caused by their enormous gravitational pull travelled 1.3 billion light-years across the universe to reach Earth -- gravitational waves propagate at the speed of light according to Einstein's Theory of Relativity. Physicists in LIGO spent several months to analyze the data; they verified and finally announced on Feb 11th, 2016: this was the first detection of gravitational waves, confirming the prediction Einstein made a century earlier. Through the sound waves converted from the signal by physicists, people heard the sound of deep universe for the first time.

\begin{figure}[htbp]
 \centering
 \includegraphics[scale=0.35]{img/gravitational-wave}
 \caption{Depiction of black holes merging and gravitational waves.}
 \label{fig:gravitational-wave}
\end{figure}

\index{octave}
It was the study of the sound of universe 2500 years ago, Pythagoras discovered the theory of music. A legend said when Pythagoras passed a blacksmith's shop, he noticed various sounds produced by hammering on anvil. Georg Friedrich Händel, a music in Baroque period, composed a piano suite named `\emph{Air and Five Variations on The Harmonious Blacksmith}' (HWV430) in 1720. It was said Pythagoras observed hammers of different sizes produced consonant sounds; some were harmonic, others were not. By investigation and experiment, Pythagoras found the harmonic sounds were produced by the hammers of numerical weights of 12, 9, 8, and 6. This was the beginning of music theory. Basically, Pythagoras identified below ratios:

\begin{itemize}
\item Ratio $12:6$ (reduced to $2:1$) corresponds to an octave;
\item Ratio $9:6$ (reduced to $3:2$) corresponds to the perfect fifth;
\item Ratio of $12:9$(reduced to $4:3$) corresponds to the perfect fourth;
\item Ratio of $9:8$ corresponds to the perfect second.
\end{itemize}

This legend is well known, as shown in \cref{fig:pythagoras-music}. The woodcut showed Pythagoras was listening to the blacksmiths hammering an anvil; then he went back to do \emph{quantitative} investigation, he experimented with bells of different sizes, glasses holding different volume of waters, strings hanging with different weights, and pipes of different length. However, there was something wrong with this legend; at least the second and the third pictures of the woodcut conflict. It's easily to verify the experiment in the third picture: using different spoons to tap the same kind of glasses with some water, one will realize the tune is determined by the water in the glass, but not by the size of spoon. In the same way, the tune from the blacksmith's shop is determined by the anvil, but not the hammer. However, there was only an anvil in the first picture, which could only produce the same tune. Besides, the numbers 4, 6, 8, 9, 12, and 16 annotated to the hammers, the bells, the glasses, the hanging weights of the strings, and the pipes, are Hindu-Arabic numerals (see \cref{sec:hindu-arabic-numerals}), they hadn't been introduced to the west until to the 13th to 14th century. Pythagoras, in ancient Greece, mustn't use such numerals. Nevertheless, the woodcut depicts a good story about Pythagoras, who was curious about life and nature, engaged in hands-on practice and pioneered quantitative research.

\begin{figure}[htbp]
 \centering
 \includegraphics[scale=0.1]{img/pythagoras-music}
 \caption{Pythagoras and music. Woodcut from \emph{Theory of music} by Franchino Gaffurio, 1492 (1480?).}
 %% Woodcut showing Pythagoras with bells, a kind of glass harmonica, a monochord and (organ?) pipes in Pythagorean tuning. From Theorica musicae by Franchino Gaffurio, 1492 (1480?)
 %% https://commons.wikimedia.org/wiki/File:Gaffurio_Pythagoras.png
 \label{fig:pythagoras-music}
\end{figure}

\index{lyre}

Lyre was a kind of popular instrument with seven strings in ancient Greek, as shown in \cref{fig:lyre}; it is often used as the symbol of music. Pythagoras might obtain an octave from half a string, that is of the $\dfrac{1}{2}$ length, obtain the perfect fifth from $\dfrac{2}{3}$ length of a string, and the perfect fourth from $\dfrac{3}{4}$ length of a string. \Cref{fig:octave} shows the tunes.

\begin{figure}[htbp]
 \centering
 \subcaptionbox{A kylix depicting the god Apollo holding a lyre which was made from the shell of a tortoise. 480 - 470 BCE, Delphi Archaeological Museum. \label{fig:lyre}}{\includegraphics[scale=0.4]{img/apollo-with-lyre}}
 \subcaptionbox{Icon of lyre.}{\includegraphics[scale=0.35]{img/lyre-icon}}
 %% https://www.worldhistory.org/image/986/apollo-with-lyre/
\end{figure}

Pythagoras established his music theory that greatly influenced the west based on mathematics, specifically based on \emph{fractions and ratio}. He thought the universe was a giant lyre, with the known seven celestial bodies (including Sun, Moon, the five planets of Mercury, Venus, Mars, Jupiter, and Uranus) as strings, vibrating at different tunes. In this way, Pythagoras believed the sound of universe was numbers.

\begin{figure}[htbp]
 \centering
 \includegraphics[scale=0.4]{img/octave}
 \caption{The next C spans an octave, where the ratio of the lengths of the string is $2:1$; G spans a perfect fifth from C, where the ratio of the lengths of the string is $3:2$; F spans a perfect fourth from C, where the ratio of the length of the string is $4:3$.}
 \label{fig:octave}
\end{figure}

The history of fraction is longer than the history of zero or negative numbers. In \emph{The Zuo Tradition (Zuozhuan): Commentary on the ‘Spring and Autumn Annals’}, a book about Chinese history from 770 - 476 BCE, the author documented that the capital city of a state should not be larger than the imperial city. People categorized capital cities into three categories: large, medium, and small. For the large category, the area should be less than one third of the imperial city; for the medium, the area should be less than one fifth; for the small, the area should be less than one ninth. However, these terms, such as one third, might not be really fractions, as people hadn't used them in arithmetic calculation yet. Among the \emph{Chu bamboo slips} (dating to the middle to late Waring State period, no earlier than 305 BCE) collected by Tsinghua University, there is a `calculation table', which was used by ancient Chinese to multiply numbers up to 99. In this table, there are dedicated rows and columns for $\times \dfrac{1}{2}$, which list the multiplication result\citepage[p20-22]{CaoNa-2017} of $\dfrac{1}{4}, \dfrac{1}{2}, 1\dfrac{1}{2}, 2\dfrac{1}{2}, \dotsc, 4\dfrac{1}{2}$. Chinese developed complete arithmetic rules for fractions by Han Dynasty (\cref{sec:chinese-fractions}). Fractions appeared earliest in ancient Egypt.

\section{Egyptian fractions}
\index{Papyrus}

We know about Egyptian fractions mainly from two historic documents: Rhind papyrus (\cref{fig:rhind-papyrus}) and Moscow papyrus (\cref{fig:moscow-papyrus}). Papyrus was the writing material in ancient times, which was widely used in ancient Egypt. \emph{Cyperus papyrus} (family Cyperaceae), also called paper plant, is the plant long cultivated in the Nile delta region in Egypt. People collected its stalk or stem, cut the central pith into thin strips, pressed together, and dried to form a smooth thin writing surface. The Greeks, and later Romans also adopted papyrus for writing. Papyrus had been replaced by other less-expensive materials by 3rd century CE in Europe; people stopped using it around 12th century. The two treasured documents in papyrus record the mathematical achievement in ancient Egypt. One was bought in 1858 in a Nile resort town by a Scottish antiquary, Alexander Henry Rhind, hence its name; it is collected in British Museum. It's a mathematical textbook around 1650 BCE; the author was Ahmes (hence is also called the Ahmes papyrus), including 85 math problems and their solution. The other one, Moscow papyrus, also known as the Golenischev Papyrus, was discovered and acquired by Vladimir Golenischev, a Russian Egyptologist, in 1893. It is now housed in the Pushkin State Museum of Fine Arts in Moscow, which gives the artifact its name. It was the work by some unknown author around 1890 BCE including 25 math problems.

\begin{figure}[htbp]
 \centering
 \subcaptionbox{Part of Rhind papyrus. \label{fig:rhind-papyrus}}{\includegraphics[scale=0.2]{img/rhind-papyrus-part}} \quad
 \subcaptionbox{Part of Moscow papyrus. \label{fig:moscow-papyrus}}{ \includegraphics[scale=0.4]{img/moscow-papyrus}}
 %% https://www.britishmuseum.org/collection/image/366139001
 %% https://old.maa.org/press/periodicals/convergence/mathematical-treasure-the-rhind-and-moscow-mathematical-papyri
 %% https://mathshistory.st-andrews.ac.uk/HistTopics/Egyptian_papyri/
\end{figure}

\index{Egyptian fractions}
The two artifacts evidently show that the Egyptian could solve problem with fractions. However, the nominator of all their factions were 1. This feature is so unique, that people call such fractions \emph{Egyptian fractions} or unit fractions. In Hieroglyphic (see \cref{sec:rosetta-stone}), the Egyptian draw an `eye' over number $a$ as the notation for $\dfrac{1}{a}$. \Cref{fig:egyptian-fractions} shows $\dfrac{1}{5}$ and $\dfrac{1}{15}$ respectively in Hieroglyphic. The meaning of the unit fraction is obvious, for example, $\dfrac{1}{3}$ means one out of the three equally divided parts (one third); $\dfrac{1}{5}$ means one out of the five divided parts (one fifth); in general, $\dfrac{1}{a}$ is one out of the $a$ equally divided parts.

\begin{figure}[htbp]
 \centering
 \includegraphics[scale=0.3]{img/egyptian-fractions}
 \caption{Egyptian fractions: $\dfrac{1}{5}$ and $\dfrac{1}{15}$}
 \label{fig:egyptian-fractions}
\end{figure}

The Egyptian didn't use fractions like $\dfrac{3}{4}$ or $\dfrac{2}{7}$ as their nominators were not 1. Instead, they decompose them as the sum of unit fractions, for example: $\dfrac{3}{4} = \dfrac{1}{2} + \dfrac{1}{4}$, written as the left part of \cref{fig:sum-egyptian-fractions}. The symbol in the middle likes two walking legs. If the walking direction is as same as the writing direction, it means addition; if reversed, it means subtraction. The right part of \cref{fig:sum-egyptian-fractions} is another decomposition of $\dfrac{2}{5} (= \dfrac{1}{3} + \dfrac{1}{15})$. When decompose, the Egyptian only allowed distinct unit fractions, hence $\dfrac{2}{7}$ should not be decomposed to $\dfrac{1}{7} + \dfrac{1}{7}$, but $\dfrac{1}{4} + \dfrac{1}{28}$. It impossible to decompose into only two unit fractions sometimes, for example, in Rhind papyrus, $\dfrac{2}{29}$ was decomposed to $\dfrac{2}{29} = \dfrac{1}{24} + \dfrac{1}{58} + \dfrac{1}{174} + \dfrac{1}{232}$. The decomposition may not be unique, for example, $\dfrac{2}{29} = \dfrac{1}{15} + \dfrac{1}{435} = \dfrac{1}{16} + \dfrac{1}{232} + \dfrac{1}{464}$. Basically, there is no obvious pattern or easy rules to find a composition; it's hard for people to master Egyptian fractions. In order to support calculation with fractions, the Egyptian had to make big decomposition table for reference.

\begin{figure}[htbp]
 \centering
 \includegraphics[scale=0.3]{img/sum-egyptian-fractions}
 \caption{Left: $\dfrac{1}{2} + \dfrac{1}{4}$ \qquad Right: $\dfrac{1}{3} + \dfrac{1}{15}$}
 \label{fig:sum-egyptian-fractions}
\end{figure}


We haven't found any direct documents giving clues about why the Egyptian developed their system of fractions in this way as the sums of distinct unit fractions. One legend said: a farmer, in his will, wanted to divide his 17 horses among his three sons in the following proportions, $\dfrac{1}{2}$ to the eldest, $\dfrac{1}{3}$ to the second, and $\dfrac{1}{9}$ to the youngest. When their father died, they were not able to divide the horses as it resulted in fractional horse, nobody wanted to kill any horses. They sought the counsel of a wise old man in the village. The old man lent them an additional horse for dividing; with $17 + 1 = 19$ horses, the eldest son got $18 \times \dfrac{1}{2} = 9$ horses; the second son got $18 \times \dfrac{1}{3} = 6$ horses; the youngest son get $18 \times \dfrac{1}{9} = 2$; there remained $18 - 9 - 6 - 2 = 1$ horse, and they returned it to the old man. This story can be interpreted in Egyptian fractions:

\[
\frac{17}{18} = \frac{1}{2} + \frac{1}{3} + \frac{1}{9}
\]

%% https://proofwiki.org/wiki/Henry_Ernest_Dudeney/Modern_Puzzles/89_-_The_Seventeen_Horses
Although very popular, this legend can \emph{not} be the origin of Egyptian fractions. We've found similar stories in many peoples, including the Arabic, the Indian, the Jewish, the Chinese, and so on. Henry Dudeney collected this puzzle in his popular book \emph{Modern Puzzles and how to solve them} in 1926. In different versions, the animals vary from horses, to camels, elephants; the number and dividing portions vary too, for example, the three sons need to divide 11 animals into portions of $\dfrac{1}{2}$, $\dfrac{1}{4}$, and $\dfrac{1}{6}$, which corresponds to:

\[
\frac{11}{12} = \frac{1}{2} + \frac{1}{4} + \frac{1}{6}
\]

\Cref{qn:three-sons} asks to give all possible numbers and dividing portions. The earliest such story was from an Iran philosopher, Mulla Muhammad Mahdi Naraqi's work in the 18th century. Though there were inheritance dividing problems in Rhind papyrus, they were about evenly dividing. For example, problem 63\citepage[p15]{MKlein-1972} was about to divide 700 breads to four persons in the following the proportions: $\dfrac{2}{3}$ to the first, $\dfrac{1}{2}$ to the second, $\dfrac{1}{3}$ to the third, and $\dfrac{1}{4}$ to the fourth\footnote{$\dfrac{2}{3}$ was few of the special values needn't be decomposed to unit fractions in Egypt.}. Ahmes solved it in the way that likes equation:

\[
\frac{2}{3}x + \frac{1}{2}x + \frac{1}{3}x + \frac{1}{4}x = 700
\]

He summed $\dfrac{2}{3}$, $\dfrac{1}{2}$, $\dfrac{1}{3}$, and $\dfrac{1}{4}$ to $1\dfrac{3}{4} (= 1 + \dfrac{1}{2} + \dfrac{1}{4})$, then divided 1 by $1\dfrac{3}{4}$ to obtain $\dfrac{4}{7} = (\dfrac{1}{2} + \dfrac{1}{14})$, finally, $700 \times \dfrac{4}{7} = 400$, which was the answer $x = 400$. This is exactly how students solve an equation of one unknown in primary school. Note that the third and fourth person get breads in fraction.

In some opinion, Egyptian fractions divide things more economically. Considering to divide 5 breads to 8 persons equally, the regular way is to divide each bread into 8 pieces and everyone takes 5 pieces as shown in \cref{fig:evenly-devide}:

\begin{figure}[htbp]
 \centering
 \includegraphics[scale=0.3]{img/evenly-divide}
 \caption{Cut equally into 40 pieces, each one get 5.}
 \label{fig:evenly-devide}
\end{figure}

The Egyptian might find a better way than this as shown in \cref{fig:egyptian-devide}, where everyone get a half and $\dfrac{1}{8}$ bread.

\begin{figure}[htbp]
 \centering
 \includegraphics[scale=0.3]{img/egyptian-divide}
 \caption{Cut into 16 pieces, everyone gets two (a big one of $\dfrac{1}{2}$ and a small one of $\dfrac{1}{8}$).}
 \label{fig:egyptian-devide}
\end{figure}

Do you have any idea why the Egyptian developed such special fractions? On one hand, it's so complex that finally disappeared in history, on the other hand, it brings interesting problems with hindsight: mathematicians thought about: (1) Can every fraction be decomposed into Egyptian fractions?  (2) if can, what is the best decomposition? The first one was solved by Fibonacci in 1202 in \emph{Liber Abaci}. The second one led to a conjecture hasn't been solved till now. Let us start with two easier problems:

\index{Egyptian fraction!decomposition}
\begin{proposition} Any Egyptian fraction can be decomposed into two distinct Egyptian fractions, i.e., $\dfrac{1}{n} = \dfrac{1}{a} + \dfrac{1}{b}$. \label{th:egyptian-fraction-split}
\end{proposition}

\begin{proof}
  \begin{align*}
    \frac{1}{n} &= \frac{n+1}{n(n+1)} & \text{nominator and denominator } \times (n + 1) & \\
                &= \frac{\cancel{n}}{\cancel{n}(n+1)} + \frac{1}{n(n+1)} && \\
                &= \frac{1}{n+1} + \frac{1}{n(n+1)} && \qedhere
  \end{align*}
\end{proof}

It follows that:

\begin{proposition}
Any $\dfrac{2}{n}$ can be decomposed into Egyptian fractions. \label{th:decompose-of-2-n}
\end{proposition}

\begin{proof}
  \begin{align*}
    \frac{2}{n} &= \frac{1}{n} + \frac{1}{n} && \\
                &= \frac{1}{n} + \frac{1}{n+1} + \frac{1}{n(n+1)} & \text{by proposition \ref{th:egyptian-fraction-split}} & \qedhere
  \end{align*}
\end{proof}

We can obtain a better result with $n$ is odd: $\dfrac{2}{n}$ can be decomposed into two Egyptian fractions.

\begin{proof}
  \begin{align*}
    \frac{1}{n} &= \frac{1}{n+1} + \frac{1}{n(n+1)} & \text{by proposition \ref{th:egyptian-fraction-split}} & \\
    \frac{2}{n} &= \frac{2}{n+1} + \frac{2}{n(n+1)} & \text{both sides} \times 2 & \\
                &= \frac{1}{\frac{1}{2}(n+1)} + \frac{1}{n\frac{1}{2}(n+1)} & n + 1\text{ is even; nominator and denominator} \div 2 &\qedhere
  \end{align*}
\end{proof}

Further, \cref{qn:unique-decompose-r} asks to show the decomposition is unique when $n$ is odd prime. As the consequence of this fact, there was a table in Rhind papyrus, which list all decomposition from $\dfrac{2}{5}$ to $\dfrac{2}{101}$. Back to the theorem proven by Fibonacci, for any irreducible fraction\footnote{A fraction is irreducible if it can't be reduced; it means there is no common divisor of the nominator and denominator except 1.}, as one may convert an improper fraction\footnote{A fraction is improper if the nominator is greater than its denominator.} to a mixed number, it's sufficient to show:

\index{Fibonacci's theorem}
\begin{theorem}[Fibonacci]
Any irreducible proper fraction $\dfrac{b}{a}$ can be decomposed to Egyptian fractions.
\end{theorem}

\begin{proof}
Fibonacci applied the division with remainders: $a = bq + r$, where $q$ is the quotient\footnote{The term quotient derives from the Latin word \emph{quotiēns}, which is an adverb meaning ``how many times'' or ``how often''.}, $r$ is the remainder and $0 < r < b$. The fraction that is most close to and less than $\dfrac{b}{a}$ is $\dfrac{1}{q + 1}$, and their difference is
\[
  \frac{b}{a} = \frac{1}{q + 1} + x
\]

With the idea of divide and conquer, Fibonacci converted the original problem to a sub-problem of decomposing $x$ into Egyptian fractions.

\begin{align*}
  x & = \frac{b}{a} - \frac{1}{q + 1} = \frac{bq + b - a}{a(q + 1)} & \text{common denominator} \\
    & = \frac{b - (a - bq)}{a(q+1)} = \frac{b - r}{a(q+1)} & \text{remainder } r = a - bq
\end{align*}

The nominator of $x$ is $b' = b - r$. As the remainder $0 < r < b$, the nominator $b' < b$, and it reduces,comparing to the nominator $b$ of the original fraction $\dfrac{b}{a}$, at least 1. If $b' = 1$, we are done; otherwise we need decompose $x$, a fraction of a smaller nominator. In this way, the nominator keeps decreasing as a sequence of $b > b' > b'' > \cdots$. For any given integer $b$, it can't infinitely decrease, but reach 1 finally, and the decomposition terminates. It proves any irreducible proper fraction can be decomposed to Egyptian fractions.
\end{proof}

For example, to decompose $\dfrac{5}{11}$ in Fibonacci's method: $11 = 2 \times 5 + 1$, the quotient $q = 2$, the remainder $r = 1$.

\begin{align*}
\frac{5}{11} & = \frac{1}{q + 1} + x = \frac{1}{3} + x  \\
 x &= \frac{5}{11} - \frac{1}{3} = \frac{4}{33} \\
\frac{5}{11} &= \frac{1}{3} + \frac{4}{33}
\end{align*}

We next decompose $\dfrac{4}{33}$. Apply the division with remainder again: $33 = 4 \times 8 + 1$, the quotient $q = 8$, the remainder $r = 1$.

\begin{align*}
\frac{4}{33} & = \frac{1}{q + 1} + x = \frac{1}{9} + x  \\
 x &= \frac{4}{33} - \frac{1}{9} = \frac{12 - 11}{99} = \frac{1}{99}
\end{align*}

We are done, the decomposition is
\[
\frac{5}{11} = \frac{1}{3} + \frac{1}{9} + \frac{1}{99}
\]

\index{greedy method}
Fibonacci always chose the unit fraction that was the \emph{closest} to $\dfrac{b}{a}$. Such strategy is called the \emph{greedy} method. However, it may not necessary be the best decomposition. For example,

\[
\frac{5}{121} = \frac{1}{25} + \frac{1}{757} + \frac{1}{763309} + \frac{1}{873960180913} + \frac{1}{1527612795642093418846225}
\]

While there is another decomposition of
\[
\frac{5}{121} = \frac{1}{33} + \frac{1}{121} + \frac{1}{363}
\]

We define the `best' decomposition to be the one with the \underdot{fewest} Egyptian fractions, if two ways of decomposition share the same number of Egyptian fractions, the less denominator the better. \Cref{app:best-egyptian-decomposition} gives an algorithm that searches the best decomposition. Because Fibonacci's method reduces the nominator at least by 1 each time, the decomposition of $\dfrac{b}{a}$ need at most $b$ Egyptian fractions. However, the above example of $\dfrac{5}{121}$ implies there may be better solution. Erdős and Straus conjectured in 1950, that any natural number $n$ greater than 1 can always be decomposed into three Egyptian fractions, \index{Erdős-Straus conjecture}

\[
\frac{4}{n} = \frac{1}{a} + \frac{1}{b} + \frac{1}{c}
\]

Where $a < b < c$. This interesting problem, known as Erdős-Straus conjecture, hasn't been solved as of 2025.
% https://mathworld.wolfram.com/Erdos-StrausConjecture.html

\begin{mdframed}

\begin{center}
 \includegraphics[scale=0.3]{img/erdos-1992}
 \captionof{figure}{Paul Erdős, 1913-1996}
 \label{fig:erdos-1992}
\end{center}

\index{Erdős} \label{sec:Erdos}
Paul Erdős was a Hungarian `freelance' mathematician known for his work in number theory and combinatorics, and a legendary eccentric who was arguably the most prolific mathematician of the 20th century\cite{Hoffman-Paul-2025}. He published over 1500 papers (including coauthored with others, this number surpassed Euler). He traveled to many places, visited mathematicians and worked with them. Erdős himself published papers with 507 coauthors\footnotemark; this made him the center of a large network of mathematical community. This allows people to define `Erdős number' that measures the influence of a mathematician. Those 507 coauthors gained the coveted distinction of having an `Erdős number of 1', meaning that they wrote a paper with Erdős himself. Those who published a paper with one of Erdős's coauthors had an Erdős number of 2; an Erdős number of 3 meant that someone wrote a paper with someone having Erdős number of 2; and so on. Albert Einstein's Erdős number, for instance, was 2. The lowest median of Erdős number among Fields award winners was 3. The highest known Erdős number is 15; this excludes non-mathematicians, who all have an Erdős number of infinity.

%% 例如 https://www.jmilne.org/math/Personal/index.html 页面底部的埃尔德什数

Erdős was born in 1913 in a Jewish family in Budapest, Hungary. His two elder sisters, aged three and five, both died of scarlet fever; his parents, fearing of his death, extremely cared him. The World War I broke out when Paul was not much over a year old. His father Lajos was captured by Russian army as it attacked the Austro-Hungarian troops, and spent six years in captivity in Siberia. Surprisingly, Lajos self-learned English to pass the long hours in captivity, but had strange accent for having no English teacher. Erdős was taught English by his father, which resulted in characteristic of the strange accent throughout life. The situation in Hungary was chaotic at the end of World War I, the strikes happened here and then. Erdős's mother Anna, as a head teacher, insisted on teaching at school even under a call for strike action. She continued working simply because didn't want children's education suffer; but this became the reason she was dismissed from her post after political power changed.By 1920s, anti-Jewish law was introduced to Hungary similar to what happened in Germany 13 years later; Europe was heading to the war.

Despite the bad situation and the restriction on Jews, Erdős entered the University Pázmány Péter in Budapest as the winner of national examination. He was awarded a doctorate in 1934, then forced to leave Hungary because he was Jewish. Erdős went to Manchester UK for a post-doctoral fellowship. He travelled widely in the UK; during that time, he met Hardy and Ulam in Cambridge. The latter became his life-lone friend. He was unable to return to Hungary because of the tough situation for Jewish by the late 1930s. When Hitler took control of Austria and Czech, Erdős had to cancel his plan to visit his parents in the middle way in 1938. He hurried returning to English, then after a few weeks, went to US for a one year fellowship at the Institute for Advanced Study in Princeton, New Jersey, where he cofounded the field of probabilistic number theory. He hoped for a renewal of fellowship, but was offered only a six month extension. At this hard time, his friend Ulam invited Erdős to visit University of Wisconsin-Madison to help out. This starts Erdős's strange life: he wandered about the US (then later to other countries) from one university to the next - Purdu, Stanford, Notre Dame, Johns Hopkins - devoted exclusively to seeking out and solving good mathematical problems.

Erdős had definite ideas about mathematical elegance for his early career. He believed that God had a transfinite book (“transfinite” being a mathematical concept for something larger than infinity) that contained the shortest, most beautiful proof for every conceivable mathematical problem. The highest compliment he could pay to a colleague’s work was to say, ``That’s straight from The Book.'' In 1845 Bertrand conjectured that there was always at least one prime between $n$ and $2n$ for $n \geq 2$; Chebyshev proved Bertrand's conjecture in 1850 but when Erdős was only an eighteen year old student in Budapest he found an elegant elementary proof of this result. Erdős was proud that he had found The Book proof. In 1998, the 85 anniversary of Erdős's birth, Martin Aigner and Günter M. Ziegler compiled the most elegant proofs to 32 theorems (one theorem often containing multiple proofs) into a book titled \emph{Proofs from THE BOOK}.

In the World War II, Erdős fully lost connections with his family in Hungary. After the war in August 1945, he received a telegram giving details of his family: his father had died of a heart attack in 1942; his mother had survived; four of his uncles and aunts had been murdered. Quite remarkably, a cousin had been sent to Auschwitz but had survived. Near the end of 1948, he was able to return to Hungary for a visit where he reunited with his surviving family and friends\cite{MacTutor-Erdos-2000}. For the next half a century, Erdős travelled frequently. He rejected any full-time job offers so that he would have the freedom to work with anyone at any time on any problem of his choice. The nomadic existence made him a legend in the mathematics community. With no home, no wife, and no job to tie him down, his wanderlust took him to Israel, China, Australia, and 22 other countries (although sometimes he was turned away at the border—during the Cold War, Hungary feared he was an American spy, and the United States feared he was a communist spy). Erdős would show up — often unannounced — on the doorstep of a fellow mathematician, declare ``My brain is open!'' and stay as long as his colleague served up interesting mathematical challenges.

Erdős was elected to many of the world’s most prestigious scientific societies, including the Hungarian Academy of Science (1956), the U.S. National Academy of Sciences (1979), and the British Royal Society (1989). In 1951 John von Neumann presented the Cole Prize to Erdős for his work in prime number theory. In 1984 he won the most lucrative award in mathematics, the Wolf Prize, and used all but \$720 of the \$50,000 prize money to establish a scholarship in his parents’ memory in Israel. Erdős went on proving and conjecturing until the age of 83, succumbing to a heart attack only hours after disposing of a nettlesome problem in geometry at a conference in Warsaw in 1996.
\end{mdframed}
\footnotetext{Some saying 511 coauthors. Over 70 papers with his name have appeared since his death\citepage[p588]{Stillwell-2010}.}

\section{Babylonian fractions}
When looking back, Egyptian fractions were hard to use and compute; it was in side way out of the main stream of mathematical history. It has more formal significance than its practical applications. Neither the Greeks nor the Romans developed fractions in arithmetic independently, instead they introduced dedicated terms to represent partial quantities. A unit of weight was \emph{as} and the \emph{unica} (from which derived ``ounce'' in English) was a twelfth part of the as. \Cref{tab:roman-fractions} list most used terms for partial quantities in ancient Roman.

\index{Roman fractions}

% https://mathshistory.st-andrews.ac.uk/Miller/mathsym/fractions/
\begin{table}[htbp]
  \centering
  \renewcommand{\arraystretch}{1.5}
  \begin{tabular}{|c|l|l||c|l|l|}
  \hline
  value & term & meaning & value & term & meaning \\
  \hline
  $\frac{1}{12}$ & uncia    & one twelfth & $\frac{9}{12}$ & dodrans    & a quarter taken away \\
  \hline
  $\frac{2}{12}$ & sextans  & one sixth & $\frac{10}{12}$ & dextans   & one sixth taken away \\
  \hline
  $\frac{3}{12}$ & quadrans & a quarter & $\frac{11}{12}$ & deunx    & a uncia taken away \\
  \hline
  $\frac{4}{12}$ & triens   & one third & $\frac{1}{24}$ & semuncia   & half a uncia  \\
  \hline
  $\frac{5}{12}$ & quincunx & five uncia & $\frac{1}{48}$ & sicilicus &  \\
  \hline
  $\frac{6}{12}$ & semis    & half    & $\frac{1}{72}$ & scriptulum    &  \\
  \hline
  $\frac{7}{12}$ & septunx  & seven uncia & $\frac{1}{144}$ & scripulum  &  $\frac{1}{12} \times \frac{1}{12}$ \\
  \hline
  $\frac{8}{12}$ & bes      & two third & $\frac{1}{288}$ & scrupulum    &  $\frac{1}{24} \times \frac{1}{12}$\\
  \hline
  \end{tabular}
  \caption{Roman fractions}
  \label{tab:roman-fractions}
\end{table}

Many terms left in our languages nowadays. For example, from `semi', the term of $\dfrac{1}{2}$, derived semicircle, semicolon, semiconductors and so on. The term `quadrans' for $\dfrac{1}{4} = \dfrac{3}{12}$ became `quarter' in English, where derived a quarter of an hour and a quarter of a year.

\index{Babylonian decimals}
We know Babylonian fractions from some material evidence: a tablet collected in Yale University as shown in \cref{fig:babylonian-yale}. It inscribed a square and its diagonal. Aside a side of the square inscribed 30 (3 wedges on up right), below the diagonal inscribed 42, 25, 35. It was also inscribed 1, 24, 51, 10 along the diagonal. We learned from school math that the length of the diagonal is the $\sqrt{2}$ times of the side, hence is $30 \times \sqrt{2} \approx 30 \times 1.4142 = 42.426$ for the side of 30. Note that the integral part, 42, is the first number inscribed below the diagonal. As the Babylonian used 60 base, are the followed numbers 25 and 35 the fractional part? Let's verify:

\begin{figure}[htbp]
 \centering
 \includegraphics[scale=0.8]{img/babylonian-yale}
 \caption{Tablet with identifier Y7289 collected in Yale University.}
 \label{fig:babylonian-yale}
\end{figure}

\[
\frac{25}{60} + \frac{35}{60^2} \approx 0.4264
\]

It showed that the Babylonian correctly gave the length of the diagonal: the value of 60-based numbers 42, 25, 35 is $42.426 \approx 30 \times \sqrt{2}$. Then is the sequence of 1, 24, 51, 10 another fraction? Let's try:

\[
1 + \frac{24}{60} + \frac{51}{60^2} + \frac{10}{60^3} \approx 1.414213
\]

It is the approximation of $\sqrt{2}$, with precision up to 6 places! While these numbers are ambiguous for lack of the point symbol, for example, the sequence 12, 15 may represent $12 \times 60 + 15 = 720$ or $12 + \dfrac{15}{60} = 12.25$. Moreover, Babylonian didn't use zero in a unified way, people have to guess the exact value from the context.

As the base of 60 was big enough, the Babylonian could get sufficiently fine quantity from $\dfrac{1}{60}$ to $\dfrac{59}{60}$. With 2 to 3 places, the fractional part was precise enough for daily life or astronomical observation. Besides, base of 60 has plenty of proper factors\footnote{A factor being not of 1 or $n$ is called proper factor of $n$.} including 2, 3, 4, 5, 6, 10, 12, 15, 20, 30, hence has less chance to repeat when dividing (comparing to 10, which only has two proper factors of 2 and 5, leads to repeating decimals in common, see \cref{thm:cyclic-decimal}). These might be the reasons that the Babylonian didn't developed general fractions.

\section{Chinese fractions}
\label{sec:chinese-fractions}

\index{Nine chapters on the Art of mathematics}
People find complete arithmetic rules for fractions with varies of problems from \emph{Nine chapters on the Art of mathematics}, a classic Chinese mathematical book in Han Dynasty (about 1st century CE, often abbreviated as \emph{Nine chapters}). There was a dedicated chapter (see \cref{fig:jiuzhang}) about fractions. It started from the notation of ``$a$分之$b$'', meaning $\dfrac{b}{a}$, which is still using in Chinese today. The denominator $a$ comes first then follows with $b$; read as out of $a$ parts taken $b$. The Chinese had mixed number, for example, 8 steps and $\dfrac{1}{3}$ step to measure the length of a piece of land. They derived the notion of fractions from division, for example from \emph{Nine chapters}: seven merchants sold four horses, each merchant sold $\dfrac{4}{7}$ horse.

\begin{figure}[htbp]
 \centering
 \includegraphics[scale=0.4]{img/jiuzhang}
 \caption{The Nine Chapters on the Mathematical Art}
 \label{fig:jiuzhang} % The Nine Chapters on the Mathematical Art
\end{figure}

\index{fraction!reduce}
With the notation of fractions, the author of \emph{Nine chapters} next dealt with the problem that whether the value of a fraction was definitive, in other word, will multiple notations correspond to the same value? This leads to the reduction operation and the concept of lowest terms (irreducible fraction). The author explained with example of problems:

\begin{enumerate}[1)]
\item Now there is twelve eighteenth ($\dfrac{12}{18}$), if reduce to its lowest terms, what is the result? The answer is two third. ($\dfrac{12}{18} = \dfrac{2}{3}$)
\item There is forty nine ninety-first ($\dfrac{49}{91}$), if reduce to its lowest terms, what is the result? The answer is seven thirteenth. ($\dfrac{49}{91} = \dfrac{7}{13}$)
\end{enumerate}

\index{Euclidean algorithm} \label{sec:gcd-minus}
How to reduce a fraction to its lowest terms? the author of \emph{Nine chapters} gave a method which was exactly known as Euclidean algorithm in the west, ``First, halve the nominator and denominator repeatedly if both are even; second, for the two numbers of the nominator and denominator, subtract the less one from the greater one repeatedly until they become the same value; finally, reduce the fraction with this value.'' For fraction $\dfrac{b}{a}$, if both $a$ and $b$ are even, divide each by 2 and repeat; otherwise, denote the greatest common divisor of $a$ and $b$ by $(a, b)$, then

\be
\label{eq:gcd-minus}
(a, b) = \begin{cases}
  a > b :& (a - b, b) \\
  a = b :& a \\
  a < b :& (a, b - a)
  \end{cases}
\ee

If $a = b$, the their greatest common divisor is $a$ (or $b$). If $a > b$, then subtract $b$ from $a$; it turns the original problem of finding the greatest common divisor of $a$ and $b$, to a new problem of finding the greatest common divisor of $a - b$ and $b$. If $a < b$, then subtract $a$ from $b$; the original problem turns to find the greatest common divisor of $a$ and $b - a$. We delay the proof to Euclidean algorithm to \cref{sec:Euclidean-algorithm}. Euclidean algorithm calculates the great common divisors much faster\footnote{The improved Euclidean algorithm in practice implements the repeat of subtraction in \cref{eq:gcd-minus} as division, which calculates in logarithmic time. For numbers of 100 digits, it can find the greatest common divisor in about 8 recursions.} than the method of factorization, which needs to first factorize $a$ and $b$ into primes, then pick the most common factors and calculate their product. It is suitable for computer machine implementation and becomes the fundamental algorithm in number theory and cryptography\footnote{Euclidean algorithm is also important in abstract algebra, such as ring and field theory.}. Below example shows the steps to find the greatest common divisor of 18 and 12 with Euclidean algorithm.

\[
(18, 12) = (18 - 12, 12) = (6, 12) = (6, 12 - 6) = (6, 6) = 6
\]

It gives the greatest common divisor in three steps. To reduce to the lowest terms, divide the nominator and denominator by 6: $\dfrac{12}{18} = \dfrac{12/6}{18/6} = \dfrac{2}{3}$.

The author of \emph{Nine chapters} next introduced addition for fractions with below example problems:

\begin{enumerate}[1)]
\item What is the sum of one third and two fifth? The answer is eleven fifth. ($\dfrac{1}{3} + \dfrac{2}{5} = \dfrac{11}{15}$)
\item What is the sum of two third, four seventh, and five ninth? The answer is one and fifty sixty-third. ($\dfrac{2}{3} + \dfrac{4}{7} + \dfrac{5}{9} = 1\dfrac{50}{63}$)
%% \item 又有二分之一,三分之二,四分之三,五分之四,问合之得几何?答曰:得二、六十分之四十三。(即:$\dfrac{1}{2} + \dfrac{2}{3} + \dfrac{3}{4} + \dfrac{4}{5} = 2\dfrac{43}{60}$)
\end{enumerate}

The author explained, `let the sum of the products of the each denominator and the other nominator be the nominator of the result; let the product of the two denominators be the denominator of the result.' that is,

\be
\frac{b}{a} + \frac{d}{c} = \frac{bc + ad}{ac}
\ee

As the result may not be in the lowest terms, the author continued with a special case, `the result is 1 if the nominator equals the denominator', and followed with the mixed number for the general case. At last the author addressed the special simple case of equal denominators: one adds the nominators directly.

\be
\frac{b}{a} + \frac{c}{a} = \frac{b + c}{a}
\ee

The author of \emph{Nine chapters} didn't reduce the fractions to their least common denominator terms, but just multiplied the denominators. We skip the subtraction part in \emph{Nine chapters} as it is similar to addition. It's not so obvious to pick the greater one from two fractions than two natural numbers, for example between $\dfrac{4}{7}$ and $\dfrac{3}{5}$. The author gave a remarkable general rule, which was essentially to compare the difference with zero by converting the two fractions of $\dfrac{b}{a}$ and $\dfrac{d}{c}$ to $bc$ and $ad$ and comparing. After defining the mean value of multiple fractions, the author extended it to the application, like, `Now there are seven persons dividing the money of eight and one third, how much does each individual get? The answer is one and four twenty-first for each.' It is $8\dfrac{1}{3} \div 7 = 1\dfrac{4}{21}$. The author explained, `Put the number of persons as the denominator, the total money as the nominator, then reduce it to the lowest term.' The author did not consider the application of fractional denominator. The last part was about the multiplication of fractions.

\begin{enumerate}[1)]
\item Now there is a land with length of four seventh a step and width of three fifth a step, what is the area? The answer is twelve thirty-fifth of a square step ($\dfrac{4}{7} \times \dfrac{3}{5} = \dfrac{12}{35}$).
\item There is a land with length of seven ninth a step and width of nine eleventh a step, what is the area? The answer is seven eleventh a square step ($\dfrac{7}{9} \times \dfrac{9}{11} = \dfrac{7}{11}$).
%% \item 又有田广五分步之四,从九分步之五,问为田几何?答曰:九分步之四。(即:$\dfrac{4}{5} \times \dfrac{5}{9} = \dfrac{4}{9}$)
\end{enumerate}

The rule, given by the author was, `The denominators and nominators are product of the two denominators and nominators respectively.' then converted to mixed number. Interestingly, there was dedicated part about the multiplication of mixed numbers:

\begin{enumerate}[1)]
\item Now there is a land with length of three steps and one third a step, width of five steps and two fifth a step, what is the area? The answer is eighteen square steps ($3\dfrac{1}{3} \times 5\dfrac{2}{5} = 18$).
\item There is a land with length of seven steps and three fourth a step, width of fifteen steps and five ninth a step, what is the area? The answer is one hundred and twenty square steps and five ninth square step ($7\dfrac{3}{4} \times 15\dfrac{5}{9} = 120\dfrac{5}{9}$).
%% 又有田广十八步七分步之五,从二十三步十一分步之六,问为田几何?答曰:一亩二百步十一分步之七。
\end{enumerate}

The author first converted the mixed numbers to fractions (with the nominator being greater than the denominator), then multiplied them and convert the result back to mixed numbers as below:

\[
3\frac{1}{3} \times 5\frac{2}{5} = \frac{10}{3} \times \frac{27}{5} = \frac{10 \times 27}{3 \times 5} = 18
\]

The Chinese saw the importance of fractions; the \emph{Nine chapters} actually started from the chapter of fractions. All the following chapters about areas, shapes, equations, and so on were built on top of fractions.

\section{Hindu fractions and decimals}

\index{Hindu fractions}
The notation for fractions we are using comes from India, which write the nominator above the denominator, but without the fraction line:

\begin{align*}
  3 \\
  4
\end{align*}

\index{fraction line}
Which denote three fourth. After the fraction line that separated the nominator and denominator was introduced by the Arabic around 1200 CE\cite{Pumfrey-2011}, it became:

\[
\dfrac{3}{4}
\]

Fibonacci was the first who introduced the modern notation of fractions to Europe. However, he adopted the right to left writing direction from the Arabic, consequently, the fractional part was written to the left of the integral part\cite{Miller-2025}. The fraction line posed considerable printing difficulties; Thomas Twining, the founder of renowned British Twining Tea company, used a slash instead of a horizontal fraction line in accounting book in 1718, for example: 1/4 pound of tea. This made it easy to print and type fractions using a typewriter. Today, many word processing and editing applications support typing fractions conveniently, with \LaTeX convention for example, the input \lstinline|\frac{b}{a}| renders $\dfrac{b}{a}$.

\index{decimal} \index{Liu Hui}
With Hindu-Arabic numeral system, the Arabic mathematicians introduced decimals. For example, Al-Kashi (920 - 980 CE) in his \emph{al-Risali al-mohitije} (Treatise on the circumference) wrote $\pi$ as `sah-hah 3 14159', where sah-hah means the integral part (corresponding to sahih in Turkish today) 3, then followed by the decimal part. There hadn't been the point symbol yet. Liu Hui, a Chinese mathematician of the Wei-Jin period (c. 225 - 295 CE, \cref{fig:liuhui}) who annotated the \emph{Nine chapters}, was not satisfied with the value 3 of $\pi$ in the book. He decided to approximate it with decimals. When using a regular 96-gon to approximate the circle, he obtained the value of $314\dfrac{169}{625}$ cun\footnote{1 cun $\approx$ 2.31 cm in Han Dynasty.} with a diameter of 100 cun. He thought this value, which was about 3.14, exceeded $\pi$; he followed by doubling the edges to 192-gon, which approximated to $314\dfrac{4}{25} = 314.16$ cun.

\begin{figure}[htbp]
 \centering
 \subcaptionbox{John Napier, oil painting, 1616, in the collection of the University of Edinburgh. \label{fig:napier}}{\includegraphics[scale=0.5]{img/napier}} \quad
 % https://www.britannica.com/biography/John-Napier
 \subcaptionbox{Portrait of Liu Hui by Jiang Zhaohe, from a Chinese textbook. \label{fig:liuhui}}{\includegraphics[scale=0.46]{img/liuhui}}
 % https://wapbaike.baidu.com/tashuo/browse/content?id=b1fc49ab180770763a28420d
\end{figure}

\index{decimal point} \index{John Napier} \index{percent}
Christoff Rudolff (1499 - 1545) in 1530 separated the integral and decimal parts with a vertical bar. Magini (1555 - 1617) changed the bar to a point, but it was not widely adopted until John Napier's (1550 - 1617, see \cref{fig:napier}) work about logarithms was translated in 1616. The percentage symbol \% was invented by some anonymous Italian in 1425.

\index{base point}
A decimal can be considered as the sum of fractions with denominators of 10, 100, 1000, ... A percentage number looks like a fraction with denominator of 100, but it is actually the result by multiplying 100 to the decimal value. This is because its nominator may not be integral necessarily. For example, one says the availability of some internet service is very high, reaching 99.999\% (or five nines). It equals the decimal value of 0.99999; or written as per mille as 999.99\textperthousand. In some areas like finance or economics, people often use base points (BPS), 1bps = 0.01\% = 0.0001, for example, we often heard the news that Fed (Federal Reserve) announced a 25-bps interest rate cut, which meant the interest rate decreased by 0.25\% = 0.0025. If the previous interest rate is 2.5\%, then the new rate became 2.5\% - 0.25\% = 2.25\%.

\subsection{Fractions and decimals}

Is any decimal a sum of fractions with denominators of 10, 100, 1000, ...? Conversely, does any fraction correspond to a \emph{unique} decimal? For instance, 0.125 = $\dfrac{1}{10} + \dfrac{2}{100} + \dfrac{5}{1000}$, is the sum of fractions of $\dfrac{1}{10}$ (denominator of 10), $\dfrac{2}{100}$ (denominator of 100), and $\dfrac{5}{1000}$ (denominator of 1000). In general a decimal $0.a_1 a_2 \dots a_n$ can be expressed as:

\be
a_1 \frac{1}{10} + a_2 \frac{1}{100} + \cdots + a_n \frac{1}{10^n}
\label{eq:decimal-as-sum}
\ee

\index{repeating decimals}
We learn from school math that there are terminating decimals, like 0.125, and non-terminating decimals, also known as infinite decimals, like $0.\dot{3} = 0.333\cdots, 0.\dot{1}\dot{5} = 0.1515\cdots, \pi = 3.14159\cdots$. By \cref{eq:decimal-as-sum}, an infinite decimal is then a sum of infinitely many fractions. It naturally arises the question: is the sum of such infinitely many positive numbers finite or infinite? Here is a proposition that looks counter-intuitive:

\begin{lemma} \label{th:cycle-of-9}
  The repeating decimal $0.999\cdots$ equals 1.
\end{lemma}

The left side $0.999\cdots$ looks obviously different from the right side 1, why are they equal?

\begin{proof}
  Let $x = 0.999\cdots$, enlarge 10 times: $10x = 9.999\cdots$, and subtract:
  \begin{align*}
    10x - x &= 9.999\cdots - 0.999\cdots && \\
         9x &= 9 & \text{the decimal parts get canceled.} & \\
          x &= 1  && \qedhere
  \end{align*}
\end{proof}

The proposition does not hold for finitely many 9. Even if there are $n = 10000$ of 9 because $9.99\cdots9$ (a 9 as whole number and 9999 places of 9) subtract $0.99\cdots9$ (a 0 as whole number and 10000 places of 9) gives $8.99\cdots91$, hence $x = \dfrac{8.99\cdots91}{9} \ne 1$. The 10 thousand digits terminating decimal $0.99\cdots9 \ne 1$. The proposition shows that the sum of infinitely many positives, $\dfrac{9}{10} + \dfrac{9}{100} + \cdots = 1$, is a finite value\footnote{Alternatively, one may prove with the concept of limitation: the decimal of $n$ places, $0.99\cdots9 = 1 - 0.00\cdots01 = 1 - \dfrac{1}{10^{n+1}}$; taking limit, $0.999\cdots = \lim\limits_{n\to\infty} (1 - \dfrac{1}{10^{n+1}}) = 1 - \lim\limits_{n\to\infty} \dfrac{1}{10^{n+1}} = 1 - 0 = 1$.}.

\begin{corollary}
  For any non-terminating decimal $0.a_1a_2\cdots$, the sum of infinitely many fractions: $a_1 \dfrac{1}{10} + a_2 \dfrac{1}{100} + \cdots$ is finite.
\end{corollary}

\begin{proof}
  For every place $0 \leq a_i \leq 9$, the sum
  \begin{align*}
0 \leq 0.a_1 a_2 \cdots = a_1\frac{1}{10} + a_2\frac{1}{10^2} + \cdots \leq \frac{9}{10} + \frac{9}{10^2} + \cdots = 0.999\cdots = 1 & \qedhere
  \end{align*}
\end{proof}

\index{paradoxes of Zeno!Achilles paradox}
Though the fact of $0.a_1a_2\cdots \leq 1$ looks obvious, it's a remarkable mathematical and logical statement that the sum of infinitely many positive values are finite. It's the key to solve a paradox with a history over two thousand years. Zeno of Elea (about 490 - 425 BCE) a Greek philosopher, designed four paradoxes; the first one called Achilles paradox was the most popular. Achilles, the greatest warrior in Trojan war according to Homer's \emph{Iliad}, was fleet-footed. However, Zeno argued that Achilles when started behind a slow-moving tortoise in a race, would never catch up with it. Because Achilles must first reach the point $A$ where the tortoise started, by which time the tortoise would have moved ahead to point $B$; by the time Achilles traversed the distance to this latter point $B$, the tortoise would have moved ahead to another point $C$, and so on. Although the distance between Achilles and the tortoise got closer and closer, Achilles was always left behind the tortoise, as shown in \cref{fig:Achilles-paradox}.

\begin{figure}[htbp]
 \centering
 \includegraphics[scale=0.4]{img/achilles-paradox}
 \caption{Achilles paradox}
 \label{fig:Achilles-paradox}
\end{figure}

This is counter our common sense. We learn from school math how to solve such problem through the method of `relative motion'. Denote the speed of the tortoise by $v_1$, the speed of Achilles by $v_2$; the relative speed of Achilles to the tortoise was $v_2 - v_1$. If the tortoise started head of Achilles by $s$, then Achilles would catch up and overtook the tortoise by the time of $t = \dfrac{s}{v_2 - v_1}$. However, it was not until Galileo, that the concept of relative motion was established systematically. Zeno's argument seemed so sound that the paradox attracted many great minds in history. Lewis Carrol and Douglas Hofstadter took Achilles and the tortoise as figures in their literary works.

In Analysis, it's common to solve Achilles paradox with the concept of limitation. Alternatively, we may solve it with the tool: the decimal as the sum of fractions. Let's quantify the problem for easy understanding. Suppose Achilles was 10 times faster than the tortoise, i.e., $v_2 : v_1 = 10$; the tortoise started a head by the distance $s = 100$ pous\footnote{1 pous $\approx$ 0.308 feet in ancient Greece.}. According to Zeno, Achilles need first traverse $s = 100$ pous, by which time the tortoise would have moved $\dfrac{1}{10}s = 10$ pous ahead; then Achilles traversed this 10 pous, by which time the tortoise would have moved another$\dfrac{1}{100}s = 1$ pous, ... the total distance Achilles need to traverse would be:

\[
s(1 + \frac{1}{10} + \frac{1}{100} + \cdots) = 100\times(1 + 0.1 + 0.01 + \cdots) = 100 \times 1.11\cdots = 111.11\cdots
\]

Suppose the fleet-footed Achilles run 100 pous in 9 second; he need run another 0.9 second for the next 10 pous, and another 0.09 second for the next 1 pous, ..., the total time would be:

\begin{align*}
9 + 0.9 + 0.09 + \cdots &= 9.99\cdots = 10 \times 0.99\cdots & \\
                        &= 10 \times 1 & \text{by Lemma \ref{th:cycle-of-9}} \\
                        &= 10
\end{align*}

Hence Achilles caught up the tortoise in 10 seconds according to Zeno's logic.

\subsection{Repeating decimals}
\index{repeating decimals}
To convert irreducible fraction $\dfrac{b}{a}$ to decimal, one divides $b$ by $a$. The division terminates sometimes, for example, $\dfrac{1}{2} = 0.5$, $\dfrac{1}{8} = 0.125$, $\dfrac{1}{25} = 0.4$, otherwise repeating, for example, $\dfrac{1}{3} = 0.\dot{3} = 0.333\cdots$, $\dfrac{1}{7} = 0.\dot{1}\dot{4}\dot{2}\dot{8}\dot{5}\dot{7}$. In what condition does it terminate? and in what condition does not? It seems depend on our luck; in fact, we are very unlucky: it repeats in common. Below theorem tells how unlucky we are.

\index{terminating decimals}
\begin{theorem} \label{thm:finite-decimal}
  A irreducible fraction $\dfrac{b}{a}$ can be expressed as terminating decimal if and only if $\dfrac{b}{a} = \dfrac{b}{2^{\alpha}5^{\beta}}$, where $\alpha$ and $\beta$ are non-negative integers.
\end{theorem}

\begin{proof}
Let $n = \max(\alpha, \beta)$, which is the greater one of $\alpha$ and $\beta$, thereby, $n - \alpha \geq 0$ and $n - \beta \geq 0$. Because the product,

\[
10^n \frac{b}{a} = \frac{2^n 5^n b}{2^{\alpha} 5^{\beta}} = 2^{n - \alpha} 5 ^{n - \beta} b
\]

is integral, there are $n$ places in the decimal of $\dfrac{b}{a}$, which is $0.a_1 a_2 \cdots a_n$, a terminating decimal.

Conversely, expand the $n$-places decimal $0.a_1 a_2 \cdots a_n$ to sum of fractions:

\begin{align*}
\frac{a_1}{10} + \frac{a_2}{10^2} + \cdots + \frac{a_n}{10^n} &= \frac{a_1 10^{n-1} + a_2 10^{n-2} + \cdots + a_n}{10^n} & \text{sum} \\
  &= \frac{B}{10^n} & \text{denote the nominator by }B \\
  &= \frac{B}{2^n 5^n} = \frac{b}{a} & \text{in the lowest terms}
\end{align*}

Because the factors of the denominator $10^n = 2^n 5^n$ are all powers of 2 and 5, so are the factors of the lowest term $a$.
\end{proof}

It turns out if there is any factor besides 2 and 5 for any irreducible fraction, like 3, 7, 11, ... then the decimal does not terminate. We may extend this theorem to $b$-based decimals, which don't terminate if the denominator contains any factor that does not divide $b$. This explains why the repeating decimal $\dfrac{5}{12} = 0.41666\cdots$ terminates as Babylonian 60-based decimal 0, 25; and why $\dfrac{5}{14}$ repeats even as Babylonian decimal. In particular, any fractions not in the form of $\dfrac{b}{2^n}$ does not terminate as binary decimal, which leads to rounding error in computer system.

The decimal of fractions either terminates or repeats (the next chapter shows fraction can't produce non-terminating non-repeating decimal); Among those repeating, some have periodic cycle of 1 digit, like $\dfrac{1}{3} = 0.\dot{3}$, some have cycle of 2 digits, like $\dfrac{5}{11} = 0.\dot{4}\dot{5}$, some even longer, like $\dfrac{1}{7} = 0.\dot{1}\dot{4}\dot{2}\dot{8}\dot{5}\dot{7}$; some is pure repeating, some is mixed, like $\dfrac{5}{12} = 0.41\dot{6}$. What are the rules?

\begin{proposition}
The decimal of $\dfrac{a}{9}$ is $0.\dot{a} = 0.aaa\cdots$
\end{proposition}

\begin{proof}
  \begin{align*}
    \text{令}x &= 0.\dot{a} = 0.aaa\cdots && \\
           10x &= a.aaa\cdots && \\
       10x - x &= 9x = a.aaa\cdots - 0.aaa = a && \\
             x &= \frac{a}{9} && \qedhere
  \end{align*}
\end{proof}

It explains $\dfrac{1}{3} = \dfrac{3}{9} = 0.\dot{3} = 0.333\cdots$, further,

\begin{proposition} \label{th:cyclic-decimal}
The fraction of the repeating decimal $0.\dot{a_1} \dot{a_2} \cdots \dot{a_n}$ is $\dfrac{a_1 a_2 \cdots a_n}{99 \cdots 9} = \dfrac{a_1 a_2 \cdots a_n}{10^{n+1} - 1}$.
\end{proposition}

We leave the proof as \cref{qn:cyclic-decimal}. Let's verify it with $\dfrac{1}{7}$:

\begin{center}
\opmul[displayshiftintermediary=all,
       voperator=bottom,
       voperation=top]{142857}{7}
\end{center}

Hence,
\[
0.\dot{1}\dot{4}\dot{2}\dot{8}\dot{5}\dot{7} = \frac{142857}{999999} = \frac{142857}{142857 \times 7} = \frac{1}{7}
\]

\index{periodic cycle}
By this proposition, we can easily tell the decimal of some fractions, like $\dfrac{2}{11}$, $\dfrac{4}{33}$, the former is $\dfrac{2}{11} = \dfrac{2 \times 9}{11 \times 9} = \dfrac{18}{99} = 0.\dot{1}\dot{8}$. The proposition also gives clue to the periodic cycle. If the lowest terms of $\dfrac{a_1 a_2 \cdots a_n}{99 \cdots 9}$ is $\dfrac{b}{a}$, then $ak = 999 \cdots 9$ for some integer $k$. It turns out that $ak + 1 = 100 \cdots 0$. If the remainder of $100\cdots 0$ (total $n$ digits of 0) divided by $a$ is 1, then the periodic cycle of $\dfrac{b}{a}$ contains \emph{at most} $n$ digits. The periodic cycle only depends on the denominator\footnote{In number theory, Fermat's theorem asserts the remainder of a prime $p$ dividing $10^{p-1}$ is 1. For example, the remainder of 7 dividing 1000000 is 1. The periodic cycle of $\dfrac{b}{7}$ contains 6 digits. Although the remainder of 11 dividing $10^{11-1} = 10000000000$ is 1, the periodic cycle of $\dfrac{2}{11} = 0.\dot{1}\dot{8}$ contains 2 digits. We need additional tool, i.e., Euler theorem in number theory, to get the definitive number of digits of a periodic cycle. Refer to chapter IX of \emph{An introduction to the theory of numbers} by Hardy and Wright.}. Below theorem answers where the periodic cycle starts.

\begin{theorem}\label{thm:cyclic-decimal}
For irreducible proper fraction $\dfrac{b}{2^{\alpha} 5^{\beta} q}$, where $q$ can't be divided by 2 or 5, let $n = \max(\alpha, \beta)$, then the decimal starts repeating after the decimal point by $n$ places, and the periodic cycle contains $m$ digits, where the remainder of $10^{m+1}$ divided by $q$ is 1.
\end{theorem}

\begin{proof}
We first show the first $n$ places don't repeat:

\[
10^n \frac{b}{2^{\alpha} 5^{\beta} q} = \frac{2^n 5^n b}{2^{\alpha} 5^{\beta} q} = \frac{2^{n - \alpha} 5^{n - \beta} b}{q} = X + \frac{b'}{q}
\]

Which is the mixed number after multiplied by $10^n$, where $X$ is the whole number part satisfying $0 \leq X < 10^n$. Since the remainder of $10^{m+1}$ divided by $q$ is 1, it follows that $q$ divides $10^{m+1} - 1$:

\begin{align*}
  (10^{m+1} - 1) \frac{b'}{q} &= k b' = a_1 a_2 \cdots a_m & \text{is integral}\\
                \frac{b'}{q} &= \frac{a_1 a_2 \cdots a_m}{99\cdots9} & m\text{ digits of }9
\end{align*}
Which is a repeating decimal by proposition \ref{th:cyclic-decimal}.
\end{proof}

For example, $\dfrac{5}{12} = 0.41\dot{6}$, where $12 = 2^2 \times 5^0 \times 3$, the greater one of the exponents 2 and 0 is 2, hence the periodic cycle starts after 2 places; as $q = 3$, the remainder of $10^1 / 3$ is 1, the cycle contains 1 digit.

\begin{mdframed}

%% \begin{center}
%%  \includegraphics[scale=0.3]{img/zeno}
%%  \captionof{figure}{芝诺,约490BC - 425BC}
%%  \label{fig:Zeno-of-Elea}
%% \end{center}

\index{Zeno of Elea}
Zeno of Elea (about 495 - 430 BCE) was a Greek philosopher and mathematician, whom Aristotle called the inventor of dialectic. We know very little about his life; the main source of our knowledge of Zeno comes from the dialog \emph{Parmenides} written by Plato. He was born in Elea, which was a Greek colony located in present-day southern Italy. Zeno was a pupil and friend of the philosopher Parmenides and studied with him in Elea. According to Plato, Parmenides and Zeno visited Socrates in Athens in around 450 BCE. Unfortunately, no work by Zeno has survived, but according to Proclus, he wrote books that containing forty paradoxes concerning the continuum, four among them are best known.

\begin{center}
 \includegraphics[scale=0.3]{img/zeno-paradox}
 \label{fig:Zeno-paradox}
\end{center}

\index{Zeno's paradoxes!dichotomy paradox}
Besides Achilles paradox, the other three are: dichotomy paradox, arrow paradox, and moving row paradox. The dichotomy paradox was about Atalanta, a virgin huntress in Greek mythology. Supposed Atalanta was going to walk to the end of a path. Before she could get there, she must travel halfway; before traveling halfway, she must travel a quarter; before traveling a quarter, she must travel one-eighth; and so on (\cref{fig:dichotomy-paradox}). This description requires one to complete infinitely many tasks, which Zeno believed was impossible. The paradoxical conclusion then would be that travel over any finite distance can neither be completed nor begin, and so all motion must be an illusion. \Cref{qn:dichotomy-paradox} asks to solve this paradox with an extension of Lemma \ref{th:cycle-of-9}.

\begin{center}
 \includegraphics[scale=0.4]{img/dichotomy-paradox}
 \captionof{figure}{Dichotomy paradox}
 \label{fig:dichotomy-paradox}
\end{center}

\index{Zeno's paradoxes!arrow paradox}
In arrow paradox, Zeno stated that for motion to occur, an object must change the position which it occupied. He gave an example of an arrow in flight (\cref{fig:Arrow-paradox}). In any (duration-less) instant of time, the arrow is neither moving to where it is, nor to where it is not. It can not move to where it is not, because no time elapses for it to move there; it can not move to where it is, because it is already there. In other words, at every instant of time there is no motion occurring. If everything is motionless at every instant, and time is entirely composed of instants, then motion is impossible. Whereas the first two paradoxes divide space, this paradox arises by dividing time - and not into segments, but into points.

\begin{center}
 \includegraphics[scale=0.4]{img/arrow-paradox}
 \captionof{figure}{arrow paradox}
 \label{fig:Arrow-paradox}
\end{center}

\index{Zeno's paradoxes!moving rows paradox}
The moving row paradox is also known as stadium paradox. It is also about dividing time into atomic points. As in \cref{fig:Moving-rows-paradox}, there are three rows in the stadium. Each row being composed of an equal number of bodies. At the beginning, they are all aligned. At the smallest time duration, row A stays, row B moves to the right one space unit, while row $\Gamma$ moves to the left one space unit. To row B, row $\Gamma$ actually moves two space units. It means, there should be time duration that $\Gamma$ moves one space unit relative to $B$. And it is the half time of the smallest duration. Since the smallest duration is atomic, it involves the conclusion that half a given time is equal to that time.

\begin{center}
 \includegraphics[scale=0.3]{img/moving-rows-paradox}
 \captionof{figure}{Moving rows paradox}
 \label{fig:Moving-rows-paradox}
\end{center}

Although Zeno's paradoxes are easy to understand, the arguments are confusing: motion and time are so real in our common sense, that Achilles must be able to catch up the tortoise. However, it is hard to solve Zeno's paradox. Aristotle, Archimedes, Russel, Weyl and etc., proposed varies of solutions to Zeno's paradoxes\cite{Britannica-Zeno-24}.
\end{mdframed}

\section{Fractions as numbers}
In some sense, fractions were quite different from natural numbers of 0, 1, 2, 3, ... a fraction has two compounds: nominator and denominator; it has different forms, including decimal, percentage, expression of division, and ratio ($a:b$), all are fractions essentially. The last two demonstrate a particular feature that natural numbers of 0, 1, 2, 3... do not have: $\dfrac{1}{2} = \dfrac{2}{4} = \dfrac{50}{100} = 3 \div 6 = 7:14 = \cdots$, a definitive value with \emph{multiple} representations. Varies of expressions may evaluate to the same value (or ratio).  Below are some questions when one wants to treat fractions and natural numbers uniformly as `numbers':

\begin{enumerate}[Q1.]
\item Which fractions (ratios) are identical?
\item How to compare two fractions? How to compare between a fraction and a whole number? Where is a fraction located on the number line?
\item What are the arithmetic operations for fractions, and for fractions and whole numbers?
\item Are the arithmetic laws applicable to fractions? For the multiplicative unit and additive unit of whole numbers, do they act same on fractions?
\end{enumerate}

\index{equivalence relation} \label{sec:frac-equiv}
For Q1, if $\dfrac{b}{a} = \dfrac{d}{c}$ holds, multiply $ac$ to both sides, it turns out $bc = ad$; this is known as `cross-multiplication' as shown in \cref{fig:cross-mul}. It also applies to the equation of ratios: the `inner' and `outer' products of $b : a = d : c$ are same, i.e., $bc = ad$. This is because the two forms, fraction and ratio,are essentially same. In mathematics, we call a relation $\sim$ \underdot{equivalence} if below three properties hold:

\begin{enumerate}[P1.]
\item Reflective: $a \sim a$;
\item Symmetric: if $a \sim b$, then $b \sim a$;
\item Transitive: if $a \sim b, b \sim c$, then $a \sim c$.
\end{enumerate}

\index{equivalence relation!fractions}
Let's verify $bc = ad$ is an equivalence relation satisfying them. First for the reflective property, comparing $\dfrac{b}{a}$ with itself, it's plainly that $ab = ab$; next for symmetric property, if $bc = ad$ holds for $\dfrac{b}{a}$ and $\dfrac{d}{c}$, then $ad = bc$ also hold for $\dfrac{d}{c}$ and $\dfrac{b}{a}$; last for transitive property, if $bc = ad$ holds for $\dfrac{b}{a}$ and $\dfrac{d}{c}$, and $de = cf$ holds for $\dfrac{d}{c}$ and $\dfrac{f}{e}$, multiply the two equations: $bcde = adcf$. Since the denominator $c \ne 0$, it can be canceled to give $bde = adf$. If $d \ne 0$ which can be further canceled to $be = af$; otherwise if $d = 0$, then $bc = ad = 0 = de = cf$, hence $b = f = 0$, which gives $be = af$ too. It follows the relation holds for $\dfrac{b}{a}$ and $\dfrac{f}{e}$.

\begin{figure}[htbp]
 \centering
 \includegraphics[scale=0.4]{img/cross-mul}
 \caption{cross-multiplication}
 \label{fig:cross-mul}
\end{figure}

\index{irreducible fraction} \index{lowest terms}
With this equivalence relation, one can tell any two fractions (and ratios) equivalent. Among all equivalent fractions, we chose the $\dfrac{b}{a}$, where $a, b$ have no common divisor except 1 (their greatest common divisor is $(a, b) = 1$, called coprime or relatively prime), as the \emph{representative}, which is called the irreducible fraction (or in the lowest terms). The only exception is 0, which we chosen as the representative of $\dfrac{0}{a}$. Such representative is unique for if $\dfrac{b}{a} = \dfrac{d}{c}$ both irreducible and their denominators are coprime, by the equivalence relationship $bc = ad$, it implies $c$ divides $ad$; meanwhile since $c$ and $d$ are coprime, $c$ divides $a$. It follows that $a$ and $c$ divides each other, hence identical, i.e., $a = c$. In the same way, $b = d$, which shows the representative irreducible fraction is unique.

It's natural to ask why need such equivalence relationship? why not check if the fractions reduce to the same lowest terms? People did compare by reduction at the beginning as we see in the book of \emph{Nine chapters} and as we were taught at school. However, as the numbers get bigger, people realized division is harder than multiplication. For example, it's a bit hard to compare $\dfrac{8723}{22814}$ and $\dfrac{143}{374}$ by reduction even with a calculator or applying Euclidean algorithm. But it's easy to multiply, which is fast, and compare the product of $8723 \times 374 = 3262402$ and $22814 \times 143 = 3262402$, which are obviously identical. This equivalence relationship can be extended to fractional expressions, for example, for $\dfrac{x^5 - 1}{x^3 - 1}$ and $\dfrac{1 + x + x^2 + x^3 + x^4}{1 + x + x^2}$, it's a bit hard to factorize\footnote{One solution is based on the obvious fact that $1^n - 1 = 0$, consequently, $x-1$ is a factor of $x^n-1$. Then by long division method, one can factorize it.} and reduce to lowest terms, but multiplication is much easier:

\begin{align*}
(x^3 - 1)(1 + x + x^2 + x^3 + x^4) &= \cancel{x^3} + \cancel{x^4} + x^5 + x^6 + x^7 - 1 - x - x^2 - \cancel{x^3} - \cancel{x^4} \\
  &= x^7 + x^6 + x^5 - x^2 - x - 1 \\
(x^5 - 1)(1 + x + x^2) &= x^5 + x^6 + x^7 - 1 - x - x^2 \\
  &= x^7 + x^6 + x^5 - x^2 - x - 1
\end{align*}

It shows the two fractions equal. This justifies why mathematicians generalize the abstract concept of equivalence relationship and chose the cross-multiplication as its application to fractions and ratios.

\vspace{3mm}
For Q2, we may reuse the method in \emph{Nine chapters}, comparing through the difference of $\dfrac{b}{a} - \dfrac{d}{c}$ with 0. It amounts to comparing the nominators after reducing them to the same denominators. As any whole number $n$ equals $\dfrac{n}{1}$, we unify the comparison across whole numbers and fractions simply to comparison of fractions. The ordering across whole numbers, fractions, and decimals sorts out their positions on the number line.

\vspace{3mm}
For Q3, the arithmetic operations were also given in \emph{Nine chapters}, which can be summarized as: for $+/-$, reduce to the common denominator then add/subtract the nominators; for $\times$, multiply the nominators and denominators respectively, followed by reduction. The reciprocal of a number $n \ne 0$ is $\dfrac{1}{n}$ because $n\dfrac{1}{n} = \dfrac{n}{n} = 1$. If $n$ is fractional, expressed as $\dfrac{b}{a}$, then its reciprocal is $\dfrac{a}{b}$ because $\dfrac{b}{a}\dfrac{a}{b} = 1$. It shows that a fraction is itself a expression of division, hence the reciprocal is a bridge between division and multiplication:

\[
\frac{b}{a} = b \div a = b \times 1 \div a = b \times \frac{1}{a}
\]

and

\[
\frac{b}{a} \div \frac{d}{c} = \frac{b}{a} \times 1 \div \frac{d}{c} = \frac{b}{a} \times \frac{c}{d}
\]

or

\begin{align*}
\frac{b}{a} \div \frac{d}{c} = \dfrac{\quad\dfrac{b}{a}\quad}{\dfrac{d}{c}} &= \frac{bc}{ad} & \text{denominator and nominator} \times ac
\end{align*}

By converting any whole number $n$ to $\dfrac{n}{1}$, we unify the arithmetic operations from whole numbers to fractions.

\vspace{3mm}
\index{close}
For Q4, fraction provides a unified way to express numbers: a whole number $n$ including 0 can be expressed as $\dfrac{n}{1}$, terminating and repeating decimals can be expressed as $\dfrac{b}{a}, a \ne 0$. This `family' of numbers (the set of fractions) is called \emph{rationals}, denoted by $\mathbb{Q}$. We'll explain this name in chapter 4. For every number $x$ of integers $\mathbb{Z}$, it has a negative (additive reverse) number $-x$, but only $\pm 1$ has reciprocal (multiplicative reverse). For every non-zero number $x$ of rationals $\mathbb{Q}$, it has a reciprocal (multiplicative reverse) $\dfrac{1}{x}$. The addition and subtraction are not closed for natural numbers $\mathbb{N}$, for example, $1 - 3 = -2 \notin \mathbb{N}$, but they are closed for integers $\mathbb{Z}$: for any two integers $m, n$, $m \pm n \in \mathbb{Z}$ always hold. While the arithmetic operations are not closed for integers $\mathbb{Z}$ because $1 \div 2 = \dfrac{1}{2} = 0.5 \notin \mathbb{Z}$, but they are closed for rationals $\mathbb{Q}$: for any two rationals $p, q$, the result of $p \pm q, pq, \dfrac{p}{q} (q \ne 0)$ are always in $\mathbb{Q}$.

The five arithmetic laws hold for fractions, we only verify the associative law for addition and the distribution law and leave the remaining three as \cref{qn:arithmatic-law-fraction}.
\begin{proof}
Associative law for fractions:
\begin{align*}
(\frac{b}{a} + \frac{d}{c}) + \frac{f}{e} & = \frac{bc + ad}{ac} + \frac{f}{e} = \frac{bce + ade + acf}{ace} && \\
  & = \frac{bce + (ade + acf)}{ace} & \text{associative law to nominators} & \\
  & = \frac{b\cancel{ce}}{a\cancel{ce}} + \frac{ade + acf}{ace} && \\
  & = \frac{b}{a} + (\frac{\cancel{a}d\cancel{e}}{\cancel{a}c\cancel{e}} + \frac{\cancel{ac}f}{\cancel{ac}e}) && \\
  & = \frac{b}{a} + (\frac{d}{c} + \frac{f}{e}) && \qedhere
\end{align*}
\end{proof}

\begin{proof}
We only show the left side distribution law, the right one rests on commutative law for multiplication.
\begin{align*}
  \frac{b}{a}(\frac{d}{c} + \frac{f}{e}) &= \frac{b}{a}\frac{de + cf}{ce} && \\
    &= \frac{b(de + cf)}{ace} = \frac{bde + bcf}{ace} & \text{distribution law to nominators} & \\
    &= \frac{bd\cancel{e}}{ac\cancel{e}} + \frac{b\cancel{c}f}{a\cancel{c}e} && \\
    &= \frac{b}{a}\frac{d}{c} + \frac{b}{a}\frac{f}{e} && \qedhere
\end{align*}
\end{proof}

\vspace{3mm}
\index{equal temperament}
Return to the legend of Pythagoras. After the discovery of the theory of music, that fractions ruled the harmonic sound, how to tune the seven strings of a lyre? If add an eighth string, then its length should be half of the first string, such that to get an octave. Pythagoras found overlapping a perfect fifth (ratio $3:2$) to a perfect four (ratio $4:3$) is an octave because $\dfrac{3}{2} \times \dfrac{4}{3} = 2$. He used the ratio $3:2$ of a perfect fifth as a basis scale, multiplied a geometric series:

\[
(\frac{3}{2})^{-1}, (\frac{3}{2})^0, (\frac{3}{2})^{1}, (\frac{3}{2})^2, (\frac{3}{2})^{3}, (\frac{3}{2})^4, (\frac{3}{2})^{5} = \frac{2}{3}, 1, \frac{3}{2}, \frac{9}{4}, \frac{27}{8}, \frac{81}{16}, \frac{243}{32}
\]

Where the last four were greater than 2, which exceeded an octave. Pythagoras divided them by either 2 or 4, which mapped them back to the octave, then reordered them as:

\[
1, \frac{9}{8}, \frac{81}{64}, \frac{4}{3}, \frac{3}{2}, \frac{27}{16}, \frac{243}{128}, \frac{2}{1}
\]

It gave the ratio of each tone to the base. To list the ratio between each adjacent tones, he calculate the quotient between every two numbers:

\[
\frac{9}{8}, \frac{9}{8}, \frac{256}{243}, \frac{9}{8}, \frac{9}{8}, \frac{9}{8}, \frac{256}{243}
\]

\index{tonality}
In music theory, the greater ratio $9:8$ is called a whole tone (step, W), the less ratio $256:243$ is called a semitone (half step, H). This list, W - W - H - W - W -W - H, is called a major scale. Musical works based on major scale typically create sense of warmth, cheerfulness, and resolution. Another type, known as minor scale, which shifts the list to W - H - W - W - H - W - W; it typically create sense of mellowness and introspection. Is there a harmonic 12-tone equal temperament? Unfortunately, it was out of what Pythagoras could design with fractions. To work out a 12-tone totality, we need setup 5 whole tones, 2 semitones, in total $5 \times 2 + 2 = 12$ ratios within an octave of $2:1$, which satisfy:

\[
1 = \alpha^0, \alpha^1, \alpha^2, \alpha^3, \alpha^4, \alpha^5, \alpha^6, \alpha^7, \alpha^8, \alpha^9, \alpha^{10}, \alpha^{11}, \alpha^{12} = 2
\]

It has a solution: $\alpha = \sqrt[12]{2}$, but it is not a fraction! While pursuing the sound of heaven, let's continue our tour of numbers.

\begin{Exercise}[label={ex:fractions}]
\Question{The term harmonic mean comes after Pythagoras's music theory. Pythagoras was trying to insert a string with length $h$ between two strings with lengths of $a$ and $b$, such that they produce harmonic sound. He found the reciprocal of $h$ is at the `middle' of the reciprocal of $a$ and the reciprocal of $b$, which is, \index{harmonic mean}
\[
\frac{1}{h} - \frac{1}{a} = \frac{1}{b} - \frac{1}{h}
\]

That is why the harmonic mean is also known as the mean of reciprocals. Verify that: 1) A boat travels between $A$ and $B$ with different speed: $v_1$ for go, and $v_2$ for back. The average speed $v$ is the harmonic mean of $v_1$ and $v_2$. 2) Connect two resisters $R_1$ and $R_2$ in parallel, it's equivalent to connect \emph{two} same resisters of $R$, where $R$ is the harmonic mean of $R_1$ and $R_2$.
}

\Question{*Verify $\dfrac{2}{p} = \dfrac{1}{a} + \dfrac{1}{b}$ equals $(2a - p)(2b - p) = p^2$, and use this fact to show: for any prime $p$ greater than 2, $\dfrac{2}{p}$ can be \underdot{unique} decomposed into two Egyptian fractions as $\dfrac{2}{p} = \dfrac{1}{a} + \dfrac{1}{b} (a \ne b)$.\label{qn:unique-decompose-r}}

\Question{Show that Erdős-Straus conjecture is true for all even number $n$.}

\Question{Explain why it's sufficient to show Erdős-Straus conjecture is true for all primes?}

\Question{Show that Erdős-Straus conjecture is true for all primes of $4k+3$. Hint: use the result of \cref{qn:unique-decompose-r}.}

\Question{*For the puzzle that three sons divide $n$ animals in the proportions of $\dfrac{1}{a}$, $\dfrac{1}{b}$, and $\dfrac{1}{c}$, where the eldest gets more than the second, the second gets more than the youngest, and they return the borrowed one at last, find all possible $n, a, b, c$, you may write a program to enumerate all solutions, but this exercise can be solved manually. Hint: the problem is to find all solutions in integers for the equation $\dfrac{n}{n+1} = \dfrac{1}{a} + \dfrac{1}{b} + \dfrac{1}{c}$, where $a < b < c$. Since every son gets at least one animal, what is the least $n$? How to show that $a$ can not exceed 3, hence can only be 2? \label{qn:three-sons}}

\Question{*By $0.999\cdots = 1$, any terminating decimal has two forms: $X.a_1a_2 \cdots a_m000\cdots$ and $X.a_1a_2 \cdots b999\cdots$, where $b = a_m - 1$, $X$ is the whole number part. Show that, except for this special case, the decimal of any value $x$ is unique, where $0 < x < 1$.}

\Question{Show that the fraction of any repeating decimal $0.\dot{a_1} \dot{a_2} \cdots \dot{a_n}$ is $\dfrac{a_1 a_2 \cdots a_n}{99 \cdots 9} = \dfrac{a_1 a_2 \cdots a_n}{10^{n+1} - 1}$. \label{qn:cyclic-decimal}}

\Question{Use fractions and Lemma \ref{th:cycle-of-9} to show: Atalanta first traverses $\dfrac{1}{2}$ distance, then traverse half of the remaining, which is $\dfrac{1}{4}$ distance, and repeats, she will walk to the end of the path. Hint: using sum of fractions and binary decimal.\label{qn:dichotomy-paradox}}

\Question{Verify the additive/multiplicative commutative laws, and multiplicative associative law for fractions.\label{qn:arithmatic-law-fraction}}
\end{Exercise}

\begin{Answer}[ref={ex:fractions}]
\Question{By the definition of harmonic mean:

\begin{align*}
\frac{1}{h} - \frac{1}{a} &= \frac{1}{b} - \frac{1}{h}    \\
\frac{2}{h} &= \frac{1}{a} + \frac{1}{b} \\
 h &= \frac{2}{\frac{1}{a} + \frac{1}{b}} = \frac{2ab}{a + b}
\end{align*}

Denote the distance by $s$, the times of go and return by $t_1 = \dfrac{s}{v_1}$ and $t_2 = \dfrac{s}{v_2}$ respectively, the average speed is:
\begin{align*}
  v & = \frac{2s}{t_1 + t_2} = \frac{2\cancel{s}}{\frac{\cancel{s}}{v_1} + \frac{\cancel{s}}{v_2}}
    = \frac{2}{\frac{1}{v_1} + \frac{1}{v_2}} = \frac{2v_1 v_2}{v_1 + v_2}
\end{align*}

It turns out to be the harmonic mean of $v_1$ and $v_2$. Alternatively, as the total time is the sum of the departure time and return time: $\dfrac{2s}{v} = \dfrac{s}{v_1} + \dfrac{s}{v_2}$, cancel $s$ on both sides: $\dfrac{2}{v} = \dfrac{1}{v_1} + \dfrac{1}{v_2}$, which gives the harmonic mean $\dfrac{2}{h} = \dfrac{1}{a} + \dfrac{1}{b}$ too.

\begin{center}
 \includegraphics[scale=0.4]{img/harmonic-meanr}
 \captionof{figure}{Resisters in parallel}
 \label{fig:harmonic-meanr}
\end{center}

As shown in \cref{fig:harmonic-meanr}, under the same voltage, the total currents of both sides, given their overall resisters are equivalent, are equal, that is:

\begin{align*}
\frac{I}{2} + \frac{I}{2} &= I_1 + I_2 \\
2\frac{\cancel{V}}{R} &= \frac{\cancel{V}}{R_1} + \frac{\cancel{V}}{R_2} \\
\frac{2}{R} &= \frac{1}{R_1} + \frac{1}{R_2} \\
R &= \frac{2}{\frac{1}{R_1} + \frac{1}{R_2}} = \frac{2R_1 R_2}{R_1 + R_2}
\end{align*}

Each resister on the right is the harmonic mean of $R_1$ and $R_2$.
}

\Question{Verify $\dfrac{2}{p} = \dfrac{1}{a} + \dfrac{1}{b}$ equals $(2a - p)(2b - p) = p^2$, and use this fact to show: for any prime $p$ greater than 2, $\dfrac{2}{p}$ can be \underdot{unique} decomposed into two Egyptian fractions as $\dfrac{2}{p} = \dfrac{1}{a} + \dfrac{1}{b} (a \ne b)$.

  \begin{proof}
  \begin{align*}
  \frac{2}{p} & = \frac{1}{a} + \frac{1}{b} = \frac{a+b}{ab} \\
  2ab & = pa + pb & p, a, b\text{ as denominators, } \ne 0 \\
  4ab - 2pa - 2pb & = 0 & \text{move and } \times 2 \\
  4ab - 2pa - 2pb + p^2 & = p^2 & \text{both sides } + p^2 \\
  (2a - p)(2b - p) & = p^2 & \text{factorize}
  \end{align*}
  Since $p$ is prime, $p^2$ can only have factors of 1, $p$, or $p^2$; by $a \ne b$, we can exclude $p$, hence $2a - p = 1$ and $2b - p = p^2$, hence,
  \[
  a = \frac{p+1}{2}, b = \frac{p(p+1)}{2}
  \]
  As $p$ is odd prime, $a$ is integral, thereby,
  \[
  \frac{2}{p} = \frac{1}{\frac{1}{2}(p + 1)} + \frac{1}{\frac{1}{2}p(p+1)} \qedhere
  \]
  \end{proof}
}

\Question{Show that Erdős-Straus conjecture is true for all even number $n$.

\vspace{2mm}
Let $n = 2m$, so that $\dfrac{4}{2m} = \dfrac{2}{m}$, which is \cref{th:decompose-of-2-n}.
}

\Question{Explain why it's sufficient to show Erdős-Straus conjecture is true for all primes?

\vspace{2mm}

Note, \[
\frac{4}{mp} = \frac{1}{ma} + \frac{1}{mb} + \frac{1}{mc}
\]
}

\Question{Show that Erdős-Straus conjecture is true for all primes of $4k+3$.
  \begin{proof}
    By \cref{qn:unique-decompose-r},
    \begin{align*}
      \frac{2}{p} &= \frac{1}{\frac{1}{2}(p + 1)} + \frac{1}{\frac{1}{2}p(p + 1)} && \\
      \frac{4}{p} &= \frac{2}{\frac{1}{2}(p + 1)} + \frac{2}{\frac{1}{2}p(p + 1)} & \text{both sides } \times 2 & \\
      \frac{4}{4k + 3} &= \frac{2}{\frac{1}{2}(4k + 4)} + \frac{2}{\frac{1}{2}(4k + 3)(4k + 4)} & \text{substitute } p = 4k + 3 & \\
                       &= \frac{1}{k + 1} + \frac{1}{(4k + 3)(k + 1)} & & \\
                       &= \frac{1}{k + 2} + \frac{1}{(k + 1)(k + 2)} + \frac{1}{(4k + 3)(k + 1)} & \text{by \cref{th:egyptian-fraction-split}} & \qedhere
    \end{align*}
  \end{proof}
}

\Question{Find all possible values of $n, a, b, c$, such that the three sons divide $n$ animals in the proportions of $\dfrac{1}{a}$, $\dfrac{1}{b}$, and $\dfrac{1}{c}$, where the eldest gets more than the second, the second gets more than the youngest, and they return the borrowed one at last.

\vspace{2mm}
We give a pure mathematical solution, followed with a `brute force' program for comparison. The problem amounts to finding all natural numbers satisfying the equation:

\[
\frac{n}{n+1} = \frac{1}{a} + \frac{1}{b} + \frac{1}{c}, a < b < c
\]

With the one borrowed animal, the sons were able to divide $n+1$ animals into three parts of $\dfrac{n + 1}{a}$, $\dfrac{n + 1}{b}$, and $\dfrac{n + 1}{c}$, whose sum is $n$, such that they could return the last animal. Let $n' = n + 1$, be the number including the borrowed animal, we write above equation in $n'$ as:

\be
\frac{n'-1}{n'} = \frac{1}{a} + \frac{1}{b} + \frac{1}{c}
\label{eq:equation-of-inherit}
\ee

We need decrease $n'$ by 1 to give $n$ finally. With the one borrowed animal, $a, b, c$ all divide $n'$. We shall first show: the eldest son could get only $\dfrac{1}{2}$ animals, hence, $a$ can only be 2.

\begin{proof}
Assume $a \geq 3$, since $a < b < c$,

\[
\frac{1}{a} + \frac{1}{b} + \frac{1}{c} \leq \frac{1}{3} + \frac{1}{4} + \frac{1}{5} = \frac{47}{60}
\]

The youngest son got at least one animal; what the second son got were more than this, which were at least two; and the eldest got even more, which were at least three; plus the borrowed one animal, the sum $n' > 3 + 2 + 1 = 6$, hence,

\[
\frac{n'-1}{n'} = 1 - \frac{1}{n'} > 1 - \frac{1}{6} = \frac{5}{6} = \frac{50}{60} > \frac{47}{60} \geq \frac{1}{a} + \frac{1}{b} + \frac{1}{c}
\]

It turns out when $a \geq 3$, the equation \cref{eq:equation-of-inherit} can't hold, therefore, $a$ can only be 2.
\end{proof}

Thus the problem is simplified to find all natural numbers satisfying a new equation of

\[
\frac{n'-1}{n'} = \frac{1}{2} + \frac{1}{b} + \frac{1}{c}
\]

We next show that $b$ can only be 3 or 4 to simplify problem further:

\begin{proof}
Assume $b \geq 5$, since $b < c$,

\[
\frac{1}{a} + \frac{1}{b} + \frac{1}{c} \leq \frac{1}{2} + \frac{1}{5} + \frac{1}{6} = \frac{13}{15} \approx 0.87
\]

As $a, b, c$ all divide $n'$ and $a = 2$, $n'$ must be even. We deduced $n' > 6$ as above, the first even number being greater 6 is 8, hence,

\[
\frac{n'-1}{n'} = 1 - \frac{1}{n'} \geq 1 - \frac{1}{8} = \frac{7}{8} = 0.875 > 0.87 \geq \frac{1}{a} + \frac{1}{b} + \frac{1}{c}
\]

It turns out when $b \geq 5$, equation \cref{eq:equation-of-inherit} can't hold, therefore, $b$ can be only 3 or 4.
\end{proof}

\begin{enumerate}[{Case}1)]
\item $a = 2, b = 3$
  \begin{align*}
    \frac{n'-1}{n'} &= \frac{1}{2} + \frac{1}{3} + \frac{1}{c} \\
    1 - \frac{1}{n'} &= \frac{5}{6} + \frac{1}{c} \\
    \frac{1}{6} &= \frac{1}{n'} + \frac{1}{c}
  \end{align*}
  As $a = 2, b = 3$ both divide $n'$, 6 divides $n'$. Let $n' = 6k$ for some integer $k > 1$ (as deduced above, $n' \geq 8$), substitute into above equation and simplify:
  \[
  k = \frac{c}{c - 6} \geq 2
  \]
  Solving the inequality $\dfrac{c}{c - 6} \geq 2$ gives $c \leq 12$. As the denominator $c-6$ is positive, hence $c > 6$. Such valid $c$ can only be $7, 8, 9, \cancel{10}, \cancel{11}, 12$, where 10 and 11 are ruled out because otherwise $k = \dfrac{c}{c - 6}$ isn't integral. We obtain four solutions:
  \begin{align*}
  a &= 2, b = 3, c = 7, n' = 42 & 42\frac{1}{2} + 42\frac{1}{3} + 42\frac{1}{7} &= 21 + 14 + 6 = 41 \\
  a &= 2, b = 3, c = 8, n' = 24 & 24\frac{1}{2} + 24\frac{1}{3} + 24\frac{1}{8} &= 12 + 8 + 3 = 23 \\
  a &= 2, b = 3, c = 9, n' = 18 & 18\frac{1}{2} + 18\frac{1}{3} + 18\frac{1}{9} &= 9 + 6 + 2 = 17 \\
  a &= 2, b = 3, c = 12, n' = 12 & 12\frac{1}{2} + 12\frac{1}{3} + 12\frac{1}{12} &= 6 + 4 + 1 = 11
  \end{align*}
\item $a = 2, b = 4$
  \begin{align*}
    \frac{n'-1}{n'} &= \frac{1}{2} + \frac{1}{4} + \frac{1}{c} \\
    1 - \frac{1}{n'} &= \frac{3}{4} + \frac{1}{c} \\
    \frac{1}{4} &= \frac{1}{n'} + \frac{1}{c}
  \end{align*}
  As $a = 2, b = 4$ both divide $n'$, 4 divides $n'$. Let $n' = 4k$ for some integer $k \geq 2 $ (also $n' \geq 8$), substitute into above equation and simplify:
  \[
  k = \frac{c}{c - 4} \geq 2
  \]
  Solving the inequality $\dfrac{c}{c - 2} \geq 2$ gives $c \leq 8$. As the denominator $c-4$ is positive, hence $c > 4$. Such valid $c$ can only be $5, 6, \cancel{7}, 8$, where 7 is ruled out because otherwise $k = \dfrac{c}{c - 4}$ isn't integral. We obtain another three solutions:
  \begin{align*}
  a &= 2, b = 4, c = 5, n' = 20 & 20\frac{1}{2} + 20\frac{1}{4} + 20\frac{1}{5} &= 10 + 5 + 4 = 19 \\
  a &= 2, b = 4, c = 6, n' = 12 & 12\frac{1}{2} + 12\frac{1}{4} + 12\frac{1}{6} &= 6 + 3 + 2 = 11 \\
  a &= 2, b = 4, c = 8, n' = 8 & 8\frac{1}{2} + 8\frac{1}{4} + 8\frac{1}{8} &= 4 + 2 + 1 = 7
  \end{align*}
\end{enumerate}
Note the solution of 11 + 1 animals, although appears in both cases, the distribution differs. There are 7 solutions in total.

\vspace{3mm}
We write a program to enumerate all possible $a, b, c$, test whether the sum of their reciprocals being the form of $\dfrac{n + 1}{n}$. We enumerate $a$ from 2, 3, 4, ... As $a < b < c$, we enumerate $b$ from $a + 1$ and enumerate $c$ from $b + 1$. We need estimate the upper limits of $a, b, c$ such that the program terminates. Since $a < b < c$, the upper limit of $c$ is also the upper limit of $a$ and $b$. The youngest son got at least 1 animal, hence, $c \leq n'$, we deduce from \cref{eq:equation-of-inherit}:

\begin{align*}
  1 - \frac{1}{n'} &= \frac{1}{a} + \frac{1}{b} + \frac{1}{c} \\
  1 - \frac{1}{n'} - \frac{1}{c} &= \frac{1}{a} + \frac{1}{b} \leq \frac{1}{2} + \frac{1}{3} = \frac{5}{6} \\
  \frac{1}{6} &\leq \frac{1}{n'} + \frac{1}{c} \leq \frac{1}{c} + \frac{1}{c} & \text{The youngest son got at least 1, } \frac{1}{n'} \leq \frac{1}{c} \\
  \frac{1}{6} &\leq \frac{2}{c} \Rightarrow c \leq 12
\end{align*}

The upper limit is 12. Besides, from $1 - \dfrac{1}{n'} = \dfrac{1}{a} + \dfrac{1}{b} + \dfrac{1}{c}$ we deduce $n' = \dfrac{abc}{abc - ab - bc - ac}$, if the denominator is not 0 and $a, b, c$ all divide $n'$, then we find a solution. \Cref{fig:inherit-program} gives an example program.

\begin{center}
 \includegraphics[scale=0.4]{img/inherit-program}
 \captionof{figure}{Program of three sons dividing animals in Scratch}
 \label{fig:inherit-program}
\end{center}

Below are example programs in Python and Haskell.

\begin{lstlisting}[language = Python, frame = single]
s = []
for a in range(2, 11):
    for b in range(a + 1, 12):
        for c in range(b + 1, 13):
            d = a*b*c - a*b - b*c - a*c
            if d > 0:
                n = a*b*c // d
                if n % a == 0 and n % d == 0 and n % c == 0:
                    s.append((a, b, c))
print(s)
\end{lstlisting}

\begin{Haskell}[frame = single]
[(a, b, c, n) | a <- [2..10], b <- [a+1..11], c <- [b+1..12],
     let d = a*b*c - a*b - b*c - a*c, d > 0,
     let n = a*b*c `div` d, n > 1,
         n `mod` a == 0, n `mod` b == 0, n `mod` c == 0]
\end{Haskell}
}

\Question{Show that the decimal of any $x$ is unique, where $0 < x < 1$.

  \begin{proof}
    Assume for contradiction that
\be
\frac{a_1}{10} + \frac{a_2}{100} + \cdots = \frac{b_1}{10} + \frac{b_2}{100} + \cdots
\label{eq:decimal-a-eq-b}
\ee
Where $a_n$ and $b_n$ are the first pair of different digits, $|a_n - b_n| \geq 1$:
\begin{align*}
\frac{a_1}{10} + \frac{a_2}{100} + \cdots - (\frac{b_1}{10} + \frac{b_2}{100} + \cdots) & \geq \frac{1}{10^n} - (\frac{|a_{n+1} - b_{n+1}|}{10^{n+1}} + \frac{|a_{n+2} - b_{n+2}|}{10^{n+2}} + \cdots) \\
  & \geq \frac{1}{10^n} - (\frac{9}{10^{n+1}} + \frac{9}{10^{n+2}} + \cdots) \\
  & = \frac{1}{10^n} - \frac{1}{10^n}(0.99\cdots) = 0
\end{align*}

contradict with \cref{eq:decimal-a-eq-b}.
  \end{proof}
}

\Question{Show that the fraction of any repeating decimal $0.\dot{a_1} \dot{a_2} \cdots \dot{a_n}$ is $\dfrac{a_1 a_2 \cdots a_n}{99 \cdots 9} = \dfrac{a_1 a_2 \cdots a_n}{10^{n+1} - 1}$.

  \begin{proof}
    \begin{align*}
    \text{设} x &= 0.\dot{a_1} \dot{a_2} \cdots \dot{a_n} \\
    10^n x & = a_1a_2\cdots a_n . \dot{a_1} \dot{a_2} \cdots \dot{a_n} \\
    10^n x - x &= a_1a_2\cdots a_n \\
    x &= \frac{a_1a_2\cdots a_n}{10^n - 1} = \frac{a_1a_2\cdots a_n}{99 \cdots 9} && \qedhere
    \end{align*}
  \end{proof}
}

\Question{Use sum of fractions to show: Atalanta first traverses $\dfrac{1}{2}$ distance, then traverse half of the remaining, which is $\dfrac{1}{4}$ distance, and repeats, she will walk to the end of the path.

\vspace{2mm}
Analogue to decimal $0.99\cdots = 1$, in binary $0.11\cdots = 1$, because:
\begin{proof}
  \begin{align*}
    x &= 0.11\cdots & 2x &= 1.11\cdots \\
    x & = 2x - x = 1 & & \qedhere
  \end{align*}
\end{proof}
Atalanta first traverses $\dfrac{1}{2}$, which is 0.1 in binary; next traverses $\dfrac{1}{4}$, which is 0.01 in binary, ... the total traversed distance is the sum of:
\[
0.1 + 0.01 + 0.001 + \cdots = 0.11\cdots = 1
\]
}

\Question{Verify the additive/multiplicative commutative laws, and multiplicative associative law for fractions.

\vspace{2mm}

Commutative law for addition:
\begin{align*}
  \frac{b}{a} + \frac{d}{c} &= \frac{bc + ad}{ac} \\
  &= \frac{ad + bc}{ac} & \text{apply integral commutative law to nominator} \\
  &= \frac{ad}{ac} + \frac{bc}{ac} = \frac{d}{c} + \frac{b}{a}
\end{align*}

Commutative law for multiplication:
\[
\frac{b}{a} \times \frac{d}{c} = \frac{bd}{ac} = \frac{db}{ca} = \frac{d}{c} \times \frac{b}{a}
\]

Associative law for multiplication:
\[
\frac{b}{a} \times \frac{d}{c} \times \frac{f}{e} = \frac{bdf}{ace} = \frac{b(df)}{a(ce)} = \frac{b}{a}\frac{df}{cd} = \frac{b}{a} \times (\frac{d}{c} \times \frac{f}{e})
\]
}
\end{Answer}

\ifx\wholebook\relax \else
\section{Answer}
\shipoutAnswer

\section{Algorithm for best Egyptian decomposition}
\subimport{inc/}{decomposition-en}

\begin{thebibliography}{99}
\subimport{inc/}{bib-en}
\end{thebibliography}

\expandafter\enddocument
\fi
