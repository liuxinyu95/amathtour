\ifx\wholebook\relax \else

\documentclass[b5paper]{article}
\usepackage[nomarginpar
  %, margin=.5in
]{geometry}

\addtolength{\oddsidemargin}{-0.05in}
\addtolength{\evensidemargin}{-0.05in}
\addtolength{\textwidth}{0.1in}

\usepackage[en]{../prelude}

\setcounter{page}{1}

\begin{document}

\title{Fraction}

\author{Xinyu LIU
\thanks{{\bfseries Xinyu LIU} \newline
  Email: liuxinyu99@hotmail.com \newline}
  }

\maketitle
\fi

\markboth{Fraction}{A tour of numbers}

\ifx\wholebook\relax
\chapter{Fraction}
\fi

\epigraph{Learn to be silent. Let your quiet mind listen and absorb.}{Pythagoras}

\index{gravitational wave}
The Laser Interferometer Gravitational-Wave Observatory (LIGO), an astronomical observatory, in USA detected the first gravitational wave signal on Sep 14th, 2015. The signal was named after the date as GW15094. It comes from a spectacular cosmic event happened 1.3 billion light-year away: a pair of black holes spiraled and merged into each other (\cref{fig:gravitational-wave}). The black holes were 36 and 29 times the mass of the Sun and formed a new black hole 62 times the mass of the Sun. The `ripples', namely the gravitational wave, in spacetime caused by their enormous gravitational pull travelled 1.3 billion light-years across the universe to reach Earth -- gravitational waves propagate at the speed of light according to Einstein's Theory of Relativity. Physicists in LIGO spent several months to analyze the data; they verified and finally announced on Feb 11th, 2016: this was the first detection of gravitational waves, confirming the prediction Einstein made a century earlier. Through the sound waves converted from the signal by physicists, people heard the sound of deep universe for the first time.

\begin{figure}[htbp]
 \centering
 \includegraphics[scale=0.35]{img/gravitational-wave}
 \caption{Depiction of black holes merging and gravitational waves.}
 \label{fig:gravitational-wave}
\end{figure}

\index{octave}
It was the study of the sound of universe 2500 years ago, Pythagoras discovered the theory of music. A legend said when Pythagoras passed a blacksmith's shop, he noticed various sounds produced by hammering on anvil. Georg Friedrich Händel, a music in Baroque period, composed a piano suite named `\emph{Air and Five Variations on The Harmonious Blacksmith}' (HWV430) in 1720. It was said Pythagoras observed hammers of different sizes produced consonant sounds; some were harmonic, others were not. By investigation and experiment, Pythagoras found the harmonic sounds were produced by the hammers of numerical weights of 12, 9, 8, and 6. This was the beginning of music theory. Basically, Pythagoras identified below ratios:

\begin{itemize}
\item Ratio $12:6$ (reduced to $2:1$) corresponds to an octave;
\item Ratio $9:6$ (reduced to $3:2$) corresponds to the perfect fifth;
\item Ratio of $12:9$(reduced to $4:3$) corresponds to the perfect fourth;
\item Ratio of $9:8$ corresponds to the perfect second.
\end{itemize}

This legend is well known, as shown in \cref{fig:pythagoras-music}. The woodcut showed Pythagoras was listening to the blacksmiths hammering an anvil; then he went back to do \emph{quantitative} investigation, he experimented with bells of different sizes, glasses holding different volume of waters, strings hanging with different weights, and pipes of different length. However, there was something wrong with this legend; at least the second and the third pictures of the woodcut conflict. It's easily to verify the experiment in the third picture: using different spoons to tap the same kind of glasses with some water, one will realize the tune is determined by the water in the glass, but not by the size of spoon. In the same way, the tune from the blacksmith's shop is determined by the anvil, but not the hammer. However, there was only an anvil in the first picture, which could only produce the same tune. Besides, the numbers 4, 6, 8, 9, 12, and 16 annotated to the hammers, the bells, the glasses, the hanging weights of the strings, and the pipes, are Hindu-Arabic numerals (see \cref{sec:hindu-arabic-numerals}), they hadn't been introduced to the west until to the 13th to 14th century. Pythagoras, in ancient Greece, mustn't use such numerals. Nevertheless, the woodcut depicts a good story about Pythagoras, who was curious about life and nature, engaged in hands-on practice and pioneered quantitative research.

\begin{figure}[htbp]
 \centering
 \includegraphics[scale=0.1]{img/pythagoras-music}
 \caption{Pythagoras and music. Woodcut from \emph{Theory of music} by Franchino Gaffurio, 1492 (1480?).}
 %% Woodcut showing Pythagoras with bells, a kind of glass harmonica, a monochord and (organ?) pipes in Pythagorean tuning. From Theorica musicae by Franchino Gaffurio, 1492 (1480?)
 %% https://commons.wikimedia.org/wiki/File:Gaffurio_Pythagoras.png
 \label{fig:pythagoras-music}
\end{figure}

\index{lyre}

Lyre was a kind of popular instrument with seven strings in ancient Greek, as shown in \cref{fig:lyre}; it is often used as the symbol of music. Pythagoras might obtain an octave from half a string, that is of the $\dfrac{1}{2}$ length, obtain the perfect fifth from $\dfrac{2}{3}$ length of a string, and the perfect fourth from $\dfrac{3}{4}$ length of a string. \Cref{fig:octave} shows the tunes.

\begin{figure}[htbp]
 \centering
 \subcaptionbox{A kylix depicting the god Apollo holding a lyre which was made from the shell of a tortoise. 480 - 470 BCE, Delphi Archaeological Museum. \label{fig:lyre}}{\includegraphics[scale=0.4]{img/apollo-with-lyre}}
 \subcaptionbox{Icon of lyre.}{\includegraphics[scale=0.35]{img/lyre-icon}}
 %% https://www.worldhistory.org/image/986/apollo-with-lyre/
\end{figure}

Pythagoras established his music theory that greatly influenced the west based on mathematics, specifically based on \emph{fractions and ratio}. He thought the universe was a giant lyre, with the known seven celestial bodies (including Sun, Moon, the five planets of Mercury, Venus, Mars, Jupiter, and Uranus) as strings, vibrating at different tunes. In this way, Pythagoras believed the sound of universe was numbers.

\begin{figure}[htbp]
 \centering
 \includegraphics[scale=0.4]{img/octave}
 \caption{The next C spans an octave, where the ratio of the lengths of the string is $2:1$; G spans a perfect fifth from C, where the ratio of the lengths of the string is $3:2$; F spans a perfect fourth from C, where the ratio of the length of the string is $4:3$.}
 \label{fig:octave}
\end{figure}

The history of fraction is longer than the history of zero or negative numbers. In \emph{The Zuo Tradition (Zuozhuan): Commentary on the ‘Spring and Autumn Annals’}, a book about Chinese history from 770 - 476 BCE, the author documented that the capital city of a state should not be larger than the imperial city. People categorized capital cities into three categories: large, medium, and small. For the large category, the area should be less than one third of the imperial city; for the medium, the area should be less than one fifth; for the small, the area should be less than one ninth. However, these terms, such as one third, might not be really fractions, as people hadn't used them in arithmetic calculation yet. Among the \emph{Chu bamboo slips} (dating to the middle to late Waring State period, no earlier than 305 BCE) collected by Tsinghua University, there is a `calculation table', which was used by ancient Chinese to multiply numbers up to 99. In this table, there are dedicated rows and columns for $\times \dfrac{1}{2}$, which list the multiplication result\citepage[p20-22]{CaoNa-2017} of $\dfrac{1}{4}, \dfrac{1}{2}, 1\dfrac{1}{2}, 2\dfrac{1}{2}, \dotsc, 4\dfrac{1}{2}$. Chinese developed complete arithmetic rules for fractions by Han Dynasty (\cref{sec:chinese-fractions}). Fractions appeared earliest in ancient Egypt.

\section{Egyptian fractions}
\index{Papyrus}

We know about Egyptian fractions mainly from two historic documents: Rhind papyrus (\cref{fig:rhind-papyrus}) and Moscow papyrus (\cref{fig:moscow-papyrus}). Papyrus was the writing material in ancient times, which was widely used in ancient Egypt. \emph{Cyperus papyrus} (family Cyperaceae), also called paper plant, is the plant long cultivated in the Nile delta region in Egypt. People collected its stalk or stem, cut the central pith into thin strips, pressed together, and dried to form a smooth thin writing surface. The Greeks, and later Romans also adopted papyrus for writing. Papyrus had been replaced by other less-expensive materials by 3rd century CE in Europe; people stopped using it around 12th century. The two treasured documents in papyrus record the mathematical achievement in ancient Egypt. One was bought in 1858 in a Nile resort town by a Scottish antiquary, Alexander Henry Rhind, hence its name; it is collected in British Museum. It's a mathematical textbook around 1650 BCE; the author was Ahmes (hence is also called the Ahmes papyrus), including 85 math problems and their solution. The other one, Moscow papyrus, also known as the Golenischev Papyrus, was discovered and acquired by Vladimir Golenischev, a Russian Egyptologist, in 1893. It is now housed in the Pushkin State Museum of Fine Arts in Moscow, which gives the artifact its name. It was the work by some unknown author around 1890 BCE including 25 math problems.

\begin{figure}[htbp]
 \centering
 \subcaptionbox{Part of Rhind papyrus. \label{fig:rhind-papyrus}}{\includegraphics[scale=0.2]{img/rhind-papyrus-part}} \quad
 \subcaptionbox{Part of Moscow papyrus. \label{fig:moscow-papyrus}}{ \includegraphics[scale=0.4]{img/moscow-papyrus}}
 %% https://www.britishmuseum.org/collection/image/366139001
 %% https://old.maa.org/press/periodicals/convergence/mathematical-treasure-the-rhind-and-moscow-mathematical-papyri
 %% https://mathshistory.st-andrews.ac.uk/HistTopics/Egyptian_papyri/
\end{figure}

\index{Egyptian fractions}
The two artifacts evidently show that the Egyptian could solve problem with fractions. However, the nominator of all their factions were 1. This feature is so unique, that people call such fractions \emph{Egyptian fractions} or unit fractions. In Hieroglyphic (see \cref{sec:rosetta-stone}), the Egyptian draw an `eye' over number $a$ as the notation for $\dfrac{1}{a}$. \Cref{fig:egyptian-fractions} shows $\dfrac{1}{5}$ and $\dfrac{1}{15}$ respectively in Hieroglyphic. The meaning of the unit fraction is obvious, for example, $\dfrac{1}{3}$ means one out of the three equally divided parts (one third); $\dfrac{1}{5}$ means one out of the five divided parts (one fifth); in general, $\dfrac{1}{a}$ is one out of the $a$ equally divided parts.

\begin{figure}[htbp]
 \centering
 \includegraphics[scale=0.3]{img/egyptian-fractions}
 \caption{Egyptian fractions: $\dfrac{1}{5}$ and $\dfrac{1}{15}$}
 \label{fig:egyptian-fractions}
\end{figure}

The Egyptian didn't use fractions like $\dfrac{3}{4}$ or $\dfrac{2}{7}$ as their nominators were not 1. Instead, they decompose them as the sum of unit fractions, for example: $\dfrac{3}{4} = \dfrac{1}{2} + \dfrac{1}{4}$, written as the left part of \cref{fig:sum-egyptian-fractions}. The symbol in the middle likes two walking legs. If the walking direction is as same as the writing direction, it means addition; if reversed, it means subtraction. The right part of \cref{fig:sum-egyptian-fractions} is another decomposition of $\dfrac{2}{5} (= \dfrac{1}{3} + \dfrac{1}{15})$. When decompose, the Egyptian only allowed distinct unit fractions, hence $\dfrac{2}{7}$ should not be decomposed to $\dfrac{1}{7} + \dfrac{1}{7}$, but $\dfrac{1}{4} + \dfrac{1}{28}$. It impossible to decompose into only two unit fractions sometimes, for example, in Rhind papyrus, $\dfrac{2}{29}$ was decomposed to $\dfrac{2}{29} = \dfrac{1}{24} + \dfrac{1}{58} + \dfrac{1}{174} + \dfrac{1}{232}$. The decomposition may not be unique, for example, $\dfrac{2}{29} = \dfrac{1}{15} + \dfrac{1}{435} = \dfrac{1}{16} + \dfrac{1}{232} + \dfrac{1}{464}$. Basically, there is no obvious pattern or easy rules to find a composition; it's hard for people to master Egyptian fractions. In order to support calculation with fractions, the Egyptian had to make big decomposition table for reference.

\begin{figure}[htbp]
 \centering
 \includegraphics[scale=0.3]{img/sum-egyptian-fractions}
 \caption{Left: $\dfrac{1}{2} + \dfrac{1}{4}$ \qquad Right: $\dfrac{1}{3} + \dfrac{1}{15}$}
 \label{fig:sum-egyptian-fractions}
\end{figure}


We haven't found any direct documents giving clues about why the Egyptian developed their system of fractions in this way as the sums of distinct unit fractions. One legend said: a farmer, in his will, wanted to divide his 17 horses among his three sons in the following proportions, $\dfrac{1}{2}$ to the eldest, $\dfrac{1}{3}$ to the second, and $\dfrac{1}{9}$ to the youngest. When their father died, they were not able to divide the horses as it resulted in fractional horse, nobody wanted to kill any horses. They sought the counsel of a wise old man in the village. The old man lent them an additional horse for dividing; with $17 + 1 = 19$ horses, the eldest son got $18 \times \dfrac{1}{2} = 9$ horses; the second son got $18 \times \dfrac{1}{3} = 6$ horses; the youngest son get $18 \times \dfrac{1}{9} = 2$; there remained $18 - 9 - 6 - 2 = 1$ horse, and they returned it to the old man. This story can be interpreted in Egyptian fractions:

\[
\frac{17}{18} = \frac{1}{2} + \frac{1}{3} + \frac{1}{9}
\]

%% https://proofwiki.org/wiki/Henry_Ernest_Dudeney/Modern_Puzzles/89_-_The_Seventeen_Horses
Although very popular, this legend can \emph{not} be the origin of Egyptian fractions. We've found similar stories in many peoples, including the Arabic, the Indian, the Jewish, the Chinese, and so on. Henry Dudeney collected this puzzle in his popular book \emph{Modern Puzzles and how to solve them} in 1926. In different versions, the animals vary from horses, to camels, elephants; the number and dividing portions vary too, for example, the three sons need to divide 11 animals into portions of $\dfrac{1}{2}$, $\dfrac{1}{4}$, and $\dfrac{1}{6}$, which corresponds to:

\[
\frac{11}{12} = \frac{1}{2} + \frac{1}{4} + \frac{1}{6}
\]

\Cref{qn:three-sons} asks to give all possible numbers and dividing portions. The earliest such story was from an Iran philosopher, Mulla Muhammad Mahdi Naraqi's work in the 18th century. Though there were inheritance dividing problems in Rhind papyrus, they were about evenly dividing. For example, problem 63\citepage[p15]{MKlein-1972} was about to divide 700 breads to four persons in the following the proportions: $\dfrac{2}{3}$ to the first, $\dfrac{1}{2}$ to the second, $\dfrac{1}{3}$ to the third, and $\dfrac{1}{4}$ to the fourth\footnote{$\dfrac{2}{3}$ was few of the special values needn't be decomposed to unit fractions in Egypt.}. Ahmes solved it in the way that likes equation:

\[
\frac{2}{3}x + \frac{1}{2}x + \frac{1}{3}x + \frac{1}{4}x = 700
\]

He summed $\dfrac{2}{3}$, $\dfrac{1}{2}$, $\dfrac{1}{3}$, and $\dfrac{1}{4}$ to $1\dfrac{3}{4} (= 1 + \dfrac{1}{2} + \dfrac{1}{4})$, then divided 1 by $1\dfrac{3}{4}$ to obtain $\dfrac{4}{7} = (\dfrac{1}{2} + \dfrac{1}{14})$, finally, $700 \times \dfrac{4}{7} = 400$, which was the answer $x = 400$. This is exactly how students solve an equation of one unknown in primary school. Note that the third and fourth person get breads in fraction.

In some opinion, Egyptian fractions divide things more economically. Considering to divide 5 breads to 8 persons equally, the regular way is to divide each bread into 8 pieces and everyone takes 5 pieces as shown in \cref{fig:evenly-devide}:

\begin{figure}[htbp]
 \centering
 \includegraphics[scale=0.3]{img/evenly-divide}
 \caption{Cut equally into 40 pieces, each one get 5.}
 \label{fig:evenly-devide}
\end{figure}

The Egyptian might find a better way than this as shown in \cref{fig:egyptian-devide}, where everyone get a half and $\dfrac{1}{8}$ bread.

\begin{figure}[htbp]
 \centering
 \includegraphics[scale=0.3]{img/egyptian-divide}
 \caption{Cut into 16 pieces, everyone gets two (a big one of $\dfrac{1}{2}$ and a small one of $\dfrac{1}{8}$).}
 \label{fig:egyptian-devide}
\end{figure}

Do you have any idea why the Egyptian developed such special fractions? On one hand, it's so complex that finally disappeared in history, on the other hand, it brings interesting problems with hindsight: mathematicians thought about: (1) Can every fraction be decomposed into Egyptian fractions?  (2) if can, what is the best decomposition? The first one was solved by Fibonacci in 1202 in \emph{Liber Abaci}. The second one led to a conjecture hasn't been solved till now. Let us start with two easier problems:

\index{Egyptian fraction!decomposition}
\begin{proposition} Any Egyptian fraction can be decomposed into two distinct Egyptian fractions, i.e., $\dfrac{1}{n} = \dfrac{1}{a} + \dfrac{1}{b}$. \label{th:egyptian-fraction-split}
\end{proposition}

\begin{proof}
  \begin{align*}
    \frac{1}{n} &= \frac{n+1}{n(n+1)} & \text{both nominator and denominator}\times (n + 1) & \\
                &= \frac{\cancel{n}}{\cancel{n}(n+1)} + \frac{1}{n(n+1)} && \\
                &= \frac{1}{n+1} + \frac{1}{n(n+1)} && \qedhere
  \end{align*}
\end{proof}

It follows that:

\begin{proposition}
Any $\dfrac{2}{n}$ can be decomposed into Egyptian fractions. \label{th:decompose-of-2-n}
\end{proposition}

\begin{proof}
  \begin{align*}
    \frac{2}{n} &= \frac{1}{n} + \frac{1}{n} && \\
                &= \frac{1}{n} + \frac{1}{n+1} + \frac{1}{n(n+1)} & \text{by proposition \ref{th:egyptian-fraction-split}} & \qedhere
  \end{align*}
\end{proof}

We can obtain a better result with $n$ is odd: $\dfrac{2}{n}$ can be decomposed into two Egyptian fractions.

\begin{proof}
  \begin{align*}
    \frac{1}{n} &= \frac{1}{n+1} + \frac{1}{n(n+1)} & \text{by proposition \ref{th:egyptian-fraction-split}} & \\
    \frac{2}{n} &= \frac{2}{n+1} + \frac{2}{n(n+1)} & \text{both sides} \times 2 & \\
                &= \frac{1}{\frac{1}{2}(n+1)} + \frac{1}{n\frac{1}{2}(n+1)} & n + 1\text{ is even; nominator and denominator} \div 2 &\qedhere
  \end{align*}
\end{proof}

Further, \cref{qn:unique-decompose-r} asks to show the decomposition is unique when $n$ is odd prime. As the consequence of this fact, there was a table in Rhind papyrus, which list all decomposition from $\dfrac{2}{5}$ to $\dfrac{2}{101}$. Back to the theorem proven by Fibonacci, for any irreducible fraction\footnote{A fraction is irreducible if it can't be reduced; it means there is no common divisor of the nominator and denominator except 1.}, as one may convert an improper fraction\footnote{A fraction is improper if the nominator is greater than its denominator.} to a mixed number, it's sufficient to show:

\index{Fibonacci's theorem}
\begin{theorem}[Fibonacci]
Any irreducible proper fraction $\dfrac{b}{a}$ can be decomposed to Egyptian fractions.
\end{theorem}

\begin{proof}
Fibonacci applied the division with remainders: $a = bq + r$, where $q$ is the quotient\footnote{The term quotient derives from the Latin word \emph{quotiēns}, which is an adverb meaning ``how many times'' or ``how often''.}, $r$ is the remainder and $0 < r < b$. The fraction that is most close to and less than $\dfrac{b}{a}$ is $\dfrac{1}{q + 1}$, and their difference is
\[
  \frac{b}{a} = \frac{1}{q + 1} + x
\]

With the idea of divide and conquer, Fibonacci converted the original problem to a sub-problem of decomposing $x$ into Egyptian fractions.

\begin{align*}
  x & = \frac{b}{a} - \frac{1}{q + 1} = \frac{bq + b - a}{a(q + 1)} & \text{common denominator} \\
    & = \frac{b - (a - bq)}{a(q+1)} = \frac{b - r}{a(q+1)} & \text{remainder } r = a - bq
\end{align*}

The nominator of $x$ is $b' = b - r$. As the remainder $0 < r < b$, the nominator $b' < b$, and it reduces,comparing to the nominator $b$ of the original fraction $\dfrac{b}{a}$, at least 1. If $b' = 1$, we are done; otherwise we need decompose $x$, a fraction of a smaller nominator. In this way, the nominator keeps decreasing as a sequence of $b > b' > b'' > \cdots$. For any given integer $b$, it can't infinitely decrease, but reach 1 finally, and the decomposition terminates. It proves any irreducible proper fraction can be decomposed to Egyptian fractions.
\end{proof}

For example, to decompose $\dfrac{5}{11}$ in Fibonacci's method: $11 = 2 \times 5 + 1$, the quotient $q = 2$, the remainder $r = 1$.

\begin{align*}
\frac{5}{11} & = \frac{1}{q + 1} + x = \frac{1}{3} + x  \\
 x &= \frac{5}{11} - \frac{1}{3} = \frac{4}{33} \\
\frac{5}{11} &= \frac{1}{3} + \frac{4}{33}
\end{align*}

We next decompose $\dfrac{4}{33}$. Apply the division with remainder again: $33 = 4 \times 8 + 1$, the quotient $q = 8$, the remainder $r = 1$.

\begin{align*}
\frac{4}{33} & = \frac{1}{q + 1} + x = \frac{1}{9} + x  \\
 x &= \frac{4}{33} - \frac{1}{9} = \frac{12 - 11}{99} = \frac{1}{99}
\end{align*}

We are done, the decomposition is
\[
\frac{5}{11} = \frac{1}{3} + \frac{1}{9} + \frac{1}{99}
\]

\index{greedy method}
Fibonacci always chose the unit fraction that was the \emph{closest} to $\dfrac{b}{a}$. Such strategy is called the \emph{greedy} method. However, it may not necessary be the best decomposition. For example,

\[
\frac{5}{121} = \frac{1}{25} + \frac{1}{757} + \frac{1}{763309} + \frac{1}{873960180913} + \frac{1}{1527612795642093418846225}
\]

While there is another decomposition of
\[
\frac{5}{121} = \frac{1}{33} + \frac{1}{121} + \frac{1}{363}
\]

We define the `best' decomposition to be the one with the \underdot{fewest} Egyptian fractions, if two ways of decomposition share the same number of Egyptian fractions, the less denominator the better. \Cref{app:best-egyptian-decomposition} gives an algorithm that searches the best decomposition. Because Fibonacci's method reduces the nominator at least by 1 each time, the decomposition of $\dfrac{b}{a}$ need at most $b$ Egyptian fractions. However, the above example of $\dfrac{5}{121}$ implies there may be better solution. Erdős and Straus conjectured in 1950, that any natural number $n$ greater than 1 can always be decomposed into three Egyptian fractions, \index{Erdős-Straus conjecture}

\[
\frac{4}{n} = \frac{1}{a} + \frac{1}{b} + \frac{1}{c}
\]

Where $a < b < c$. This interesting problem, known as Erdős-Straus conjecture, hasn't been solved as of 2025.
% https://mathworld.wolfram.com/Erdos-StrausConjecture.html

\begin{mdframed}

\begin{center}
 \includegraphics[scale=0.3]{img/erdos-1992}
 \captionof{figure}{Paul Erdős, 1913-1996}
 \label{fig:erdos-1992}
\end{center}

\index{Erdős} \label{sec:Erdos}
Paul Erdős was a Hungarian `freelance' mathematician known for his work in number theory and combinatorics, and a legendary eccentric who was arguably the most prolific mathematician of the 20th century\cite{Hoffman-Paul-2025}. He published over 1500 papers (including coauthored with others, this number surpassed Euler). He traveled to many places, visited mathematicians and worked with them. Erdős himself published papers with 507 coauthors\footnotemark; this made him the center of a large network of mathematical community. This allows people to define `Erdős number' that measures the influence of a mathematician. Those 507 coauthors gained the coveted distinction of having an `Erdős number of 1', meaning that they wrote a paper with Erdős himself. Those who published a paper with one of Erdős's coauthors had an Erdős number of 2; an Erdős number of 3 meant that someone wrote a paper with someone having Erdős number of 2; and so on. Albert Einstein's Erdős number, for instance, was 2. The lowest median of Erdős number among Fields award winners was 3. The highest known Erdős number is 15; this excludes non-mathematicians, who all have an Erdős number of infinity.

%% 例如 https://www.jmilne.org/math/Personal/index.html 页面底部的埃尔德什数

Erdős was born in 1913 in a Jewish family in Budapest, Hungary. His two elder sisters, aged three and five, both died of scarlet fever; his parents, fearing of his death, extremely cared him. The World War I broke out when Paul was not much over a year old. His father Lajos was captured by Russian army as it attacked the Austro-Hungarian troops, and spent six years in captivity in Siberia. Surprisingly, Lajos self-learned English to pass the long hours in captivity, but had strange accent for having no English teacher. Erdős was taught English by his father, which resulted in characteristic of the strange accent throughout life. The situation in Hungary was chaotic at the end of World War I, the strikes happened here and then. Erdős's mother Anna, as a head teacher, insisted on teaching at school even under a call for strike action. She continued working simply because didn't want children's education suffer; but this became the reason she was dismissed from her post after political power changed.By 1920s, anti-Jewish law was introduced to Hungary similar to what happened in Germany 13 years later; Europe was heading to the war.

Despite the bad situation and the restriction on Jews, Erdős entered the University Pázmány Péter in Budapest as the winner of national examination. He was awarded a doctorate in 1934, then forced to leave Hungary because he was Jewish. Erdős went to Manchester UK for a post-doctoral fellowship. He travelled widely in the UK; during that time, he met Hardy and Ulam in Cambridge. The latter became his life-lone friend. He was unable to return to Hungary because of the tough situation for Jewish by the late 1930s. When Hitler took control of Austria and Czech, Erdős had to cancel his plan to visit his parents in the middle way in 1938. He hurried returning to English, then after a few weeks, went to US for a one year fellowship at the Institute for Advanced Study in Princeton, New Jersey, where he cofounded the field of probabilistic number theory. He hoped for a renewal of fellowship, but was offered only a six month extension. At this hard time, his friend Ulam invited Erdős to visit University of Wisconsin-Madison to help out. This starts Erdős's strange life: he wandered about the US (then later to other countries) from one university to the next - Purdu, Stanford, Notre Dame, Johns Hopkins - devoted exclusively to seeking out and solving good mathematical problems.

Erdős had definite ideas about mathematical elegance for his early career. He believed that God had a transfinite book (“transfinite” being a mathematical concept for something larger than infinity) that contained the shortest, most beautiful proof for every conceivable mathematical problem. The highest compliment he could pay to a colleague’s work was to say, ``That’s straight from The Book.'' In 1845 Bertrand conjectured that there was always at least one prime between $n$ and $2n$ for $n \geq 2$; Chebyshev proved Bertrand's conjecture in 1850 but when Erdős was only an eighteen year old student in Budapest he found an elegant elementary proof of this result. Erdős was proud that he had found The Book proof. In 1998, the 85 anniversary of Erdős's birth, Martin Aigner and Günter M. Ziegler compiled the most elegant proofs to 32 theorems (one theorem often containing multiple proofs) into a book titled \emph{Proofs from THE BOOK}.

In the World War II, Erdős fully lost connections with his family in Hungary. After the war in August 1945, he received a telegram giving details of his family: his father had died of a heart attack in 1942; his mother had survived; four of his uncles and aunts had been murdered. Quite remarkably, a cousin had been sent to Auschwitz but had survived. Near the end of 1948, he was able to return to Hungary for a visit where he reunited with his surviving family and friends\cite{MacTutor-Erdos-2000}. For the next half a century, Erdős travelled frequently. He rejected any full-time job offers so that he would have the freedom to work with anyone at any time on any problem of his choice. The nomadic existence made him a legend in the mathematics community. With no home, no wife, and no job to tie him down, his wanderlust took him to Israel, China, Australia, and 22 other countries (although sometimes he was turned away at the border—during the Cold War, Hungary feared he was an American spy, and the United States feared he was a communist spy). Erdős would show up — often unannounced — on the doorstep of a fellow mathematician, declare ``My brain is open!'' and stay as long as his colleague served up interesting mathematical challenges.

Erdős was elected to many of the world’s most prestigious scientific societies, including the Hungarian Academy of Science (1956), the U.S. National Academy of Sciences (1979), and the British Royal Society (1989). In 1951 John von Neumann presented the Cole Prize to Erdős for his work in prime number theory. In 1984 he won the most lucrative award in mathematics, the Wolf Prize, and used all but \$720 of the \$50,000 prize money to establish a scholarship in his parents’ memory in Israel. Erdős went on proving and conjecturing until the age of 83, succumbing to a heart attack only hours after disposing of a nettlesome problem in geometry at a conference in Warsaw in 1996.
\end{mdframed}
\footnotetext{Some saying 511 coauthors. Over 70 papers with his name have appeared since his death\citepage[p588]{Stillwell-2010}.}

\section{Babylonian fractions}
When looking back, Egyptian fractions were hard to use and compute; it was in side way out of the main stream of mathematical history. It has more formal significance than its practical applications. Neither the Greeks nor the Romans developed fractions in arithmetic independently, instead they introduced dedicated terms to represent partial quantities. A unit of weight was \emph{as} and the \emph{unica} (from which derived ``ounce'' in English) was a twelfth part of the as. \Cref{tab:roman-fractions} list most used terms for partial quantities in ancient Roman.

\index{Roman fractions}

% https://mathshistory.st-andrews.ac.uk/Miller/mathsym/fractions/
\begin{table}[htbp]
  \centering
  \renewcommand{\arraystretch}{1.5}
  \begin{tabular}{|c|l|l||c|l|l|}
  \hline
  value & term & meaning & value & term & meaning \\
  \hline
  $\frac{1}{12}$ & uncia    & one twelfth & $\frac{9}{12}$ & dodrans    & a quarter taken away \\
  \hline
  $\frac{2}{12}$ & sextans  & one sixth & $\frac{10}{12}$ & dextans   & one sixth taken away \\
  \hline
  $\frac{3}{12}$ & quadrans & a quarter & $\frac{11}{12}$ & deunx    & a uncia taken away \\
  \hline
  $\frac{4}{12}$ & triens   & one third & $\frac{1}{24}$ & semuncia   & half a uncia  \\
  \hline
  $\frac{5}{12}$ & quincunx & five uncia & $\frac{1}{48}$ & sicilicus &  \\
  \hline
  $\frac{6}{12}$ & semis    & half    & $\frac{1}{72}$ & scriptulum    &  \\
  \hline
  $\frac{7}{12}$ & septunx  & seven uncia & $\frac{1}{144}$ & scripulum  &  $\frac{1}{12} \times \frac{1}{12}$ \\
  \hline
  $\frac{8}{12}$ & bes      & two third & $\frac{1}{288}$ & scrupulum    &  $\frac{1}{24} \times \frac{1}{12}$\\
  \hline
  \end{tabular}
  \caption{Roman fractions}
  \label{tab:roman-fractions}
\end{table}

Many terms left in our languages nowadays. For example, from `semi', the term of $\dfrac{1}{2}$, derived semicircle, semicolon, semiconductors and so on. The term `quadrans' for $\dfrac{1}{4} = \dfrac{3}{12}$ became `quarter' in English, where derived a quarter of an hour and a quarter of a year.

\index{Babylonian decimals}
We know Babylonian fractions from some material evidence: a tablet collected in Yale University as shown in \cref{fig:babylonian-yale}. It inscribed a square and its diagonal. Aside a side of the square inscribed 30 (3 wedges on up right), below the diagonal inscribed 42, 25, 35. It was also inscribed 1, 24, 51, 10 along the diagonal. We learned from school math that the length of the diagonal is the $\sqrt{2}$ times of the side, hence is $30 \times \sqrt{2} \approx 30 \times 1.4142 = 42.426$ for the side of 30. Note that the integral part, 42, is the first number inscribed below the diagonal. As the Babylonian used 60 base, are the followed numbers 25 and 35 the fractional part? Let's verify:

\begin{figure}[htbp]
 \centering
 \includegraphics[scale=0.8]{img/babylonian-yale}
 \caption{Tablet with identifier Y7289 collected in Yale University.}
 \label{fig:babylonian-yale}
\end{figure}

\[
\frac{25}{60} + \frac{35}{60^2} \approx 0.4264
\]

It showed that the Babylonian correctly gave the length of the diagonal: the value of 60-based numbers 42, 25, 35 is $42.426 \approx 30 \times \sqrt{2}$. Then is the sequence of 1, 24, 51, 10 another fraction? Let's try:

\[
1 + \frac{24}{60} + \frac{51}{60^2} + \frac{10}{60^3} \approx 1.414213
\]

It is the approximation of $\sqrt{2}$, with precision up to 6 places! While these numbers are ambiguous for lack of the point symbol, for example, the sequence 12, 15 may represent $12 \times 60 + 15 = 720$ or $12 + \dfrac{15}{60} = 12.25$. Moreover, Babylonian didn't use zero in a unified way, people have to guess the exact value from the context.

As the base of 60 was big enough, the Babylonian could get sufficiently fine quantity from $\dfrac{1}{60}$ to $\dfrac{59}{60}$. With 2 to 3 places, the fractional part was precise enough for daily life or astronomical observation. Besides, base of 60 has plenty of proper factors\footnote{A factor being not of 1 or $n$ is called proper factor of $n$.} including 2, 3, 4, 5, 6, 10, 12, 15, 20, 30, hence has less chance to repeat when dividing (comparing to 10, which only has two proper factors of 2 and 5, leads to repeating decimals in common, see \cref{thm:cyclic-decimal}). These might be the reasons that the Babylonian didn't developed general fractions.

\section{Chinese fractions}
\label{sec:chinese-fractions}

\index{Nine chapters on the Art of mathematics}
People find complete arithmetic rules for fractions with varies of problems from \emph{Nine chapters on the Art of mathematics}, a classic Chinese mathematical book in Han Dynasty (about 1st century CE, often abbreviated as \emph{Nine chapters}). There was a dedicated chapter (see \cref{fig:jiuzhang}) about fractions. It started from the notation of ``$a$分之$b$'', meaning $\dfrac{b}{a}$, which is still using in Chinese today. The denominator $a$ comes first then follows with $b$; read as out of $a$ parts taken $b$. The Chinese had mixed number, for example, 8 steps and $\dfrac{1}{3}$ step to measure the length of a piece of land. They derived the notion of fractions from division, for example from \emph{Nine chapters}: seven merchants sold four horses, each merchant sold $\dfrac{4}{7}$ horse.

\begin{figure}[htbp]
 \centering
 \includegraphics[scale=0.4]{img/jiuzhang}
 \caption{The Nine Chapters on the Mathematical Art}
 \label{fig:jiuzhang} % The Nine Chapters on the Mathematical Art
\end{figure}

\index{fraction!reduce}
With the notation of fractions, the author of \emph{Nine chapters} next dealt with the problem that whether the value of a fraction was definitive, in other word, will multiple notations correspond to the same value? This leads to the reduction operation and the concept of lowest terms (irreducible fraction). The author explained with example of problems:

\begin{enumerate}[1)]
\item Now there is twelve eighteenth ($\dfrac{12}{18}$), if reduce to its lowest terms, what is the result? The answer is two third. ($\dfrac{12}{18} = \dfrac{2}{3}$)
\item There is forty nine ninety-first ($\dfrac{49}{91}$), if reduce to its lowest terms, what is the result? The answer is seven thirteenth. ($\dfrac{49}{91} = \dfrac{7}{13}$)
\end{enumerate}

\index{Euclidean algorithm} \label{sec:gcd-minus}
How to reduce a fraction to its lowest terms? the author of \emph{Nine chapters} gave a method which was exactly known as Euclidean algorithm in the west, ``First, halve the nominator and denominator repeatedly if both are even; second, for the two numbers of the nominator and denominator, subtract the less one from the greater one repeatedly until they become the same value; finally, reduce the fraction with this value.'' For fraction $\dfrac{b}{a}$, if both $a$ and $b$ are even, divide each by 2 and repeat; otherwise, denote the greatest common divisor of $a$ and $b$ by $(a, b)$, then

\be
\label{eq:gcd-minus}
(a, b) = \begin{cases}
  a > b :& (a - b, b) \\
  a = b :& a \\
  a < b :& (a, b - a)
  \end{cases}
\ee

If $a = b$, the their greatest common divisor is $a$ (or $b$). If $a > b$, then subtract $b$ from $a$; it turns the original problem of finding the greatest common divisor of $a$ and $b$, to a new problem of finding the greatest common divisor of $a - b$ and $b$. If $a < b$, then subtract $a$ from $b$; the original problem turns to find the greatest common divisor of $a$ and $b - a$. We delay the proof to Euclidean algorithm to \cref{sec:Euclidean-algorithm}. Euclidean algorithm calculates the great common divisors much faster\footnote{The improved Euclidean algorithm in practice implements the repeat of subtraction in \cref{eq:gcd-minus} as division, which calculates in logarithmic time. For numbers of 100 digits, it can find the greatest common divisor in about 8 recursions.} than the method of factorization, which needs to first factorize $a$ and $b$ into primes, then pick the most common factors and calculate their product. It is suitable for computer machine implementation and becomes the fundamental algorithm in number theory and cryptography\footnote{Euclidean algorithm is also important in abstract algebra, such as ring and field theory.}. Below example shows the steps to find the greatest common divisor of 18 and 12 with Euclidean algorithm.

\[
(18, 12) = (18 - 12, 12) = (6, 12) = (6, 12 - 6) = (6, 6) = 6
\]

It gives the greatest common divisor in three steps. To reduce to the lowest terms, divide the nominator and denominator by 6: $\dfrac{12}{18} = \dfrac{12/6}{18/6} = \dfrac{2}{3}$.

The author of \emph{Nine chapters} next introduced addition for fractions with below example problems:

\begin{enumerate}[1)]
\item What is the sum of one third and two fifth? The answer is eleven fifth. ($\dfrac{1}{3} + \dfrac{2}{5} = \dfrac{11}{15}$)
\item What is the sum of two third, four seventh, and five ninth? The answer is one and fifty sixty-third. ($\dfrac{2}{3} + \dfrac{4}{7} + \dfrac{5}{9} = 1\dfrac{50}{63}$)
%% \item 又有二分之一,三分之二,四分之三,五分之四,问合之得几何?答曰:得二、六十分之四十三。(即:$\dfrac{1}{2} + \dfrac{2}{3} + \dfrac{3}{4} + \dfrac{4}{5} = 2\dfrac{43}{60}$)
\end{enumerate}

The author explained, `let the sum of the products of the each denominator and the other nominator be the nominator of the result; let the product of the two denominators be the denominator of the result.' that is,

\be
\frac{b}{a} + \frac{d}{c} = \frac{bc + ad}{ac}
\ee

As the result may not be in the lowest terms, the author continued with a special case, `the result is 1 if the nominator equals to the denominator', and followed with the mixed number for the general case. At last the author pointed out for the special simple case that the two denominators are same, one add the nominator directly.

\be
\frac{b}{a} + \frac{c}{a} = \frac{b + c}{a}
\ee

The author of \emph{Nine chapters} didn't reduce the fractions to their least common denominator terms, but just multiplied the denominators. We skip the subtraction part in \emph{Nine chapters} as it is similar to addition. It's not so obvious to pick the greater one from two fractions than two natural numbers, for example between $\dfrac{4}{7}$ and $\dfrac{3}{5}$. The author gave a remarkable general rule, which was essentially to compare the difference with zero by converting the two fractions of $\dfrac{b}{a}$ and $\dfrac{d}{c}$ to $bc$ and $ad$ and comparing. After defining the mean value of multiple fractions, the author extended it to the application, like, `Now there are seven persons dividing the money of eight and one third, how much does each individual get? The answer is one and four twenty-first for each.' It is $8\dfrac{1}{3} \div 7 = 1\dfrac{4}{21}$. The author explained, `Put the number of persons as the denominator, the total money as the nominator, then reduce it to the lowest term.' The author did not consider the application of fractional denominator. The last part is about the multiplication of fractions.

\begin{enumerate}[1)]
\item Now there is a land with length of four seventh a step and width of three fifth a step, what is the area? The answer is twelve thirty-fifth of a square step ($\dfrac{4}{7} \times \dfrac{3}{5} = \dfrac{12}{35}$).
\item There is a land with length of seven ninth a step and width of nine eleventh a step, what is the area? The answer is seven eleventh a square step ($\dfrac{7}{9} \times \dfrac{9}{11} = \dfrac{7}{11}$).
%% \item 又有田广五分步之四,从九分步之五,问为田几何?答曰:九分步之四。(即:$\dfrac{4}{5} \times \dfrac{5}{9} = \dfrac{4}{9}$)
\end{enumerate}

作者给出的计算规则是:“母相乘为法,子相乘为实,实如法而一。”即分子分母相乘再化带分数。有趣的是《九章算术》对带分数乘法进行了单独处理:

\begin{enumerate}[1)]
\item 今有田广三步三分步之一,从五步五分步之二,问为田几何?答曰:十八步。(即:$3\dfrac{1}{3} \times 5\dfrac{2}{5} = 18$)
\item 又有田广七步四分步之三,从十五步九分步之五,问为田几何?答曰:一百二十步九分步之五。(即:$7\dfrac{3}{4} \times 15\dfrac{5}{9} = 120\dfrac{5}{9}$)
%% 又有田广十八步七分步之五,从二十三步十一分步之六,问为田几何?答曰:一亩二百步十一分步之七。
\end{enumerate}

作者解释说:“分母各乘其全,分子从之,相乘为实。分母相乘为法。实如法而一。”即化成假分数,相乘再化带分数。例如:

\[
3\frac{1}{3} \times 5\frac{2}{5} = \frac{10}{3} \times \frac{27}{5} = \frac{10 \times 27}{3 \times 5} = 18
\]

《九章算术》一开篇就系统地引入了分数及其四则运算。可见中国古人对分数的重视。后继的各种图形面积、方程计算都建立在这个基础上。

\section{印度分数和小数}

\index{印度分数}
我们今天使用的分数记法源自古印度。印度人将表示分子的数写在上方,将表示分母的数写在下方,但并未使用分数线,如:

\begin{align*}
  3 \\
  4
\end{align*}

\index{分数线}
表示四分之三。后来阿拉伯人在约公元1200年引入了分数线来分开分子与分母\cite{Pumfrey-2011},如:

\[
\dfrac{3}{4}
\]

斐波那契最早将现代分数记法介绍到欧洲,但在写带分数时,他照搬了阿拉伯人从右向左的书写习惯,将分数写在整数的左边\cite{Miller-2025}。分数线带来了不小的印刷困难。在1718年,英国著名的川宁茶叶公司创世人托马斯·川宁\footnote{Thomas Twining}在记账时用斜线代替水平分数线,如:1/4磅茶叶。这样就便于印刷和用打字机书写分数。今天很多字处理和编辑软件可以方便输入分数。不少网页支持了 \LaTeX 排版,可以用\lstinline|\frac{b}{a}|来输入$\dfrac{b}{a}$。

\index{小数} \index{数学家!刘徽}
随着印度——阿拉伯计数系统的确立,阿拉伯学者开始引入了十进制小数。数学家阿尔·卡西(Al-Kashi,公元920年~980年)在他的著作《圆周的研究》\footnote{{\em al-Risali al-mohitije}译作英文{\em Treatise on the circumference}}中将圆周率$\pi$的值写为:“sah-hah 3 14159”,其中sah-hah的意思是整数部分(今天土耳其语中的sahih),修饰了3,接下来是小数部分。可见此时尚未引入小数点。中国魏晋时的数学家刘徽(见\cref{fig:liuhui})在注解《九章算术》时,不满于古代用3作为圆周率,于是:“微数无名知以为分子,以十为分母”,决定采用十进分数来求得更精确的值。当他用正96边形逼近圆时求得:“三百一十四寸六百二十五分寸之一百六十九”,这个值的表示已经非常接近小数的形式了\footnote{关于这个值,刘徽接着写道:“则出圆之表矣。故还就一百九十二觚之全幂三百一十四寸以为圆幂之定率而弃其余分。”他觉得大了,故得到此时的圆周率3.14。但他不甘心,又继续用192边形计算到“三百一十四寸二十五分寸之四”。这个带分数是$314\dfrac{4}{25} = 314.16$寸。可见刘徽是混合使用小数和分数的。}。

\begin{figure}[htbp]
 \centering
 \subcaptionbox{纳皮尔肖像,1616年油画,爱丁堡大学收藏\label{fig:napier}}{\includegraphics[scale=0.5]{img/napier}} \quad
 % https://www.britannica.com/biography/John-Napier
 \subcaptionbox{蒋兆和创作的刘徽像,部编数学课本\label{fig:liuhui}}{\includegraphics[scale=0.46]{img/liuhui}}
 % https://wapbaike.baidu.com/tashuo/browse/content?id=b1fc49ab180770763a28420d
\end{figure}

\index{小数点} \index{数学家!纳皮尔} \index{百分数}
1530年,鲁道夫(Christoff Rudolff,1499~1545年)使用一条数线分隔整数和小数部分。后来马吉尼(Magini,1555年~1617年)将竖线改进成了小数点,但并未得到广泛使用。直到20年后英国学者约翰·纳皮尔(John Napier,1550年~1617年,见\cref{fig:napier})发明对数时使用了小数点,现代小数形式才被广泛接受。我们常用的百分数符号\%是佚名的某位意大利人在1425年引入的。

\index{基点}
从另一角度看,小数是分母为10、100、1000……的分数的和。百分数看似是分母为100的分数,实则只是小数乘以100的结果。这是因为百分数的分子不一定是整数。例如某网站说它的可靠性非常高,达99.999\%(俗称5个9),这相当于小数0.99999。在日常生活中,人们还会用千分数,写作999.99\textperthousand。在金融领域还会用基点(BPS:base points的缩写)这个量。1bps = 0.01\% = 0.0001。比如我们看见新闻说美联储(美国联邦储备委员会,是美国的中央银行的核心管理机构)宣布降息25个基点。实际上就是降低利息0.25\% = 0.0025。例如利率从2.5\%降低25个基点就变成2.5\% - 0.25\% = 2.25\%。

\subsection{分数与小数的关系}

现在我们回过头来说为何小数是分母为10、100、1000……的分数和?并提出一个问题:是否每个分数都可以\underdot{唯一}地表示成小数?举例来说0.125 = $\dfrac{1}{10} + \dfrac{2}{100} + \dfrac{5}{1000}$,恰好是分母为10的$\dfrac{1}{10}$,分母为100的$\dfrac{2}{100}$,分母为1000的$\dfrac{5}{1000}$的和。一般来说,小数$0.a_1 a_2 \dots a_n$可表示为:

\be
a_1 \frac{1}{10} + a_2 \frac{1}{100} + \cdots + a_n \frac{1}{10^n}
\label{eq:decimal-as-sum}
\ee

\index{无限循环小数}
我们在数学课上知道,小数分为有限小数(如0.125)和无限小数(如$0.\dot{3} = 0.333\cdots, 0.\dot{1}\dot{5} = 0.1515\cdots, \pi = 3.14159\cdots$)。根据\cref{eq:decimal-as-sum},无限小数可写成无限个分数的和。那么这无限个正数的和是个有限值还是无限大?为了解答这个问题,我们先引入一个看似反直觉的命题:

\begin{lemma} \label{th:cycle-of-9}
  无限循环小数$0.999\cdots$的值是1。
\end{lemma}

乍看上去左边的$0.999\cdots$和右边的1明显是两个不同的数,它们怎么会相等呢?

\begin{proof}
  令$x = 0.999\cdots$,扩大10倍:$10x = 9.999\cdots$,相减:
  \begin{align*}
    10x - x &= 9.999\cdots - 0.999\cdots && \\
         9x &= 9 & \text{右侧的小数部分减光了} & \\
          x &= 1  && \qedhere
  \end{align*}
\end{proof}

注意,小数部分只有有限个9时等号不成立,哪怕有$n = 10000$个9。因为$9.99\cdots9$(个位的9和小数部分的9999个9)减$0.99\cdots9$(小数部分的1万个9)等于$8.99\cdots91$。因此$x = \dfrac{8.99\cdots91}{9} \ne 1$。即1万位有限小数$0.99\cdots9 \ne 1$。这样无限个正数的和$\dfrac{9}{10} + \dfrac{9}{100} + \cdots = 1$,是个确定的有限值\footnote{我们也可以用高中数学的极限概念进行证明。注意到$n$位有限小数:$0.99\cdots9 = 1 - 0.00\cdots01 = 1 - \dfrac{1}{10^{n+1}}$,所以$0.999\cdots = \lim\limits_{n\to\infty} (1 - \dfrac{1}{10^{n+1}}) = 1 - \lim\limits_{n\to\infty} \dfrac{1}{10^{n+1}} = 1 - 0 = 1$。}。

\begin{corollary}
  把任何无限小数$0.a_1a_2\cdots$写成无穷多个分数的和:$a_1 \dfrac{1}{10} + a_2 \dfrac{1}{100} + \cdots$这个和是有限的。
\end{corollary}

\begin{proof}
  由于每个小数位$0 \leq a_i \leq 9$,所以
  \begin{align*}
0 \leq 0.a_1 a_2 \cdots = a_1\frac{1}{10} + a_2\frac{1}{10^2} + \cdots \leq \frac{9}{10} + \frac{9}{10^2} + \cdots = 0.999\cdots = 1 & \qedhere
  \end{align*}
\end{proof}

\index{芝诺悖论!阿基里斯与乌龟悖论}
这个结论:$0.a_1a_2\cdots \leq 1$看似说了一句废话,但无限多个(正)值加起来是有限的是一个极为惊人的数学与逻辑断言。它是解决一个有着两千多年历史的悖论的钥匙。古希腊的哲学家,埃利亚的芝诺(Zeno of Elea,约公元前490~425年)提出了著名的四个悖论。第一个悖论最为人们所津津乐道,名叫阿基里斯与乌龟悖论。阿基里斯是荷马史诗《伊里亚特》中的英雄,以英勇著称。这个悖论说:如果让爬得很慢的乌龟在阿基里斯前面一段路程出发,那么阿基里斯将永远追不上乌龟。这是因为,阿基里斯为了赶上乌龟,必须先到达乌龟的出发点$A$,但当阿基里斯到达$A$点时,乌龟已经在这段时间前进到了$B$点。但当阿基里斯到达$B$点时,乌龟又已经到了前面的$C$点……以此类推,两者间的距离虽然越来越近,但阿基里斯总是落在乌龟的后面而追不上乌龟。如\cref{fig:Achilles-paradox}所示。

\begin{figure}[htbp]
 \centering
 \includegraphics[scale=0.4]{img/achilles-paradox}
 \caption{阿基里斯与乌龟悖论}
 \label{fig:Achilles-paradox}
\end{figure}

但这与我们生活中的常识是不相符的。我们在小学数学课上学习过“追及问题”,通常利用相对运动加以解决。令乌龟的速度为$v_1$,阿基里斯的速度为$v_2$,则阿基里斯相对于乌龟的运动速度为$v_2 - v_1$。若乌龟在阿基里斯前方距离$s$处,则阿基里斯在$t = \dfrac{s}{v_2 - v_1}$时与乌龟相遇并接下来超过乌龟。但系统严密的相对运动概念要等到现代科学之父伽利略在两千多年之后才提出。芝诺的推理是如此让人信服,以至于千百年来吸引了无数学者的研究。刘易斯·卡罗尔(Lewis Carrol)、侯世达(Douglas Hofstadter)甚至拿乌龟和阿基里斯作为文学作品中的主人公。

阿基里斯与乌龟悖论的解决通常需要极限的概念。但我们手里的武器——作为分数和的小数已经足够强大到解决它。为了让推理更直观,假设阿基里斯的速度是乌龟的10倍,即$v_2 : v_1 = 10$,乌龟在阿基里斯前$s = 100$米\footnote{古希腊类似的长度单位叫做Pous,翻译为“尺”,也叫做“脚长”,是一种基于人体尺度的长度单位,大约合0.308米。}处。按照芝诺的推理,阿基里斯需要先跑完$s = 100$米,此时乌龟向前爬了$\dfrac{1}{10}s = 10$米;接下来阿基里斯要继续跑这10米,但乌龟又向前爬了$\dfrac{1}{100}s = 1$米……所以阿基里斯要跑的距离是:

\[
s(1 + \frac{1}{10} + \frac{1}{100} + \cdots) = 100\times(1 + 0.1 + 0.01 + \cdots) = 100 \times 1.11\cdots = 111.11\cdots
\]

假设阿基里斯是个“百米飞人”,只用9秒就可以跑完100米。但接下来他需要再用0.9秒跑完10米,然后再用0.09秒跑完1米……总共需要的时间是:

\begin{align*}
9 + 0.9 + 0.09 + \cdots &= 9.99\cdots = 10 \times 0.99\cdots & \\
                        &= 10 \times 1 & \text{据引理\ref{th:cycle-of-9}} \\
                        &= 10
\end{align*}

这样即使按照芝诺的推理,阿基里斯在10秒时就可以追上并超越乌龟。

\subsection{无限循环小数}
\index{无限循环小数}
为了把既约分数$\dfrac{b}{a}$转化为小数,我们用$b$除以$a$。有的能除尽,如$\dfrac{1}{2} = 0.5$、$\dfrac{1}{8} = 0.125$、$\dfrac{1}{25} = 0.4$,有的除不尽产生循环,如$\dfrac{1}{3} = 0.\dot{3} = 0.333\cdots$、$\dfrac{1}{7} = 0.\dot{1}\dot{4}\dot{2}\dot{8}\dot{5}\dot{7}$。什么时候能除尽?什么时候除不尽?这好像是一个运气问题。实际上我们的运气非常“差”,通常是除不尽的。下面的定理揭示了我们要有多幸运才能除尽。

\index{有限小数}
\begin{theorem} \label{thm:finite-decimal}
  当且仅当既约分数$\dfrac{b}{a} = \dfrac{b}{2^{\alpha}5^{\beta}}$时能化成有限小数,其中$\alpha$、$\beta$是非负整数。
\end{theorem}

\begin{proof}
令$n = \max(\alpha, \beta)$,即$\alpha$、$\beta$中的较大者。故$n - \alpha \geq 0$,$n - \beta \geq 0$。

\[
10^n \frac{b}{a} = \frac{2^n 5^n b}{2^{\alpha} 5^{\beta}} = 2^{n - \alpha} 5 ^{n - \beta} b
\]

一定是整数。所以$\dfrac{b}{a}$的小数部分有$n$位,即$0.a_1 a_2 \cdots a_n$是有限小数。

反过来,$n$位有限小数$0.a_1 a_2 \cdots a_n$展开成分数和为:

\begin{align*}
\frac{a_1}{10} + \frac{a_2}{10^2} + \cdots + \frac{a_n}{10^n} &= \frac{a_1 10^{n-1} + a_2 10^{n-2} + \cdots + a_n}{10^n} & \text{通分} \\
  &= \frac{B}{10^n} & \text{记分子为}B \\
  &= \frac{B}{2^n 5^n} = \frac{b}{a} & \text{约分}
\end{align*}

因为分母$10^n = 2^n 5^n$只有因子2和5的幂\footnote{幂在《说文解字》中来自象形文字“$\cap$”,本意是盖东西用的巾。数学中取反复覆盖的意思表达自乘。},所以约分后$a$也只有2和5的幂。
\end{proof}

有了这个结论,当我们看到一个既约分数的分母含有2和5以外的其它因子时,比如3、7、11……则它一定不能化成有限小数。这个定理的证明还可以扩展到$b$进制小数。如果分母含有任何不能整除$b$的因子,则除不尽,不能转换成$b$进制有限小数。所以$\dfrac{5}{12} = 0.41666\cdots$尽管不能化成10进制有限小数,但可以化成古巴比伦60进制小数0, 25。而$\dfrac{5}{14}$却不能化成60进制有限小数。特别地,形如$\dfrac{b}{2^n}$以外的任何既约分数都不能化成二进制有限小数,这就造成了电子计算机内部的小数计算的误差。

除此之外的分数就只能化成无限循环小数了(下一章我们将看到既约分数不可能产生无限\underdot{不}循环小数)。但我们发现有的有一位循环节,如$\dfrac{1}{3} = 0.\dot{3}$,有的有两位,如$\dfrac{5}{11} = 0.\dot{4}\dot{5}$,有的更长,如$\dfrac{1}{7} = 0.\dot{1}\dot{4}\dot{2}\dot{8}\dot{5}\dot{7}$,有的一开始不循环,后面才开始循环,如$\dfrac{5}{12} = 0.41\dot{6}$,这里面有什么规律么?

\begin{proposition}
$\dfrac{a}{9}$的小数形式是$0.\dot{a} = 0.aaa\cdots$
\end{proposition}

\begin{proof}
  \begin{align*}
    \text{令}x &= 0.\dot{a} = 0.aaa\cdots && \\
           10x &= a.aaa\cdots && \\
       10x - x &= 9x = a.aaa\cdots - 0.aaa = a && \\
             x &= \frac{a}{9} && \qedhere
  \end{align*}
\end{proof}

所以$\dfrac{1}{3} = \dfrac{3}{9} = 0.\dot{3} = 0.333\cdots$进一步:

\begin{proposition} \label{th:cyclic-decimal}
无限循环小数$0.\dot{a_1} \dot{a_2} \cdots \dot{a_n}$的分数形式是$\dfrac{a_1 a_2 \cdots a_n}{99 \cdots 9} = \dfrac{a_1 a_2 \cdots a_n}{10^{n+1} - 1}$。
\end{proposition}

我们把此命题的证明留作\cref{qn:cyclic-decimal}。可以用$\dfrac{1}{7}$验证一下:

\begin{center}
\opmul[displayshiftintermediary=all,
       voperator=bottom,
       voperation=top]{142857}{7}
\end{center}

故:
\[
0.\dot{1}\dot{4}\dot{2}\dot{8}\dot{5}\dot{7} = \frac{142857}{999999} = \frac{142857}{142857 \times 7} = \frac{1}{7}
\]

\index{循环节}
这个命题使得我们可以轻松看出$\dfrac{2}{11}$、$\dfrac{4}{33}$等分数的循环小数表示,例如$\dfrac{2}{11} = \dfrac{2 \times 9}{11 \times 9} = \dfrac{18}{99} = 0.\dot{1}\dot{8}$。通过这个命题还可以发现循环节长度的规律。如果$\dfrac{a_1 a_2 \cdots a_n}{99 \cdots 9}$约分后是$\dfrac{b}{a}$,那么$ak = 999 \cdots 9$。也就是说$ak + 1 = 100 \cdots 0$。因此如果$100\cdots 0$($n$个0)除以$a$的余数是1,那么$\dfrac{b}{a}$的循环节的长度\underdot{最长}为$n$。循环节的长度仅和分母有关\footnote{数论中的费马小定理断言,素数$p$除$10^{p-1}$的余数等于1。例如1000000除以7余1,$\dfrac{b}{7}$的循环节长6。尽管$10^{11-1} = 10000000000$除以11余1,但$\dfrac{2}{11} = 0.\dot{1}\dot{8}$的循环节长2。我们需要更多的数论知识,例如欧拉定理,才能唯一确定循环节的长度。这超出了本书的范围。参见《哈代数论》IX}。我们还需要最后一个定理来回答什么时候开始循环的问题。

\begin{theorem}\label{thm:cyclic-decimal}
既约真分数$\dfrac{b}{2^{\alpha} 5^{\beta} q}$,其中$q$不含任何2、5为因子。令$n = \max(\alpha, \beta)$,则其小数形式从$n$位后开始循环,循环节的长度是$m$,并且$10^{m+1}$除以$q$余1。
\end{theorem}

\begin{proof}
我们组合之前的证明过程。首先证明前$n$位不循环:

\[
10^n \frac{b}{2^{\alpha} 5^{\beta} q} = \frac{2^n 5^n b}{2^{\alpha} 5^{\beta} q} = \frac{2^{n - \alpha} 5^{n - \beta} b}{q} = X + \frac{b'}{q}
\]
这相当于扩大$10^n$后化带分数,其中$X$是整数部分,$0 \leq X < 10^n$。接下来由于$10^{m+1}$除以$q$余1,所以它减1后能被$q$整除:

\begin{align*}
  (10^{m+1} - 1) \frac{b'}{q} &= k b' = a_1 a_2 \cdots a_m & \text{是整数}\\
                \frac{b'}{q} &= \frac{a_1 a_2 \cdots a_m}{99\cdots9} & m\text{个}9
\end{align*}
由命题\ref{th:cyclic-decimal}知它是无限循环小数。
\end{proof}

例如$\dfrac{5}{12} = 0.41\dot{6}$,其中$12 = 2^2 \times 5^0 \times 3$,指数2和0的最大值是2,所以2位后开始循环;$q = 3$,循环节长1。

\begin{mdframed}

%% \begin{center}
%%  \includegraphics[scale=0.3]{img/zeno}
%%  \captionof{figure}{芝诺,约490BC - 425BC}
%%  \label{fig:Zeno-of-Elea}
%% \end{center}

\index{埃利亚的芝诺}
芝诺(约公元前490年——公元前425年),古希腊哲学家,生于意大利半岛南部的埃利亚。所以我们常称其为埃利亚的芝诺。关于他的生平,缺少可靠的文字记载。据传,他早年是一个自学成才的乡村孩子,一生经历坎坷,最终遭到一位暴君的陷害而被拘捕、拷打、直至被处死\cite{HanXueTao16}。芝诺是著名的哲学学派——埃利亚学派的代表人物之一,这一学派的领袖是芝诺的老师巴门尼德。巴门尼德认为整个世界是个不变的整体,即“不变的一”,运动、变化与多样性都只是幻象。芝诺以其悖论闻名。他一生曾巧妙地构想出40多个悖论,在流传下来的悖论中以关于运动的四个“无限微妙、无限深邃”的悖论最为著名。他提出这些悖论很可能是为他老师的哲学观点辩护。

\begin{center}
 \includegraphics[scale=0.3]{img/zeno-paradox}
 \label{fig:Zeno-paradox}
\end{center}

\index{芝诺悖论!二分悖论}
除了阿基里斯与乌龟悖论外,另外三个悖论分别叫做“二分悖论”、“飞矢不动悖论”、“运动场悖论”。二分悖论的主人公是希腊神话中善于疾走的女猎手阿塔兰塔。如果阿塔兰塔想从$A$到$B$,那么她必须先走到$1/2$的位置。同样在此之前,她必须要到达$1/4$的位置。而为了到达这一位置,她必须先到达$1/8$的位置……以此类推。由于这样的中点有无限多个,阿塔兰塔永远也也无法到达目的。芝诺利用这个悖论来说明了运动根本无法发生。\cref{qn:dichotomy-paradox}要求读者扩展引理\ref{th:cycle-of-9}来解决二分和的问题。

\begin{center}
 \includegraphics[scale=0.4]{img/dichotomy-paradox}
 \captionof{figure}{二分悖论}
 \label{fig:dichotomy-paradox}
\end{center}

\index{芝诺悖论!飞矢不动悖论}
飞矢不动悖论从另一个角度描述无穷导致运动无法发生。芝诺指出,任何物体停在相同的位置都不叫运动,可是飞行的箭矢在任一时刻不也是停在一个地方么?这样说来,自然飞失也是不动的。如果说前两个悖论是由于分割空间导致的,则这个悖论是由对时间的分割导致的。

\begin{center}
 \includegraphics[scale=0.4]{img/arrow-paradox}
 \captionof{figure}{飞失不动悖论}
 \label{fig:Arrow-paradox}
\end{center}

\index{芝诺悖论!运动场悖论}
第四个悖论叫作“运动场悖论”。这一悖论主要针对时间原子论的观点,即认为存在最小的不可分割的时间单位。如\cref{fig:Moving-rows-paradox}所示,运动场中有3列人。最初他们都首尾对齐。在最小的时间单元内,A列不动,B向右移一个单位,而$\Gamma$向左移动一个单位。容易得知,相对B而言,$\Gamma$其实移动了两个单位。这就意味着,应该存在这一让$\Gamma$相对于B移动一个单位的时间。而这一时间应该是最小单位时间的一半。但如果存在不可分割的“时间原子”,那么这两个时间就是相同的,即最小时间和它的一半相等。

\begin{center}
 \includegraphics[scale=0.3]{img/moving-rows-paradox}
 \captionof{figure}{运动场悖论}
 \label{fig:Moving-rows-paradox}
\end{center}

芝诺悖论并不复杂,稍加琢磨就能理解。但是导出的结果却出人意料。根据生活中的常识,运动和时间是如此真实,阿基里斯不可能赶不上乌龟。可是驳倒这些悖论却并不容易,从亚里士多德到罗素,从阿基米德到赫尔曼·外尔,都对芝诺悖论提出了各种不同的解法\cite{Britannica-Zeno-24}。芝诺悖论在当时曾给古希腊人造成深深的困惑。而芝诺悖论所涉及的对时间、空间、无限、连续、运动的看法,也都在极长的历史岁月中困扰着后来的哲学家和数学家。
\end{mdframed}

\section{数系的扩展}
在某种意义上,分数是人类在历史上第一次正式扩展数系。分数看起来和0、1、2、3……大相径庭。它有两个部分:分子和分母;它抽象了多种不同的东西:小数、百分数、除法、比例$a:b$,它们本质上都是分数。尤其是后两种,它们展示出0、1、2、3……不具备的一个特性:$\dfrac{1}{2} = \dfrac{2}{4} = \dfrac{50}{100} = 3 \div 6 = 7:14 = \cdots$这说明一个值的分数表示是\underdot{不唯一}的。这反映了不同的除法式子可以得出相等的值(或比例)。如果要把分数加入“数”的大家庭,我们就必须解决一系列问题:

\begin{enumerate}[问题1.]
\item 哪些分数(比例)是等价的?
\item 怎样比较分数间的大小?怎样比较分数与其它数的大小?分数在数轴上的位置是什么?
\item 分数间的四则运算是怎样的?分数与其它数的四则运算是怎样的?
\item 五大运算定律对分数适用么?加法、乘法的单位元对分数的意义一样么?
\end{enumerate}

\index{等价关系} \label{sec:frac-equiv}
先看问题1,若$\dfrac{b}{a} = \dfrac{d}{c}$,两边乘以$ac$有$bc = ad$,这个结果被形象地称为“交叉相乘”(见\cref{fig:cross-mul})。在小学数学课中,我们知道比例相等时$b : a = d : c$有外项积等于内项积,即$bc = ad$,这和分数等价是一致的。在数学上,一个关系$\sim$要成为\underdot{等价关系}需要满足三个条件:

\begin{enumerate}[条件1.]
\item 自反性,即$a \sim a$。
\item 对称性,若$a \sim b$则$b \sim a$。
\item 传递性,若$a \sim b, b \sim c$则$a \sim c$。
\end{enumerate}

\index{等价关系!分数}
我们不难验证$bc = ad$是一个等价关系。首先验证自反性,比较$\dfrac{b}{a}$和$\dfrac{b}{a}$自己,显然$ab = ab$,满足自反性。其次验证对称性,如果$\dfrac{b}{a}$和$\dfrac{d}{c}$满足$bc = ad$,显然也有$ad = bc$,这样$\dfrac{d}{c}$和$\dfrac{b}{a}$之间也成立,满足对称性。最后验证传递性,若$\dfrac{b}{a}$和$\dfrac{d}{c}$满足$bc = ad$,并且$\dfrac{d}{c}$和$\dfrac{f}{e}$满足$de = cf$。等量相乘有$bcde = adcf$。因为$c \ne 0$是分母,可以消去得$bde = adf$。若$d \ne 0$也可以消去,进一步有$be = af$;否则$d = 0$,则$bc = ad = 0 = de = cf$,因此$b = f = 0$,同样有$be = af$。满足传递性。

\begin{figure}[htbp]
 \centering
 \includegraphics[scale=0.4]{img/cross-mul}
 \caption{交叉相乘}
 \label{fig:cross-mul}
\end{figure}

\index{既约分数} \index{最简分数}
有了这个等价关系,就可以判断任何两个分数(或比例)是否等价。我们可以在众多的分数表示中选取$\dfrac{b}{a}$,且$a, b$没有除1以外的公因子(也称为互素,记作$(a, b) = 1$)的作为\underdot{代表},称为“既约分数”(也叫最简分数)。唯一的例外是0,我们选取0作为$\dfrac{0}{a}$的代表。这样的代表只有一个,因为若$\dfrac{b}{a} = \dfrac{d}{c}$,且各自的分子分母互素,由等价关系$bc = ad$,所以$a$整除$bc$。又因为$a$和$b$互素,所以$a$必然整除$c$;反过来由等价关系$bc = ad$还可以推出$c$整除$ad$。又因为$c$和$d$互素,所以$c$必然整除$a$。这样$a$和$c$互相能够整除对方,所以它们必然相等,即$a = c$。同理可证$b = d$,这就证明了既约分数的代表是唯一的。

也许有读者问,为什么要引入这么复杂的等价关系?为什么不直接约分看结果是否一样?古人一开始的确是自然地采用约分来判断相等的,如同上节《九章算术》中介绍的那样。并且在小学数学课上我们也是这样做的。但随着我们用分数这个武器挑战更大的数和更复杂的问题,就发现新的情况了。乘法远远比除法更容易。面对$\dfrac{8723}{22814}$和$\dfrac{143}{374}$这样的分数时,恐怕很少有人能一眼看出如何约分,即使借助计算器或使用欧几里得算法。但做乘法就快多了,用计算器很快就算出$8723 \times 374 = 3262402$并且$22814 \times 143 = 3262402$,所以它们相等。中学数学把分数拓展到了分式,我们发现这个等价关系依然好用。例如$\dfrac{x^5 - 1}{x^3 - 1}$和$\dfrac{1 + x + x^2 + x^3 + x^4}{1 + x + x^2}$,同学们对它们进行因式分解再约分通常会遇到困难\footnote{其中一个方法是:看出$1^n - 1 = 0$,所以$x-1$是$x^n-1$的一个因子。接下来用多项式长除法得出另一个因子。}。但多项式乘法却容易得多:

\begin{align*}
(x^3 - 1)(1 + x + x^2 + x^3 + x^4) &= \cancel{x^3} + \cancel{x^4} + x^5 + x^6 + x^7 - 1 - x - x^2 - \cancel{x^3} - \cancel{x^4} \\
  &= x^7 + x^6 + x^5 - x^2 - x - 1 \\
(x^5 - 1)(1 + x + x^2) &= x^5 + x^6 + x^7 - 1 - x - x^2 \\
  &= x^7 + x^6 + x^5 - x^2 - x - 1
\end{align*}

所以这两个分式是相等的。这就是数学家们为何要抽象出等价关系的概念,并精挑细选地用“交叉相乘”来判断分数、比例、分式相等。

\vspace{3mm}
对于问题2,我们可以利用《九章算术》中的办法,通过$\dfrac{b}{a} - \dfrac{d}{c}$与0的关系来判断大小。这个过程相当于通分后比较新的分子间大小。我们可以把任何整数$n$转化为$\dfrac{n}{1}$,这样所有的大小关系就转化为分数之间的比较。这也意味着任何一个分数都对应到数轴上的一点。对于任何多个数,包括整数、分数、小数都可以对应到数轴上的各自位置。然后从左到右整理出它们的序关系。

\vspace{3mm}
对于问题3,《九章算术》给出了分数间的四则运算规则。概括为加减先通分,再加减分子,最后约分;乘法把分子分母各自相乘、约分。一个数$n \ne 0$的倒数是$\dfrac{1}{n}$,因为$n\dfrac{1}{n} = \dfrac{n}{n} = 1$。如果$n$是分数$\dfrac{b}{a}$,它的倒数是$\dfrac{a}{b}$,因为$\dfrac{b}{a}\dfrac{a}{b} = 1$。分数本质上相当于除法,我们可以通过倒数在分数——除法——乘法间转换:

\[
\frac{b}{a} = b \div a = b \times 1 \div a = b \times \frac{1}{a}
\]

以及:

\[
\frac{b}{a} \div \frac{d}{c} = \frac{b}{a} \times 1 \div \frac{d}{c} = \frac{b}{a} \times \frac{c}{d}
\]

或:

\begin{align*}
\frac{b}{a} \div \frac{d}{c} = \dfrac{\quad\dfrac{b}{a}\quad}{\dfrac{d}{c}} &= \frac{bc}{ad} & \text{上下同} \times ac
\end{align*}

同样,任何整数$n$可以转化为$\dfrac{n}{1}$,从而将整数与分数的四则运算转化为分数间的四则运算。

\vspace{3mm}
\index{封闭运算}
对于问题4,我们发现将分数引入数系后,任何数都可表示为分数:整数$n$包括0,表示为$\dfrac{n}{1}$;分数、有限小数、无限循环小数表示为$\dfrac{b}{a}, a \ne 0$。这个数系称做\underdot{有理数},记作$\mathbb{Q}$。我们将在第4章解释这个名字的含义。在整数$\mathbb{Z}$中,每个数$x$都有相反数(加法的逆元)$-x$,但只有$\pm 1$有倒数(乘法的逆元)。在有理数$\mathbb{Q}$中,每个不等于0的数$x$都有倒数(乘法的逆元)$\dfrac{1}{x}$。自然数$\mathbb{N}$对加法减运算不是封闭的,例如$1 - 3 = -2 \notin \mathbb{N}$。但加减运算在整数$\mathbb{Z}$内是封闭的,任何两个整数$m, n$,总有$m \pm n \in \mathbb{Z}$。整数$\mathbb{Z}$对四则运算不是封闭的,因为$1 \div 2 = \dfrac{1}{2} = 0.5 \notin \mathbb{Z}$。但四则运算在有理数$\mathbb{Q}$内是封闭的,任何两个有理数$p, q$,总有$p \pm q, pq, \dfrac{p}{q} (q \ne 0)$都在$\mathbb{Q}$内。

可以验证五大运算定律对于分数都成立,我们这里验证加法的结合率、乘法对加法的分配律,其余三个留作\cref{qn:arithmatic-law-fraction}。
\begin{proof}
分数加法的结合律:
\begin{align*}
(\frac{b}{a} + \frac{d}{c}) + \frac{f}{e} & = \frac{bc + ad}{ac} + \frac{f}{e} = \frac{bce + ade + acf}{ace} && \\
  & = \frac{bce + (ade + acf)}{ace} & \text{对分子和用结合律} & \\
  & = \frac{b\cancel{ce}}{a\cancel{ce}} + \frac{ade + acf}{ace} && \\
  & = \frac{b}{a} + (\frac{\cancel{a}d\cancel{e}}{\cancel{a}c\cancel{e}} + \frac{\cancel{ac}f}{\cancel{ac}e}) && \\
  & = \frac{b}{a} + (\frac{d}{c} + \frac{f}{e}) && \qedhere
\end{align*}
\end{proof}

\begin{proof}
我们只证明左侧的分数分配律,右侧的证明可以利用分数乘法交换律。
\begin{align*}
  \frac{b}{a}(\frac{d}{c} + \frac{f}{e}) &= \frac{b}{a}\frac{de + cf}{ce} && \\
    &= \frac{b(de + cf)}{ace} = \frac{bde + bcf}{ace} & \text{分子用分配律} & \\
    &= \frac{bd\cancel{e}}{ac\cancel{e}} + \frac{b\cancel{c}f}{a\cancel{c}e} && \\
    &= \frac{b}{a}\frac{d}{c} + \frac{b}{a}\frac{f}{e} && \qedhere
\end{align*}
\end{proof}

\vspace{3mm}
\index{平均律}
让我们回到毕达哥拉斯追寻天籁之音的故事。发现了分数导致和谐的乐音后,如何调整里尔琴上的七根琴弦呢?如果加上第八根琴弦,那么它应该是第一根琴弦的一半,这样才能跨越八度音程。毕达哥拉斯发现纯五度音程$3:2$叠加一个纯四度音程$4:3$恰好是一个八度音程,因为:$\dfrac{3}{2} \times \dfrac{4}{3} = 2$。他用纯五度音程的比例$3:2$作为倍数,不断提高音调,得到一组等比数列:

\[
(\frac{3}{2})^{-1}, (\frac{3}{2})^0, (\frac{3}{2})^{1}, (\frac{3}{2})^2, (\frac{3}{2})^{3}, (\frac{3}{2})^4, (\frac{3}{2})^{5} = \frac{2}{3}, 1, \frac{3}{2}, \frac{9}{4}, \frac{27}{8}, \frac{81}{16}, \frac{243}{32}
\]

其中后面4个数大于2,超过了八度音程。因此毕达哥拉斯把它们每个数都除以2或4,找到其对应的前一个八度音程的数,然后重新从小到大排列得到:

\[
1, \frac{9}{8}, \frac{81}{64}, \frac{4}{3}, \frac{3}{2}, \frac{27}{16}, \frac{243}{128}, \frac{2}{1}
\]

它表示的是音阶中每个音与最低音的比例关系。为了知道每个音的相对关系,于是让相邻的两个数相除,又得到:

\[
\frac{9}{8}, \frac{9}{8}, \frac{256}{243}, \frac{9}{8}, \frac{9}{8}, \frac{9}{8}, \frac{256}{243}
\]

\index{调性}
较大的9:8叫全音,较小的256:243叫半音。这个列表就是“全、全、半、全、全、全、半”,在音乐上叫做“大调”。通常表现欢快的,阳刚的主题。而音乐上的“小调”,对应的列表移动一下是“全、半、全、全、半、全、全”。通常表现含蓄的、柔美的主题。可是有没有可能使得每个比例相同,得到完美的“平均律”呢?这超出了毕达哥拉斯所能想到的所有设计。为了达到这样的天籁之音,我们需要在一个八度音程$2:1$中设置5个全音,2个半音,共$5 \times 2 + 2 = 12$个比例,使得:

\[
1 = \alpha^0, \alpha^1, \alpha^2, \alpha^3, \alpha^4, \alpha^5, \alpha^6, \alpha^7, \alpha^8, \alpha^9, \alpha^{10}, \alpha^{11}, \alpha^{12} = 2
\]

显然$\alpha = \sqrt[12]{2}$,但它不是一个分数!让我们沿着数的旅程继续追寻天籁之音吧。

\begin{Exercise}[label={ex:fractions}]
\Question{“调和”平均值\footnote{英文harmonic mean,意为“和谐的”平均值。}的名字来自毕达哥拉斯的音乐理论。毕达哥拉斯尝试在长度为$a$和$b$的琴弦之间插入一根琴弦$h$,使得它们发出和谐的声音。他发现弦长$h$的倒数恰好在$a$和$b$的倒数正中间,即:\index{调和平均值}
\[
\frac{1}{h} - \frac{1}{a} = \frac{1}{b} - \frac{1}{h}
\]
调和平均值因此也称做倒数平均值。请根据调和平均值的定义验证:1) 在两地之间往返的平均速度是前往的速度和返回速度的调和平均值。例如从船只从南京顺流而下到达上海的平均速度是$v_1$,从上海逆流而上的平均速度是$v_2$,则往返的平均速度$v$是$v_1$与$v_2$的调和平均值。2) 在电路中并联的两个不同电阻$R_1$和$R_2$若等效于并联了\underdot{两个相同}的电阻$R$,则$R$的大小是$R_1$与$R_2$的调和平均值。(注意:$R$相当于并联电阻的2倍)
}

\Question{$\bigstar$验证$\dfrac{2}{p} = \dfrac{1}{a} + \dfrac{1}{b}$等价于$(2a - p)(2b - p) = p^2$,并由此证明:若$p$为大于2的素数,则$\dfrac{2}{p}$可\underdot{唯一}分解为两个埃及分数$\dfrac{2}{p} = \dfrac{1}{a} + \dfrac{1}{b} (a \ne b)$。\label{qn:unique-decompose-r}}

\Question{证明埃尔德什——斯特劳斯猜想对所有偶数$n$成立。}

\Question{实际上我们只要证明埃尔德什——斯特劳斯猜想对于所有素数成立即可。这是为什么?}

\Question{证明埃尔德什——斯特劳斯猜想对所有$4k+3$型的素数成立。提示:利用\cref{qn:unique-decompose-r}的结论。}

\Question{$\bigstar$找出三个儿子按照$\dfrac{1}{a}$、$\dfrac{1}{b}$、$\dfrac{1}{c}$分配$n$个动物的所有可能数值和比例。要求大儿子分得的财产多于二儿子,二儿子分得的财产多于小儿子。恰好借来一只动物后可以分好并余下一只动物。如果你会编程,可以用计算机枚举出所有解。但此题要求手工找出所有可能的$n, a, b, c$。提示:本题本质是求方程$\dfrac{n}{n+1} = \dfrac{1}{a} + \dfrac{1}{b} + \dfrac{1}{c}, a < b < c$的所有正整数解。三儿子至少分得一只动物,那么$n$的最小值是多少?你能据此推出$a$不可能超过3,因而只能是2么?\label{qn:three-sons}}

\Question{$\bigstar$由于$0.999\cdots = 1$,所以任何有限小数有两种表示:$X.a_1a_2 \cdots a_m000\cdots$和$X.a_1a_2 \cdots b999\cdots$,其中$b = a_m - 1$,$X$是整数部分。证明除了这种情况外,任何数值$x$的小数表示是唯一的,其中$0 < x < 1$。}

\Question{证明无限循环小数$0.\dot{a_1} \dot{a_2} \cdots \dot{a_n}$的分数形式是$\dfrac{a_1 a_2 \cdots a_n}{99 \cdots 9} = \dfrac{a_1 a_2 \cdots a_n}{10^{n+1} - 1}$。\label{qn:cyclic-decimal}}

\Question{利用分数和引理\ref{th:cycle-of-9}解释:如果阿塔兰塔先跑完全程的$\dfrac{1}{2}$,接下来再跑完剩余的一半,即$\dfrac{1}{4}$,然后不断跑完剩余路程的一半,则她最终可以跑完全程。提示:考虑二进制分数和。\label{qn:dichotomy-paradox}}

\Question{验证分数的加法交换律、乘法交换律,乘法结合率。\label{qn:arithmatic-law-fraction}}
\end{Exercise}

\begin{Answer}[ref={ex:fractions}]
\Question{根据调和平均值的原始定义:

\begin{align*}
\frac{1}{h} - \frac{1}{a} &= \frac{1}{b} - \frac{1}{h}    \\
\frac{2}{h} &= \frac{1}{a} + \frac{1}{b} \\
 h &= \frac{2}{\frac{1}{a} + \frac{1}{b}} = \frac{2ab}{a + b}
\end{align*}

令两地间距离为$s$,往返的时间分别为:往$t_1 = \dfrac{s}{v_1}$、返$t_2 = \dfrac{s}{v_2}$。平均速度为:
\begin{align*}
  v & = \frac{2s}{t_1 + t_2} = \frac{2\cancel{s}}{\frac{\cancel{s}}{v_1} + \frac{\cancel{s}}{v_2}}
    = \frac{2}{\frac{1}{v_1} + \frac{1}{v_2}} = \frac{2v_1 v_2}{v_1 + v_2}
\end{align*}

恰好是$v_1$和$v_2$的调和平均值。另外,根据往返总时间等于去的时间加返回时间:$\dfrac{2s}{v} = \dfrac{s}{v_1} + \dfrac{s}{v_2}$,两边约去路程$s$就得到$\dfrac{2}{v} = \dfrac{1}{v_1} + \dfrac{1}{v_2}$。恰好也是上边调和平均值$\dfrac{2}{h} = \dfrac{1}{a} + \dfrac{1}{b}$的形式。

\begin{center}
 \includegraphics[scale=0.4]{img/harmonic-meanr}
 \captionof{figure}{电阻并联}
 \label{fig:harmonic-meanr}
\end{center}

如\cref{fig:harmonic-meanr}所示,左右电路等效时,加载同样的电压,流过的总电流相等,即:

\begin{align*}
\frac{I}{2} + \frac{I}{2} &= I_1 + I_2 \\
2\frac{\cancel{V}}{R} &= \frac{\cancel{V}}{R_1} + \frac{\cancel{V}}{R_2} \\
\frac{2}{R} &= \frac{1}{R_1} + \frac{1}{R_2} \\
R &= \frac{2}{\frac{1}{R_1} + \frac{1}{R_2}} = \frac{2R_1 R_2}{R_1 + R_2}
\end{align*}
所以图中右侧每个电阻等于$R_1$和$R_2$的调和平均值。
}

\Question{验证$\dfrac{2}{p} = \dfrac{1}{a} + \dfrac{1}{b}$等价于$(2a - p)(2b - p) = p^2$,并由此证明:若$p$为大于2的素数,则$\dfrac{2}{p}$可唯一分解为两个埃及分数$\dfrac{2}{p} = \dfrac{1}{a} + \dfrac{1}{b} (a \ne b)$。

  \begin{proof}
  \begin{align*}
  \frac{2}{p} & = \frac{1}{a} + \frac{1}{b} = \frac{a+b}{ab} & \text{通分} \\
  2ab & = pa + pb & p, a, b\text{是分母,不为}0 \\
  4ab - 2pa - 2pb & = 0 & \text{移项} \times 2 \\
  4ab - 2pa - 2pb + p^2 & = p^2 & \text{两边} + p^2 \\
  (2a - p)(2b - p) & = p^2 & \text{因式分解}
  \end{align*}
  因为$p$是素数,$p^2$的因子只有1、$p$、$p^2$。由$a \ne b$可以排除掉$p$,所以$2a - p = 1$,$2b - p = p^2$,即:
  \[
  a = \frac{p+1}{2}, b = \frac{p(p+1)}{2}
  \]
  因为$p$是奇素数,所以$a$是整数。因此:
  \[
  \frac{2}{p} = \frac{1}{\frac{1}{2}(p + 1)} + \frac{1}{\frac{1}{2}p(p+1)} \qedhere
  \]
  \end{proof}
}

\Question{证明埃尔德什——斯特劳斯猜想对所有偶数$n$成立。

\vspace{2mm}
令$n = 2m$,则$\dfrac{4}{2m} = \dfrac{2}{m}$,这就是\cref{th:decompose-of-2-n}。
}

\Question{实际上我们只要证明埃尔德什——斯特劳斯猜想对于所有素数成立即可。这是为什么?

\vspace{2mm}

注意到:\[
\frac{4}{mp} = \frac{1}{ma} + \frac{1}{mb} + \frac{1}{mc}
\]
}

\Question{证明埃尔德什——斯特劳斯猜想对所有$4k+3$型的素数成立。
  \begin{proof}
    利用\cref{qn:unique-decompose-r}的结论
    \begin{align*}
      \frac{2}{p} &= \frac{1}{\frac{1}{2}(p + 1)} + \frac{1}{\frac{1}{2}p(p + 1)} && \\
      \frac{4}{p} &= \frac{2}{\frac{1}{2}(p + 1)} + \frac{2}{\frac{1}{2}p(p + 1)} & \text{左右} \times 2 & \\
      \frac{4}{4k + 3} &= \frac{2}{\frac{1}{2}(4k + 4)} + \frac{2}{\frac{1}{2}(4k + 3)(4k + 4)} & \text{带入} p = 4k + 3 & \\
                       &= \frac{1}{k + 1} + \frac{1}{(4k + 3)(k + 1)} & & \\
                       &= \frac{1}{k + 2} + \frac{1}{(k + 1)(k + 2)} + \frac{1}{(4k + 3)(k + 1)} & \text{由\cref{th:egyptian-fraction-split}} & \qedhere
    \end{align*}
  \end{proof}
}

\Question{找出三个儿子按照$\dfrac{1}{a}$、$\dfrac{1}{b}$、$\dfrac{1}{c}$分配$n$个动物的所有可能数值和比例。要求大儿子分得的财产多于二儿子,二儿子分得的财产多于小儿子。恰好借来一只动物后可以分好并余下一只动物。

\vspace{2mm}
我们先给出一个纯数学解法,然后再给出用计算机穷举的解法。读者朋友们可以加以对比。本题本质上是求方程
\[
\frac{n}{n+1} = \frac{1}{a} + \frac{1}{b} + \frac{1}{c}, a < b < c
\]
的所有正整数解。即借来一只后,把$n+1$只动物分成三份$\dfrac{n + 1}{a}$、$\dfrac{n + 1}{b}$、$\dfrac{n + 1}{c}$。它们的和恰好是$n$,因而可以把剩余的那一只归还。为了简洁,我们令$n' = n + 1$,表示借来一只动物后的数量,把上述方程改写为:

\be
\frac{n'-1}{n'} = \frac{1}{a} + \frac{1}{b} + \frac{1}{c}
\label{eq:equation-of-inherit}
\ee

最后只要把$n'$减一就可以了。借来一只动物后能恰好分配,意味着$a, b, c$都能整除$n'$。我们先证明一个重要的结论:大儿子只能分得$\dfrac{1}{2}$的财产,即$a$只能等于2。
\begin{proof}
若$a \geq 3$,由于$a < b < c$,所以:

\[
\frac{1}{a} + \frac{1}{b} + \frac{1}{c} \leq \frac{1}{3} + \frac{1}{4} + \frac{1}{5} = \frac{47}{60}
\]

小儿子至少分得1只;二儿子分得的比小儿子多,所以至少分得2只;同理,大儿子至少分得3只动物,再加上借来的一只。所以$n' > 3 + 2 + 1 = 6$。于是:

\[
\frac{n'-1}{n'} = 1 - \frac{1}{n'} > 1 - \frac{1}{6} = \frac{5}{6} = \frac{50}{60} > \frac{47}{60} \geq \frac{1}{a} + \frac{1}{b} + \frac{1}{c}
\]

也就是说$a \geq 3$时,方程\cref{eq:equation-of-inherit}不可能成立。所以$a$只能是2。
\end{proof}

问题于是就化简为,求方程
\[
\frac{n'-1}{n'} = \frac{1}{2} + \frac{1}{b} + \frac{1}{c}
\]
的正整数解。我们接下来证明$b$只能是3和4,从而进一步化简问题:

\begin{proof}
如果$b \geq 5$,由于$b < c$,故:

\[
\frac{1}{a} + \frac{1}{b} + \frac{1}{c} \leq \frac{1}{2} + \frac{1}{5} + \frac{1}{6} = \frac{13}{15} \approx 0.87
\]

由于$a, b, c$都能整除$n'$,而$a = 2$,所以$n'$必然是偶数。我们上面推出$n' > 6$,比6大的第一个偶数是8。因此:

\[
\frac{n'-1}{n'} = 1 - \frac{1}{n'} \geq 1 - \frac{1}{8} = \frac{7}{8} = 0.875 > 0.87 \geq \frac{1}{a} + \frac{1}{b} + \frac{1}{c}
\]

也就是说$b \geq 5$时,方程\cref{eq:equation-of-inherit}不可能成立。所以$b$只能是3或4。
\end{proof}

\begin{enumerate}[情况1)]
\item $a = 2, b = 3$
  \begin{align*}
    \frac{n'-1}{n'} &= \frac{1}{2} + \frac{1}{3} + \frac{1}{c} \\
    1 - \frac{1}{n'} &= \frac{5}{6} + \frac{1}{c} \\
    \frac{1}{6} &= \frac{1}{n'} + \frac{1}{c}
  \end{align*}
  由于$a = 2, b = 3$都整除$n'$,所以6整除$n'$。不妨设$n' = 6k$,其中$k$为大于1的整数(因为我们前面确定了$n'$最小是8)。代入上式,整理得:
  \[
  k = \frac{c}{c - 6} \geq 2
  \]
  解不等式$\dfrac{c}{c - 6} \geq 2$得$c \leq 12$。同时分母$c-6$是正数,所以$c > 6$。这样$c$的值只可能是$7, 8, 9, \cancel{10}, \cancel{11}, 12$。其中10,11不可能,因为此时$k = \dfrac{c}{c - 6}$不是整数。我们得到4个解:
  \begin{align*}
  a &= 2, b = 3, c = 7, n' = 42 & 42\frac{1}{2} + 42\frac{1}{3} + 42\frac{1}{7} &= 21 + 14 + 6 = 41 \\
  a &= 2, b = 3, c = 8, n' = 24 & 24\frac{1}{2} + 24\frac{1}{3} + 24\frac{1}{8} &= 12 + 8 + 3 = 23 \\
  a &= 2, b = 3, c = 9, n' = 18 & 18\frac{1}{2} + 18\frac{1}{3} + 18\frac{1}{9} &= 9 + 6 + 2 = 17 \\
  a &= 2, b = 3, c = 12, n' = 12 & 12\frac{1}{2} + 12\frac{1}{3} + 12\frac{1}{12} &= 6 + 4 + 1 = 11
  \end{align*}
\item $a = 2, b = 4$
  \begin{align*}
    \frac{n'-1}{n'} &= \frac{1}{2} + \frac{1}{4} + \frac{1}{c} \\
    1 - \frac{1}{n'} &= \frac{3}{4} + \frac{1}{c} \\
    \frac{1}{4} &= \frac{1}{n'} + \frac{1}{c}
  \end{align*}
  由于$a = 2, b = 4$都整除$n'$,所以4整除$n'$。不妨设$n' = 4k$,其中$k \geq 2 $为整数(同理$n'$最小是8)。代入上式,整理得:
  \[
  k = \frac{c}{c - 4} \geq 2
  \]
  解不等式$\dfrac{c}{c - 2} \geq 2$得$c \leq 8$。同时分母$c-4$是正数,所以$c > 4$。这样$c$的值只可能是$5, 6, \cancel{7}, 8$。其中7不可能,因为此时$k = \dfrac{c}{c - 4}$不是整数。我们得到3个解:
  \begin{align*}
  a &= 2, b = 4, c = 5, n' = 20 & 20\frac{1}{2} + 20\frac{1}{4} + 20\frac{1}{5} &= 10 + 5 + 4 = 19 \\
  a &= 2, b = 4, c = 6, n' = 12 & 12\frac{1}{2} + 12\frac{1}{4} + 12\frac{1}{6} &= 6 + 3 + 2 = 11 \\
  a &= 2, b = 4, c = 8, n' = 8 & 8\frac{1}{2} + 8\frac{1}{4} + 8\frac{1}{8} &= 4 + 2 + 1 = 7
  \end{align*}
\end{enumerate}
注意到两种情况都有分11+1只动物的可能,但分配比例不同。总共有7个解。

\vspace{3mm}
用计算机求解时,可以让机器枚举所有可能的$a, b, c$。判断其倒数和是否满足$\dfrac{n + 1}{n}$的形式。显然,$a$从2开始,依次尝试2、3、4……并且由于$a < b < c$,所以我们从$a+1$开始枚举$b$,从$b + 1$开始枚举$c$。接下来我们需要估计$a, b, c$的一个上限,使得枚举的过程可以结束。因为$a < b < c$,所以$c$的上限一定也是$a, b$的上限。此外,三儿子最少分到1只动物,所以$c \leq n'$。从\cref{eq:equation-of-inherit}可以推出:

\begin{align*}
  1 - \frac{1}{n'} &= \frac{1}{a} + \frac{1}{b} + \frac{1}{c} \\
  1 - \frac{1}{n'} - \frac{1}{c} &= \frac{1}{a} + \frac{1}{b} \leq \frac{1}{2} + \frac{1}{3} = \frac{5}{6} \\
  \frac{1}{6} &\leq \frac{1}{n'} + \frac{1}{c} \leq \frac{1}{c} + \frac{1}{c} & \text{三儿子至少分得一只,故} \frac{1}{n'} \leq \frac{1}{c} \\
  \frac{1}{6} &\leq \frac{2}{c} \Rightarrow c \leq 12
\end{align*}

因此枚举的上限是12。其次从$1 - \dfrac{1}{n'} = \dfrac{1}{a} + \dfrac{1}{b} + \dfrac{1}{c}$还可以解出$n' = \dfrac{abc}{abc - ab - bc - ac}$,如果其分母不为0,且$a, b, c$都整除$n'$,则找到了一组解。\cref{fig:inherit-program}给出了例子程序。

\begin{center}
 \includegraphics[scale=0.4]{img/inherit-program}
 \captionof{figure}{三子继承问题的Scratch程序}
 \label{fig:inherit-program}
\end{center}

下面是对应的Python和Haskell例子程序。

\begin{lstlisting}[language = Python, frame = single]
s = []
for a in range(2, 11):
    for b in range(a + 1, 12):
        for c in range(b + 1, 13):
            d = a*b*c - a*b - b*c - a*c
            if d > 0:
                n = a*b*c // d
                if n % a == 0 and n % d == 0 and n % c == 0:
                    s.append((a, b, c))
print(s)
\end{lstlisting}

\begin{Haskell}[frame = single]
[(a, b, c, n) | a <- [2..10], b <- [a+1..11], c <- [b+1..12],
     let d = a*b*c - a*b - b*c - a*c, d > 0,
     let n = a*b*c `div` d, n > 1,
         n `mod` a == 0, n `mod` b == 0, n `mod` c == 0]
\end{Haskell}
}

\Question{$\bigstar$证明任何数值$x$的小数表示是唯一的,其中$0 < x < 1$。

  \begin{proof}
    用反证法,令:
\be
\frac{a_1}{10} + \frac{a_2}{100} + \cdots = \frac{b_1}{10} + \frac{b_2}{100} + \cdots
\label{eq:decimal-a-eq-b}
\ee
设$a_n$和$b_n$是第一对不相等的。$|a_n - b_n| \geq 1$。于是:
\begin{align*}
\frac{a_1}{10} + \frac{a_2}{100} + \cdots - (\frac{b_1}{10} + \frac{b_2}{100} + \cdots) & \geq \frac{1}{10^n} - (\frac{|a_{n+1} - b_{n+1}|}{10^{n+1}} + \frac{|a_{n+2} - b_{n+2}|}{10^{n+2}} + \cdots) \\
  & \geq \frac{1}{10^n} - (\frac{9}{10^{n+1}} + \frac{9}{10^{n+2}} + \cdots) \\
  & = \frac{1}{10^n} - \frac{1}{10^n}(0.99\cdots) = 0
\end{align*}

这与\cref{eq:decimal-a-eq-b}矛盾。
  \end{proof}
}

\Question{证明无限循环小数$0.\dot{a_1} \dot{a_2} \cdots \dot{a_n}$的分数形式是$\dfrac{a_1 a_2 \cdots a_n}{99 \cdots 9} = \dfrac{a_1 a_2 \cdots a_n}{10^{n+1} - 1}$。

  \begin{proof}
    \begin{align*}
    \text{设} x &= 0.\dot{a_1} \dot{a_2} \cdots \dot{a_n} \\
    10^n x & = a_1a_2\cdots a_n . \dot{a_1} \dot{a_2} \cdots \dot{a_n} \\
    10^n x - x &= a_1a_2\cdots a_n \\
    x &= \frac{a_1a_2\cdots a_n}{10^n - 1} = \frac{a_1a_2\cdots a_n}{99 \cdots 9} && \qedhere
    \end{align*}
  \end{proof}
}

\Question{利用分数和解释:如果阿塔兰塔先跑完全程的$\dfrac{1}{2}$,接下来再跑完剩余的一半,即$\dfrac{1}{4}$,然后不断跑完剩余路程的一半,则她最终可以跑完全程。

\vspace{2mm}
十进制无限循环小数$0.99\cdots = 1$,对应的二进制无限循环小数$0.11\cdots = 1$。证明与十进制类似:
\begin{proof}
  \begin{align*}
    x &= 0.11\cdots & 2x &= 1.11\cdots \\
    x & = 2x - x = 1 & & \qedhere
  \end{align*}
\end{proof}
阿塔兰塔先跑到$\dfrac{1}{2}$处,即二进制0.1处;再跑到$\dfrac{1}{4}$处,即二进制0.01处……她跑过的路程和等于二进制小数和:
\[
0.1 + 0.01 + 0.001 + \cdots = 0.11\cdots = 1
\]
}

\Question{验证分数的加法交换律、乘法交换律,乘法结合率。

\vspace{2mm}

加法交换律:
\begin{align*}
  \frac{b}{a} + \frac{d}{c} &= \frac{bc + ad}{ac} &\text{通分} \\
  &= \frac{ad + bc}{ac} & \text{分子用整数加法交换律} \\
  &= \frac{ad}{ac} + \frac{bc}{ac} = \frac{d}{c} + \frac{b}{a}
\end{align*}

乘法交换律:
\[
\frac{b}{a} \times \frac{d}{c} = \frac{bd}{ac} = \frac{db}{ca} = \frac{d}{c} \times \frac{b}{a}
\]

乘法结合律:
\[
\frac{b}{a} \times \frac{d}{c} \times \frac{f}{e} = \frac{bdf}{ace} = \frac{b(df)}{a(ce)} = \frac{b}{a}\frac{df}{cd} = \frac{b}{a} \times (\frac{d}{c} \times \frac{f}{e})
\]
}
\end{Answer}

\ifx\wholebook\relax \else
\section{Answer}
\shipoutAnswer

\section{埃及分数分解算法}
\subimport{inc/}{decomposition-en}

\begin{thebibliography}{99}
\subimport{inc/}{bib-en}
\end{thebibliography}

\expandafter\enddocument
\fi
