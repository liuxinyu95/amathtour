\ifx\wholebook\relax \else

\documentclass[b5paper]{article}
\usepackage[nomarginpar
  %, margin=.5in
]{geometry}

\addtolength{\oddsidemargin}{-0.05in}
\addtolength{\evensidemargin}{-0.05in}
\addtolength{\textwidth}{0.1in}

\usepackage[en]{../prelude}

\setcounter{page}{1}

\begin{document}

\title{Ancient time}

\author{Xinyu LIU
\thanks{{\bfseries Xinyu LIU} \newline
  Email: liuxinyu99@hotmail.com \newline}
  }

\maketitle
\fi

\markboth{Ancient time}{A tour of numbers}

\ifx\wholebook\relax
\chapter{Ancient time}
\numberwithin{Exercise}{chapter}
\fi

%% \epigraph{Numbers are the highest degree of knowledge. It is knowledge itself.}{Plato}


\ifx\wholebook\relax \else
\section{Answer}
\shipoutAnswer

\begin{thebibliography}{99}

%% \bibitem{wiki-number}
%% Wikipedia. ``History of ancient numeral systems''. \url{https://en.wikipedia.org/wiki/History_of_ancient_numeral_systems}

%% \bibitem{Calvin-Clawson-1994}
%% Calvin C Clawson. ``The Mathematical Traveler, Exploring the Grand History of Numbers''. Springer. 1994, ISBN: 9780306446450

%% \bibitem{wiki-babylonian-num}
%% Wikipedia. ``Babylonian numerals''. \url{https://en.wikipedia.org/wiki/Babylonian_numerals}

\end{thebibliography}

\expandafter\enddocument
%\end{document}

\fi
