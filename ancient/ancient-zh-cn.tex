\ifx\wholebook\relax \else

\documentclass[b5paper]{ctexart}
\usepackage[nomarginpar
  %, margin=.5in
]{geometry}

\addtolength{\oddsidemargin}{-0.05in}
\addtolength{\evensidemargin}{-0.05in}
\addtolength{\textwidth}{0.1in}

\usepackage[cn]{../prelude}

\setcounter{page}{1}

\begin{document}

\title{数的诞生}

\author{刘新宇
\thanks{{\bfseries 刘新宇} \newline
  Email: liuxinyu99@hotmail.com \newline}
  }

\maketitle
\fi

\markboth{数的诞生}{数的旅程}

\ifx\wholebook\relax
\chapter{数的诞生}
\numberwithin{Exercise}{chapter}
\fi

\epigraph{一二三四五,金木水火土。\\
天地分上下,日月照今古。}{部编小学一年级\\
语文课本第一课}

数充满了我们的生活。例如报纸上这段新闻报道:“2024年巴黎奥运会(第33届奥运会)已于当地时间2024年8月1日闭幕。巴黎是继伦敦后的世界第2个至少3次举办夏奥会的城市。这是首届男女比例完全平衡的奥运会,男女运动员各为5250名。本届奥运会共设有32个大项,329个小项,共有206个国家和地区参赛,新增了滑板、冲浪、竞技攀岩和霹雳舞四个大项。中国代表团最终在巴黎奥运会上夺得40金27银24铜的优异成绩。”

这短短的180字中有14个数字。数是谁发明的?历史书上没有答案。数出现在所有历史文字记录中,数也许诞生在史前,伴随着语言和文字。要找到答案,我们有两条线索:1、追寻古老的历史物证,石刻、壁画、器物上关于数的印记;2、追溯数在语言演变中的痕迹。比如英文中的eleven (11)来自古英语endleofan,意思是(数到10还)剩余1;tweleve (12)来自twelf,意思是剩余2。

\section{罗塞塔石碑}
\index{罗塞塔石碑}

\begin{figure}[htbp]
 \centering
 \includegraphics[scale=1.0]{img/Rosetta-stone-recons}
 \caption{古埃及罗赛塔石碑。石碑不完整,从中间断裂。}
 \label{fig:rosetta-stone-recons}
\end{figure}

走进大英博物馆第4展厅,有一件展品吸引着观众。这是一块残缺的石碑,长114厘米,宽72厘米,上面刻满了文字(\cref{fig:rosetta-stone-recons})。人们能一眼辨认出位于底部的内容是希腊字母(\cref{fig:rosetta-greek},参见\cref{ch:greek-letters}希腊字母表),但上面的部分犹如天书。仔细观察会发现余下的文字大致分成两种:一种是弯弯曲曲的符号,如\cref{fig:rosetta-demotic},位于石碑中部;另一种是奇妙的图案,如\cref{fig:rosetta-hieroglyphs},位于石碑上部。

\begin{figure}[htbp]
  \centering
  \subcaptionbox{古希腊文,共54行\label{fig:rosetta-greek}}{\includegraphics[scale=0.5]{img/Rosetta-Greek}} \\
  \subcaptionbox{古埃及世俗文字,是当时埃及平民使用的文字。共32行\label{fig:rosetta-demotic}}{\includegraphics[scale=0.5]{img/Rosetta-Demotic}} \\
  \subcaptionbox{古埃及象形文字,又称圣书体,代表献给神明的文字。共14行\label{fig:rosetta-hieroglyphs}}{\includegraphics[scale=0.3]{img/Rosetta-Hieroglyphs-c}}
  \caption{罗赛塔石碑上的三种文字}
  \label{fig:rosetta-stone}
 \end{figure}

这一展品叫做罗塞塔石碑。1799年,拿破仑率领法军远征埃及。远征军有3万余人,各类舰只350艘,另有学者工程师146人。这位未来的法兰西皇帝从一名炮兵军官脱颖而出。他敏锐地看出数学不仅可以算出大炮的弹道,还关乎法兰西的国运。拿破仑的大军巧妙地避开了英国海军的封锁,7月2日攻占了埃及的亚历山大城,随即向开罗进军。7月15日,一位士兵在尼罗河三角洲前线的小镇罗塞塔(Rosetta)挖掘防御工事。他偶然发现了一块断碑砌在一段古老的墙中。尽管并不完整,这仍是一个重大的发现。法军指挥官决定把它送到拿破仑设立于开罗的埃及研究所,并于8月运抵开罗。根据发现地点,它被命名为罗塞塔石碑。1801年,英军击败了拿破仑,罗塞塔石碑也落入了英军之手。1802年2月,它被运抵英国朴次茅斯港,并最终收藏于大英博物馆。

这块石碑有何特殊之处呢?它上面用三种不同的文字记录了同一内容:公元前196年,13岁的国王托勒密5世加冕一周年。他从父亲托勒密4世袭得正统王位,并做了许多善行,如捐助神庙、减免税收等等。埃及当时处于托勒密王朝的统治之下,统治者是希腊人\footnote{马其顿国王亚历山大大帝征服了埃及。他死后,埃及总督托勒密一世与公元前305年自称国王建立王朝,统治埃及275年。}。国王宣称自己是法老。因此铭文使用了古埃及象形文字、古埃及世俗文字、希腊官方文字三种文字镌刻,并颁布全埃及各个神庙勒石立碑。而罗塞塔石碑就是其中之一,并且是至今唯一发现的一块。罗塞塔石碑于是成为了破解古埃及文字的钥匙。英国物理学家托马斯·杨(1773~1829,就是高中物理课本中杨氏双缝实验——发现光的干涉现象的物理学家)和法国学者商博良通过研究此碑,成功破译了古埃及象形文字\cite{BM-RS-17}。

\section{古埃及和古巴比伦的计数系统}
\index{计数系统}

通过对古文字的破译,我们发现古埃及大约在公元前3400年就出现了表示数字的符号。是众多古文明中最早的(美索不达米亚大约在公元前3000年出现了数字符号,中国大约在公元前1600年出现了数字符号)。各文明中最早出现的数字符号是多是“|”、“||”、“||”或“-”、“=”、“$\equiv$”。这表明我们的老祖先是从“数数”开始认识数字的。很可能是采集、狩猎到更多的东西。随着生产生活的进步,人们逐渐认识更大的数,并逐渐从一数到了十——显然我们的祖先搬着手指头数数。但接下来遇到了困难。人只有两只手十个手指。这时有三种解决方法:(1)把脚也用上可以数到二十,但接下来会遇到同样的问题。(2)自由分组。在西伯利亚的尤卡吉尔语中有这样的例子:三和一、两个三、两个四、十差一等等。(3)固定分组。数到某一固定数目,例如十,分成一组,把这个组看成一个单位并起个名字。比如古埃及用$\cap$表示十。这样$\cap \cap$表示二十(古罗马用X表示十,I表示一。23表示为XXIII)。

\begin{figure}[htbp]
 \centering
 \includegraphics[scale=0.8]{img/hieroglyphic-numbers}
 \caption{古埃及象形文字中的数字符号。}
 \label{fig:egypt-hieroglyphic-numerals}
\end{figure}

为了建造宏伟的金字塔,古埃及人还创造了更大的单位符号,如\cref{fig:egypt-hieroglyphic-numerals}。他们用永恒之神(Heh)代表一百万,用蝌蚪表示十万,用弯曲的手指表示一万,莲花表示一千、弯曲的绳子表示一百……计数时按照单位分组,超出后就组成更大的单位。最后把每个单位的数目和单位表示的大小乘起来。\cref{fig:egypt-number-examples}给出了两个例子。

\begin{figure}[htbp]
 \centering
 \subcaptionbox{数字214427表示为$2(100000)+1(10000)+4(1000)+4(100)+2(10)+7(1)$}{\includegraphics[scale=0.5]{img/egypt-num-eg2}}
 \subcaptionbox{古埃及埃德富(Edfu)神庙中的象形文字数字。表示1333330}{\includegraphics[scale=0.2]{img/Hieroglyphic-num-eg}}
 \caption{古埃及象形文字数字的例子}
 \label{fig:egypt-number-examples}
\end{figure}

尽管这种方法能够处理很大的数,但方法(2)自由分组对付小数字也很方便。古巴比伦人使用60进制(分组单位是60)。这种进制今天也出现在我们的日常生活中。60秒是1分,60分是1小时。所以1小时12分30秒,或写成1:12:30,包括$1(60\times 60) + 12(60) + 30 = 4350$秒。60对古巴比伦人足够大,所以他们在60以内使用自由分组法,超过60用$60^2$、$60^3$、$60^4$……分组。\cref{fig:babylonian-numerals}是古巴比伦楔形文字60以内的数字符号。古巴比伦人生活在美索不达比亚平原,在两河流域(位于底格里斯河、幼发拉底河之间的平原,现今的伊拉克境内)。他们在湿泥板上刻划文字,然后在阳光下晒干或用火烘干变硬。这些文字呈楔形。可以明显看出其中的规律:一是一个纵向的楔形,从一到九每数一就增加一个纵向的楔形。十变成了一个横向的楔形,从十一到十九是一个横向的楔形(10)加上相应纵向的楔形。二十是两个横向的楔形,然后重复这个规律。所以60以内的数字是横向的楔形个数乘以10加上纵向的楔形个数。

\begin{figure}[htbp]
 \centering
 \subcaptionbox{古巴比伦楔形文字中的数字符号\label{fig:babylonian-numerals}}{\includegraphics[scale=1.3]{img/Babylonian-numerals}} \\
 \subcaptionbox{古巴比伦楔形数字$1(60^3) + 19(60^2) + 21(60) + 54$\label{fig:babylonian-num-eg}}{\includegraphics[scale=0.25]{img/Babylonian-num-eg}}
 \caption{古巴比伦数字}
\end{figure}

但60以上的数字展现出不一样的规律,如\cref{fig:babylonian-num-eg}所示。285714的60进制表示是1:19:21:54,但19写成了$20-1$。其中较小的纵向楔形表示减法,减数的上方有一个向右的楔形。可以看出,这并非是我们现代意义上严格的60进制,而是一种混合进制——60以内用十进制或减法表示。和古埃及相比,古巴比伦没有为$60$、$60^2$、$60^3$……创建单独的符号,而是依赖数字所在的位置决定单位的大小。最左边的数字表示有多少个1,向右的第二个位置上的数字表示有多少个60,第三个位置上的数字表示有多少个$60^2$,以此类推。这种方法叫做“位值制系统”。但由于缺乏0,古巴比伦的数字有歧义。直到约公元前300年后才偶尔出现了0,但只用于数字的中间而非结尾,无法区分出11和1100.古文字学者至今仍面临这个问题,只能通过上下文推测数值。

\section{古罗马和中国计数系统}
\index{乘法分组计数系统}

随着古罗马文明的崛起并建立起横跨欧亚非的帝国,罗马计数系统产生了巨大的影响,直到今天我们仍然可以在钟表盘上、建筑上、著作的章节编号上看到罗马数字(\cref{fig:clock-plate})。罗马数字的优点是只用少数几个符号,如I、V、X、C、M就够了。对于较小的数字很直观,例如III表示3,VII表示5 + 2 = 7,2025可表示为MMXXV(2个1000,2个10加5)。但是罗马数字的缺点是用法混乱。早期的罗马数字只包含加法,后来出现了“左减右加”的用法,如IV表示5 - 1 = 4,而VI表示5 + 1 = 6。这样IXX = (10 - 1) + 10 = 19,但19 = 10 + (10 - 1) = XIX就有歧义了:XIX还可以理解为(XI)X = 11 + 10 = 21。同样XVIII = 16,但16 = 8 + 8 = IIXXIIX和16 = (10 - 4) + 10 = IIIIXX都有歧义。其中IIIIXX出现在1388年巴黎的一份协定中来表示88\cite{LeVeque-Smith-25}。

\btab{|c|c|c|c|c|}
\hline
1 & 5 & 10 & 100 & 1000 \\
\hline
I & V & X  & C   & M \\
\hline
\etab

\begin{figure}[htbp]
 \centering
 \includegraphics[scale=0.4]{img/clock-plate}
 \caption{钟表盘上的罗马数字}
 \label{fig:clock-plate}
\end{figure}

在古代中国产生了非常接近现代计数系统的十进制“乘法分组”法。华夏文明的前10个数字符号是:

\btab{cccccccccc}
一 & 二 & 三 & 四 & 五 & 六 & 七 & 八 & 九 & 十 \\
\etab

接下来中文没有出现英文中eleven(剩一)和twelve(剩二)的问题,而是直接进展到十一、十二……十九、二十……九十九。数字较小时,也使用廿(20)和卅(30),通常用在日历中。然后增加了百、千、万的分组单位。例如6147写为六千一百四十七,表示(六千)(一百)(四十)(七),即6个千、1个百、4个十、7个一,或6千1百4十7个,换成罗马单位相当于6M1C4X7I。每个数字$d_i$乘以单位的大小$u_i$,然后加到一起$d_nu_n + \cdots + d_1u_1$就是数值。这看起来和我们日常生活中熟悉的数字几乎一样了,可是几乎毕竟是几乎,究竟还差了一些。比如六千一百七,究竟是6170还是6107?汉字中不是有“零”么?可是在公元十世纪前的古代中国,零从来没有用在数学上。零是一个形声字,雨表形,令表声。本意是下雨。例如《诗经·豳风·东山》:我来自东,零雨其濛。引申意是落下,例如《诗经·郑风·野有蔓草》:野有蔓草,零露漙(tu\'{a}n)兮。《楚辞·离骚》:惟草木之零落兮,恐美人之迟暮。在可能成书于东汉时期的《九章算术》中,没有出现任何含有汉字零的数字。即使在宋元以后引入了零,但用法并不一致。例如103可以写成一百零三,也可以写成一百有三。《水浒传》里的好汉数目常读作一百单八将。在没有零的情况下,要想避免歧义就必须明确每个数字所代表的单位。这就需要给不同大小的分组单位命名或创造符号。随着数字的增大就需要更多的单位。但数的增加是无限的,而符号的数目是有限的,早晚有用光的时候。下表列出了中文和英文中越来越大的单位名称。

\begin{center}
% for Pinyin tones: \={a}, \'{a}, \v{}, \.{a}
\begin{tabular}{|l|r|l|r|l|r|}
\hline
百            & $100$      & 秭(z\v{i})    & $10^{24}$ &  \textbf{恒河沙}  & $10^{52}$ \\
\hline
千            & $1000$     & 穰(r\'{a}ng)  & $10^{28}$ & 阿僧祗(zh\={i})  & $10^{56}$ \\
\hline
万            & $10000$    & 沟            & $10^{32}$ & 那由他        & $10^{60}$  \\
\hline
亿            & $10^8$     & 涧            & $10^{36}$ &  不可思议      & $10^{64}$ \\
\hline
兆            & $10^{12}$  & 正            & $10^{40}$ &  无量大数      & $10^{68}$ \\
\hline
京            & $10^{16}$  & 载            & $10^{44}$ &               & \\
\hline
垓(g\={a}i)   & $10^{20}$  & 极            & $10^{48}$ &               & \\
\hline
\end{tabular}
\end{center}

可以看到,汉语中这些大单位词汇,有许多来自佛教。包括恒河沙,它表示1后面跟着52个0。英语中的大单位如下表。从一开始,每增加一千倍就有一个对应的单位。万进位和千进位的不同,也是文化上的一种差异。

\begin{center}
\begin{tabular}{|l|r|l|r|l|r|}
\hline
thousand & $10^{3}$ & quattuordecillion & $10^{45}$ & octovigintillion & $10^{87}$ \\
\hline
million & $10^{6}$ & quindecillion & $10^{48}$ & novemvigintillion & $10^{90}$ \\
\hline
billion & $10^{9}$ & sexdecillion & $10^{51}$ & trigintillion & $10^{93}$ \\
\hline
trillion  & $10^{12}$ & septdecillion & $10^{54}$ & untrigintillion & $10^{96}$ \\
\hline
quadrillion  & $10^{15}$ & octodecillion & $10^{57}$ & duotrigintillion & $10^{99}$ \\
\hline
quintillion  & $10^{18}$ & novemdecillion & $10^{60}$ & \textbf{googol} & $10^{100}$ \\
\hline
sexillion    & $10^{21}$ & vigintillion & $10^{63}$ & & \\
\hline
septillion   & $10^{24}$ & unvigintillion & $10^{66}$ & & \\
\hline
octillion    & $10^{27}$ & duovigintillion & $10^{69}$ & & \\
\hline
noniliion  & $10^{30}$ & trevigintillion & $10^{72}$ & & \\
\hline
decillion  & $10^{33}$ & quattuorvigintillion & $10^{75}$ & & \\
\hline
undecillion   & $10^{36}$ & quinvigintillion & $10^{78}$ & & \\
\hline
duodecillion  & $10^{39}$ & sexvigintillion & $10^{81}$ & & \\
\hline
tredecillion  & $10^{42}$ & seprvigintillion & $10^{84}$ & & \\
\hline
\end{tabular}
\end{center}

表中最后一个大单位古格尔(googol)是在1920年由9岁的米尔顿$\cdot$西洛塔(Milton Sirotta)想出的名字。这个数字是1后面跟着100个零。著名的互联网公司谷歌的名字就来自它。

\section{位值制计数系统}
\index{位值制计数系统}

在大航海时代,西班牙探险者在中美洲的尤卡坦半岛(位于墨西哥湾和加勒比海之间)发现玛雅文明创造出了完美的计数系统。\cref{fig:maya-numerals}给出了玛雅数字。类似古巴比伦的10-60混合进制,玛雅人使用5-20混合进制。20以内采用5进制,每数1点一个点,每数到5划一横线。20以上的数竖着写,并规范使用0做占位符,如\cref{fig:maya-numerals}。玛雅人的计数系统和玛雅历法相互影响。以地球围绕太阳旋转一周作为一个太阳年,每月20天。玛雅人很快发现$20 \times 18 = 360$,很接近一年。于是规定1年18个月,在加上5个禁忌日,一年恰好是365天。每4年再加上1天。这和我们现今太阳年的天数几乎一样。

\begin{figure}[htbp]
 \centering
 \subcaptionbox{20以内采用5进制}{\includegraphics[scale=1.0]{img/Maya-numerals}}
 \subcaptionbox{$2(20^3) + 6(20) + 13 = 16133$}{\includegraphics[scale=0.4]{img/Maya-num-eg}}
 \caption{玛雅数字}
 \label{fig:maya-numerals}
\end{figure}

无论是玛雅的20进制还是古巴比伦的60进制,都不需要额外命名单位,而是通过数字所在位置赋予大小的意义。这种方法叫做位值制计数系统(positional numeral system),归纳起来有3个要求:

\begin{enumerate}
\item 确定一个进制$b$\footnote{英文base的首字母}。如我们今天日常使用的$b = 10$、古巴比伦使用的$b = 60$、玛雅使用的$b = 20$。
\item 给1、2……$b-1$这些数字命名。例如汉语的一、二、三……九;英语的one, two, three, ..., nine;玛雅的点、点点……四点三划。
\item 代表0的符号。
\end{enumerate}

这样任何一个数$n$等于:

\be
n = a_m b^m + a_{m-1} b^{m-1} + \cdots + a_a b + a_0
\ee

其中$a_i$是被命名的数字,包括0、1、2……$b-1$。$n$写为$a_ma_{m-1} \cdots a_0$。例如$2025 = 2 \times 10^3 + 0 \times 10^2 + 2 \times 10 + 5$, 其中$b = 10, m = 3, a_3 = 2, a_2 = 0, a_1 = 2, a_0 = 5$,恰好也写为2025。我们再看两个例子:

\begin{example}
猫咪王国中,每只猫爪只有4个可动的指(\cref{fig:cat-paw}),左右共8个可动的指。如果采用8进制,并规定0的符号为“喵”,1到7为:(1)苗、(2)秒、(3)妙、(4)咪、(5)谜、(6)米、(7)蜜。王国一年的天数$365 = 5 \times 64 + 5 \times 8 + 5$,写成“谜谜谜”。

\begin{figure}[htbp]
 \centering
 \includegraphics[scale=0.8]{img/cat-paw}
 \caption{猫爪}
 \label{fig:cat-paw}
\end{figure}

\end{example}

\begin{example}
我国古代用天干地支纪年。天干有10个符号:甲、乙、丙、丁、戊、己、庚、辛、壬、癸;地支有12个符号:子、丑、寅、卯、辰、巳、午、未、申、酉、戌、亥。组合起来可以记录60以内的数字,从甲子开始,接下来天干地支各向前进一到已丑,然后是丙寅……到癸亥结束。比如2025年是农历乙巳年。乙的上一个天干符号是甲,巳的上一个地支符号是辰,所以前一年2024年是甲辰年,后一年2026年是丙午年。尽管我们不说“丙天午地”,看似用位置表示(第一个符号是天干、第二个符号是地支),但干支纪年\textbf{不是位值制计数法}。例如丙午并非一个甲子中第$3 \times 10 + 7 = 37$年,而是第43年。
\end{example}

\ifx\wholebook\relax \else
\section{参考答案}
\shipoutAnswer

\begin{thebibliography}{99}

\bibitem{BM-RS-17}
British Museum. ``Everything you ever wanted to know about the Rosetta Stone''. British Museum blog, 14 July. 2017, \url{https://www.britishmuseum.org/blog/everything-you-ever-wanted-know-about-rosetta-stone}. Accessed 14 February 2025.

\bibitem{LeVeque-Smith-25}
LeVeque, William Judson, Smith, David Eugene. ``numerals and numeral systems''. Encyclopedia Britannica, 7 Jan. 2025, \url{https://www.britannica.com/science/numeral}. Accessed 14 February 2025.

%% \bibitem{wiki-number}
%% Wikipedia. ``古代计数系统的历史''. \url{https://en.wikipedia.org/wiki/History_of_ancient_numeral_systems}

%% \bibitem{trip-to-number-kingdom}
%% [美]\ 卡尔文$\cdot$C$\cdot$克劳森\ 著\ 袁向东、袁钧\ 译. ``数学旅行家:漫游数王国''. 上海教育出版社。ISBN: 7-5320-7883-3/G $\cdot$ 7972

%% \bibitem{wiki-babylonian-num}
%% Wikipedia. ``古巴比伦数字''. \url{https://en.wikipedia.org/wiki/Babylonian_numerals}

\end{thebibliography}

\expandafter\enddocument
%\end{document}

\fi
